\addcontentsline{toc}{chapter}{Abstract}
\begin{abstract}
  Proton-proton collision data recorded by the ATLAS detector at CERN's Large
  Hadron Collider, in which protons collide at a centre of mass energy of
  13~\TeV, is used study the process called VH(bb). The VH(bb) analysis aims to
  measure events in which a Higgs boson is produced in association with a vector
  boson and also in which that Higgs boson decays to a pair of b-quarks. Both
  the VH production mechanism and the Hbb decay channel were first measured in
  2018 at ATLAS and CMS. Subsequent measurements of the VH(bb) process aim to
  further our understanding of the Higgs by providing more precise measurements
  of the VH signal strength as well as the individual WH and ZH signal
  strengths, further testing the predictions of The Standard Model. The analysis
  uses a number of by-hand categorisations as well as a multi-variate classifier
  to obtain distributions that enter into a profile-likelihood fit. The results
  of the analysis performed using the full run 2 dataset (140~\invfb recorded
  between 2015--2018) agree with the standard model predictions for signal
  strength and cross-section in a number of fiducial regions.
\end{abstract}
