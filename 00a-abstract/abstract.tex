\addcontentsline{toc}{chapter}{Abstract}
\begin{abstract}

  Proton-proton collision data recorded by the ATLAS detector at CERN's Large
  Hadron Collider, in which protons collide at a centre of mass energy of 13
  TeV, is used study the process called VH(bb). The VH(bb) resolved analysis
  aims to measure events in which a Higgs boson is produced in association with
  a vector boson and also in which that Higgs boson decays to a pair of
  b-quarks. The resolved version of the analysis is concerned with events in
  which jets initiated by the b-quarks can be resolved from one another in the
  detector. Both the VH production mechanism and the Hbb decay channel were
  discovered in 2018 at ATLAS and CMS. Subsequent measurements of the VH(bb)
  process aim to further our understanding of the Higgs by providing more
  precise measurements of the VH signal strength as well as using methods to
  obtain the WH and ZH signal strengths, The Standard Model can be tested by
  comparing these values with its predictions. Many improvements have been made
  to the analysis since the previous publication including a method that uses
  machine learning to compare multi-dimensional distributions in order to
  evaluate modelling uncertainties. The analysis strategy will be outlined
  highlighting a number of the aforementioned improvements along the way. The
  expected results of the upcoming publication using the full run 2 dataset (140
  fb$^{-1}$ recorded between 2012 - 2018) will be shown.
  
% This report will detail the motivation for measuring the Higgs boson decaying to
% a pair of b quarks. First the ATLAS detector will described in enough detail
% such that the capabilities and limitations of that hardware can be referred to
% when discussing the analysis. Then the physics theory which predicts the decay
% in question, the standard model of particle physics, is described including a
% short review of physical phenomena that are measured to exist but not included
% within the model. Following these elementary details the strategy for the main
% analysis is laid out. Due to the fact that the current iteration of this
% analysis is now finished details given will be those most relevant to possible
% future improvements of the analysis. Finally the results will be shown of work
% that has been completed on quantifying the uncertainty of a specific systematic
% error associated with the single-top background.
\end{abstract}
