\addcontentsline{toc}{chapter}{Abstract}
\begin{abstract}
  Proton-proton collisions with centre of mass energy of 13~\TeV\ are analysed
  in order to measure the \VHbb\ process. These collisions are provided by the
  Large Hadron Collider and measured by the ATLAS detector both of which are
  based at CERN. The \VHbb\ process occurs when a Higgs boson is produced in
  association with a vector boson and decays to a pair of $b$-quarks.
  Measurement of the \VHbb\ process serves the purpose of furthering our
  understanding of the Higgs by providing more precise measurements of the \VH\
  signal strength as well as the individual \WH\ and \ZH\ signal strengths,
  testing the predictions of The Standard Model of Particle Physics. The
  analysis uses a number of by-hand categorisations as well as a multi-variate
  classifier to obtain distributions that enter into a profile-likelihood fit.
  Systematic uncertainties are considered in the fit as nuisance parameters. The
  results of the analysis performed using the full run 2 dataset (140~fb$^{-1}$
  recorded between 2015--2018) agree with predictions for signal strength and
  cross-section in a number of fiducial regions.
\end{abstract}
