\chapter{Introduction}%
\label{ch:intro}

In 2012 the Higgs boson~\cite{Brout-Englert, Higgs:1964, Kibble} was discovered
by the ATLAS and CMS collaborations at the Large Hadron
Collider~\cite{DiscoHiggsATLAS, DiscoHiggsCMS}. It was said to form the last
piece of the Standard Model of Particle Physics~\cite{Glashow1959, Salam:1968,
Weinberg:1967, GellMann:1962}, a framework that describes three of the four
fundamental forces of nature. Despite the theory being apparently complete, it
does not describe gravity, the fourth of the known fundamental forces of nature.
The theory also has other shortcomings, it cannot explain the presence of dark
matter~\cite{DM-ev-sloan, DM-ev-nucleosynth, DM-ev-supernova, DM-ev-scaffold,
DM-ev-direct, DM-ev-strong-lens, DM-ev-candidates, DM-ev-PDG, DM-ev-Zwicky,
DM-ev-nonbaryonic, DM-ev-particle} or a number of other observed
phenomena~\cite{anom-BD-branching, anom-Dtau-excess, anom-g-2,
anom-proton-radius, anom-bsll-trans}. So far the model has stood up to all
experimental tests~\cite{EWtests, 1998-SMtests} concerning its own predictions
but there are still parameters of the model that have not been measured. Given
the theory's understood shortcomings, it is hoped that continued scrutiny of the
models predictions will yield unexpected results, perhaps hinting at a new way
forward in terms of a theory that describes all matter and forces in the
universe or simply exposing further gaps in our knowledge. For this reason it is
more important than ever to study in detail the most recently discovered piece
of the model, the Higgs boson.

This work focuses on studying Higgs boson particles that decay into two bottom
quarks and are produced alongside a vector boson, denoted \VHbb. This decay mode
is of importance as it is currently the only decay mode of the Higgs decaying to
quarks that has been observed~\cite{vhbb-obs}.

The data used to study this decay mode was collected using the ATLAS detector by
members of the ATLAS collaboration. The collection of this data is only made
possible by their hard work and by the hard work of everyone working on the
Large Hadron Collider at CERN.

A rough blueprint of this analysis is shown in figure~\ref{fig:roadmap}.%
\begin{figure}[h]
  \centering
  %Define block styles
  \tikzstyle{decision} = [diamond, draw, fill=blue!20, 
  text width=4.5em, text badly centered, node distance=3cm, inner sep=0pt]
  \tikzstyle{block} = [rectangle, draw, fill=blue!20, text width=12em, text
  centered, rounded corners, minimum height=3em]
  \tikzstyle{line} = [draw, -latex']
  \tikzstyle{cloud} = [fill=none, node distance=3cm, minimum
  height=2em]
  
    
  \begin{tikzpicture}[node distance = 2cm, auto]
    % Place nodes
    \node[cloud] (theory) {\includegraphics[width=.25\textwidth]{hand_written_lagrangian-1
      }};
    
    \node [cloud, right=1.5cm of theory] (detector)  {\includegraphics[width=.25\textwidth]{atlas_big}};

    
    % Draw edges 
    \path [line] (theory) -- (detector);
    
  \end{tikzpicture}
  \caption{A flow chart showing the roadmap of the VHbb analysis and of this thesis.}
  \label{fig:cxaod-flow}
\end{figure}
 The blueprint starts with the theory of
the Standard Model of Particle Physics, described in chapter~\ref{ch:theory}.
The theory's predictions inform the design of the ATLAS detector which is
detailed in chapter~\ref{ch:detector}. Furthermore, the theory aids the choice
of particles to collide, which events to analyse, and allows us to generate
predictions of what should happen in the collisions. Chapter~\ref{ch:ml} gives
an overview of two machine learning algorithms that are used throughout the
analysis. Events must be reconstructed before they can be analysed, a selection
process is used on the reconstructed objects to filter events not relevant to
the analysis. The reconstruction and selection process is described in
chapter~\ref{ch:recon}. Events in data must be categorised into those that are
signal-like and those that are background-like in order to extract the most
signal sensitivity from the analysis. This is achieved with a number of
strategies including a multi-variate algorithm which is trained on the
simulations.

This categorisation process is described in chapter~\ref{ch:strategy}, along
with the overall analysis strategy including the details of the
profile-likelihood fit. A choice is made of simulated events that well model the
data. Considering the modelling, the theory and the shortcomings of the detector
a set of systematic uncertainties are estimated, these are detailed
in~\ref{ch:systematics}. The results of the analysis are shown in
chapter~\ref{ch:results} and conclusions are drawn in
chapter~\ref{ch:conclusion}.

My contributions to the \VHbb\ analysis published in 2021 have been numerous and
spanned a number of different areas of the analysis. At the beginning of each
chapter I will aim to make clear which content represents my work and which is
that of my colleagues. In short, my main contributions have been to study and
estimate systematic uncertainties as well as the behaviour of the
profile-likelihood fit. Additionally I was responsible with maintaining and
operating the analysis software framework.

