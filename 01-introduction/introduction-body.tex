\chapter{Introduction}%
\label{sec:intro}

In 2012 the Higgs boson was discovered by the ATLAS and CMS collaborations at
the Large Hadron Collider~\cite{DiscoHiggsATLAS, DiscoHiggsCMS}. It was said to
form the last piece of the Standard Model of Particle Physics, a framework that
describes three of the four fundamental forces of nature, described in more
detail in Chapter~\ref{sec:theory}. Despite apparent completeness after the
Higgs discovery, it is known that the theory does not describe gravity, the
fourth of the known fundamental forces of nature. The theory also has other
shortcomings, it cannot explain the presence of dark matter~\cite{DM-ev-sloan,
  DM-ev-nucleosynth, DM-ev-supernova, DM-ev-scaffold, DM-ev-direct,
  DM-ev-strong-lens, DM-ev-candidates, DM-ev-PDG, DM-ev-Zwicky,
  DM-ev-nonbaryonic, DM-ev-particle} or a number of other observed
phenomena~\cite{anom-BD-branching, anom-Dtau-excess, anom-g-2,
  anom-proton-radius, anom-bsll-trans}. So far the model has stood up to all
experimental tests~\cite{EWtests, 1998-SMtests} concerning its own predictions
but there are still parameters of the model that have not been measured. Given
the theory's understood shortcomings, it is hoped that continued scrutiny of the
models predictions will yield unexpected results, perhaps hinting at a new
way forwards in terms of a theory that describes everything or simply exposing
further gaps in our knowledge of the universe. For this reason it is more
important than ever to study in detail the most recently discovered piece of the
model, the Higgs boson.

This work focuses on studying a specific production mechanism and decay mode of
the Higgs boson, specifically a vector boson associated Higgs boson decaying to
two bottom quarks, denoted $VH(bb)$. This decay mode is of importance as it is
currently the only decay mode of the Higgs decaying to quarks that has been
observed~\cite{vhbb-obs}. A summary of the full spectrum of production
mechanisms and decay modes of the Higgs will be given in
Chapter~\ref{sec:theory}.

The study of this decay mode was carried out with the ATLAS detector, and made
possible by the hard work of all members of the ATLAS collaboration. In
Chapter~\ref{sec:detector} the detector is described in full.

Paragraph describing my contributions:
Derivation of Z + jets modelling uncertainties (shapes, extrapolation, flavour
composition)
 - improvements include: changing the b-tagging, using top e mu control region
 data to subtract top backgrounds, removing signal contamination from PTV region
Derivation of ttbar/single top flavour composition uncertainties
Maintenance of CxAOD Framework, general fixes etc.
Tested the BDT-reweighting code used for W+jets and ttbar backgrounds in 1
lepton channel, wrote code that prepared reco samples for training (originally
everyone was training on truth samples but I was looking at reco, everyone moved
to reco and used my code to convert reco tuples to a readable format by the
training code)
Training the analysis MVA for the diboson cross-check
Studying the analysis fit pre/post lots of different changes (b-tagging, using
top e-mu control region data etc.) including running the final unblinded fit for
the diboson cross-check.
