\chapter{Introduction}%
\label{sec:intro}

In 2012 the Higgs boson was discovered by the ATLAS and CMS collaborations at
the Large Hadron Collider~\cite{DiscoHiggsATLAS, DiscoHiggsCMS}. It was said to
form the last piece of the Standard Model of Particle Physics, a framework that
describes three of the four fundamental forces of nature, described in more
detail in Chapter~\ref{sec:theory}. Despite apparent completeness after the
Higgs discovery, it is known that the theory does not describe gravity, the
fourth of the known fundamental forces of nature. The theory also has other
shortcomings, it cannot explain the presence of dark matter~\cite{DM-ev-sloan,
  DM-ev-nucleosynth, DM-ev-supernova, DM-ev-scaffold, DM-ev-direct,
  DM-ev-strong-lens, DM-ev-candidates, DM-ev-PDG, DM-ev-Zwicky,
  DM-ev-nonbaryonic, DM-ev-particle} or a number of other observed
phenomena~\cite{anom-BD-branching, anom-Dtau-excess, anom-g-2,
  anom-proton-radius, anom-bsll-trans}. So far the model has stood up to all
experimental tests~\cite{EWtests, 1998-SMtests} concerning its own predictions
but there are still parameters of the model that have not been measured. Given
the theory's understood shortcomings, it is hoped that continued scrutiny of the
models predictions will yield unexpected results, perhaps hinting at a new
way forwards in terms of a theory that describes everything or simply exposing
further gaps in our knowledge of the universe. For this reason it is more
important than ever to study in detail the most recently discovered piece of the
model, the Higgs boson.

This work focuses on studying a specific production mechanism and decay mode of
the Higgs boson, specifically a vector boson associated Higgs boson decaying to
two bottom quarks, denoted $VH(bb)$. This decay mode is of importance as it is
currently the only decay mode of the Higgs decaying to quarks that has been
observed~\cite{vhbb-obs}. A summary of the full spectrum of production
mechanisms and decay modes of the Higgs will be given in
Chapter~\ref{sec:theory}.

The study of this decay mode was carried out with the ATLAS detector, and made
possible by the hard work of all members of the ATLAS collaboration. In
Chapter~\ref{sec:detector} the detector is described in full.

My contributions to the VHbb analysis published in 2021 have been numerous and
spanned a number of different areas of the analysis.

My largest contribution was the derivation of new estimates of background
modelling uncertainties. This includes deriving new shape uncertainty estimates
for the Z + jets background in the 0 and 2 lepton channels which use a method
which includes a number of improvements over the previous analysis (explained in
detail in chapter~\ref{sec:modelling}). I also calculated estimates of
uncertainties arising due to differences in the flavour composition and yields
in given analysis regions between simulation and data for the Z + jets and top
backgrounds. I helped to develop and test code which performs a
multi--dimensional re--weighting of one dataset to another. This technique has
been used to estimate uncertainties of the top backgrounds and W + jets
backgrounds in the 1 lepton channel.

My next largest contribution was to studying the profile likelihood fit that
outputs the final analysis results. During key milestones of the analysis it is
often necessary to compare the blinded results of the fit to study the behaviour
of nuisance parameters and how they are pulled, constrained and correlated with
one another. I performed many of these comparisons paying close attention to the
nuisance parameters relating to the aforementioned estimates on background
modelling uncertainties. After approval was granted to unblind the analysis I
was also responsible for running the final fit for the diboson measurement which
serves as a cross-check of the all of the analysis methodology used for the
Higgs measurement.

Finally I have also made contributions to the general running of the analysis.
These including training the classification algorithm which provides the final
discriminating metric upon which the profile-likelihood fit is performed,
helping to maintain the analysis framework (which is also used by many other
ATLAS analyses) and participating in regular meetings with the analysis team
where ideas are discussed and the general analysis direction is decided. 

