\begin{figure}[ht]
  \centering
  \begin{tikzpicture}
    \tkzDefPoint(0,0){O}
    \tkzDefPoint(2,-2){A}
    \tkzDefPoint(3.5,0){Bfield}
    \tkzLabelPoint[right](Bfield){{\Large$\otimes$}$\vec{B}$}
    \tkzLabelPoint[below right](A){First hit}
    \tkzDefPointBy[rotation= center O angle 90](A)
    \tkzGetPoint{B}
    \tkzLabelPoint[above right](B){Last hit}
   
    \tkzDefPointBy[rotation=center O angle -10](A)
    \tkzGetPoint{trackstart}
    \tkzDefPointBy[rotation=center O angle 100](A)
    \tkzGetPoint{trackend}
    \tkzDrawLines[add=0 and 0](A,B)
    \tkzDrawArc[color=blue, ->](O,trackstart)(trackend)
    \tkzLabelLine[left, pos=0.25](A,B){$L$}
    \tkzDefMidPoint(A,B) \tkzGetPoint{D}
    \tkzDrawLines[add=0 and 0](O,D)
    
    \tkzDefPointBy[rotation=center O angle 45](A)
    \tkzGetPoint{E}
    \tkzDrawLines[add=0 and 0, color=red](D,E)
    \tkzLabelLine[above](D,E){$S$}
    \tkzDrawLines[add = 0 and 0](O,A O,B)
    \tkzDefPointBy[rotation=center O angle 10](A)
    \tkzGetPoint{h1}
    \tkzDefPointBy[rotation=center O angle 20](A)
    \tkzGetPoint{h2}
    \tkzDefPointBy[rotation=center O angle 30](A)
    \tkzGetPoint{h3}
    \tkzDefPointBy[rotation=center O angle 40](A)
    \tkzGetPoint{h4}
    \tkzDefPointBy[rotation=center O angle 50](A)
    \tkzGetPoint{h5}
    \tkzDefPointBy[rotation=center O angle 60](A)
    \tkzGetPoint{h6}
    \tkzDefPointBy[rotation=center O angle 70](A)
    \tkzGetPoint{h7}
    \tkzDefPointBy[rotation=center O angle 80](A)
    \tkzGetPoint{h8}
    
    \tkzDrawPoints[cross out](A,B,h1,h2,h3,h4,h5,h6,h7,h8)
    \tkzDrawPoints(trackstart)
    \tkzLabelPoint[left](trackstart){Interaction vertex}
    \tkzDefTangent[at=A](O)
    \tkzGetPoint{h}
    \tkzMarkRightAngle[size=0.2](O,A,h)
    \tkzDefTangent[at=B](O)
    \tkzGetPoint{h}
    \tkzDefPointOnLine[pos=-1](B,h)
    \tkzGetPoint{h}
    \tkzMarkRightAngle[size=0.2](O,B,h)
  \end{tikzpicture}
  \caption[The geometric construction of the sagitta.]{A diagram showing the
    geometric construction of the sagitta ($S$) of a track. The track is
    comprised of a sequence of hits marked by crosses and the ``lever-arm''
    distance between the first and last hit in the track is marked $L$. Charged
    tracks in a magnetic field pointing into the page, as shown, form arcs of
    circles. The two lines marked as normal to the track are radii of the circle
    to which this track's arc belongs.}
  \label{fig:sagitta}
\end{figure}
