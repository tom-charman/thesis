\begin{figure}[ht]
  \centering
  \tikzset{
    treenode/.style = {align=center, inner sep=0pt, text centered,
      font=\sffamily},
    arn_n/.style = {treenode, circle, white, font=\sffamily\bfseries\small, draw=black,
      fill=black, text width=3em},% arbre rouge noir, noeud noir
    arn_r/.style = {treenode, circle, red, font=\small, draw=red, 
      text width=2em, very thick},% arbre rouge noir, noeud rouge
    arn_b/.style = {treenode, circle, blue, font=\small, draw=blue, 
      text width=2em, very thick},% arbre rouge noir, noeud rouge
    sq_r/.style = {treenode, red, font=\small, draw=red, 
      minimum width=2em, minimum height=2em, very thick},% arbre rouge noir, noeud rouge
    sq_b/.style = {treenode, blue, font=\small, draw=blue, 
      minimum width=2em, minimum height=2em, very thick},% arbre rouge noir, noeud rouge
    arn_x/.style = {treenode, rectangle, draw=black,
      minimum width=2em, minimum height=2em}% arbre rouge noir, nil
  }
  \begin{tikzpicture}[->,>=stealth',level/.style={sibling distance = 7cm/#1,
      level distance = 2cm}] 
    \node [arn_n] {100}
    child{ node [arn_r] {60} 
      child{ node [sq_r] {39} 
        }
      child{ node [arn_b] {21}
        child{ node [sq_r] {4}}
        child{ node [sq_b] {17}}
      }                            
    }
    child{ node [arn_b] {40}
      child{ node [sq_r] {15} 
      }
      child{ node [sq_b] {25}
      }
		}; 
  \end{tikzpicture}
  \caption[The structure of a decision tree.]{The structure of a decision tree
    set up for a classification problem. This diagram serves as a summary of
    where 100 examples end up after being passed through the tree. The number in
    each node corresponds to the number of examples that pass through that
    node.}
  \label{fig:bdt}
\end{figure}

