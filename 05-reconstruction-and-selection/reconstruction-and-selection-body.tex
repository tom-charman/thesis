\chapter{Reconstruction and Selection}%
\label{ch:recon}

At the top level events are required to have one Higgs boson candidate and one
vector boson candidate. In all cases a Higgs boson candidate is comprised of two
b-tagged jets. More on the jet collection and b-tagging strategy used in
section~\ref{sec:jets}. Vector boson candidates are characterised by a number of
different decay products defined in section~\ref{sec:lepton}, these decay
products trigger the recording of events, specific triggers used are discussed
in section~\ref{sec:triggers}. Reconstruction of basic quantities as well as
higher level algorithms such as overlap removal are handled by Athena and the
CxAOD Framework, more on these in section~\ref{sec:cxaod}. Finally with all
quantities reconstructed events are categorised in different analysis regions,
these are described in section~\ref{sec:ana-regions}.

My contributions to the reconstruction and selection include maintenance of the
CxAOD Framework, running production campaigns to produce new datasets when
impactful changes have been made further up the analysis chain, and implementing
new systematic uncertainties in the CxAOD Framework (not discussed until
chapter~\ref{ch:systematics}).

\section{Athena and the CxAOD Framework}
\label{sec:cxaod}
Data recorded by the ATLAS detector is passed through the central collaboration
software framework Athena before entering the analysis level data processing.
\begin{figure}[ht]
  \centering
  %Define block styles
  \tikzstyle{decision} = [diamond, draw, fill=blue!20, 
  text width=4.5em, text badly centered, node distance=3cm, inner sep=0pt]
  \tikzstyle{block} = [rectangle, draw, fill=blue!20, text width=12em, text
  centered, rounded corners, minimum height=3em]
  \tikzstyle{line} = [draw, -latex']
  \tikzstyle{cloud} = [draw, ellipse,fill=red!20, node distance=3cm, minimum
  height=2em]
  
    
  \begin{tikzpicture}[node distance = 2cm, auto]
    % Place nodes
    \node [block] (gen) {Generation};
    
    \node[below=0.75cm of gen] (gen-half) {};
    \node[right=0.75cm of gen-half] (evnt) {EVNT};
    
    \node [block, below=1.5cm of gen] (sim) {Simulation};

    \node[below=0.75cm of sim] (sim-half) {};
    \node[right=0.75cm of sim-half] (hits) {HITS};
    
    \node [block, below=1.5cm of sim] (digi) {Digitisation};

    \node[below=0.75cm of digi] (digi-half) {};
    \node[right=0.75cm of digi-half] (rdo) {RDO};
    
    \node [block, below=1.5cm of digi] (mc-reco) {Reconstruction};

    \node[below=0.75cm of mc-reco] (mc-reco-half) {};
    \node[right=0.75cm of mc-reco-half] (xaod) {xAOD};
    
    \node [block, below=1.5cm of mc-reco] (mc-der) {Derivation};

    \node[below=0.75cm of mc-der] (mc-der-half) {};
    \node[right=0.75cm of mc-der-half] (dxaod) {DxAOD}; 
    
    \node [block, left=3cm of digi, fill=red!20] (trig) {Trigger};

    \node[below=0.75cm of trig] (trig-half) {};
    \node[left=0.75cm of trig-half] (dxaod) {RAW};
    
    \node [block, below=1.5cm of trig, fill=red!20] (data-reco) {Reconstruction};

    \node[below=0.75cm of data-reco] (data-reco-half) {};
    \node[left=0.75cm of data-reco-half] (xaod) {xAOD};
    
    \node [block, below=1.5cm of data-reco, fill=red!20] (data-der) {Derivation};

    \node[below=0.75cm of data-der] (data-der-half) {};
    \node[left=0.75cm of data-der-half] (dxaod) {DxAOD};

    \node[left=1.5cm of mc-der] (ghost) {};
    \node [block, below=1.5cm of ghost, fill=purple!40] (ana) {Analysis};
    
    % Draw edges
    \path [line] (gen) -- (sim);
    \path [line] (sim) -- (digi);o
    \path [line] (digi) -- (mc-reco);
    \path [line] (mc-reco) -- (mc-der);

    \path [line] (trig) -- (data-reco);
    \path [line] (data-reco) -- (data-der);

    \path [line] (mc-der) -- (ana);
    \path [line] (data-der) -- (ana);
  \end{tikzpicture}
  \caption[Data flow central to ATLAS.]{A flow chart showing the Athena data
    processing chain. Red nodes indicate the presence of data recorded from
    collisions and blue nodes indicate the presence of Monte-Carlo simulated
    events.}
  \label{fig:data-flow}
\end{figure}

Athena is responsible for the steps shown in figure~\ref{fig:data-flow}. As can
be seen in the figure Athena processes both data recorded from collisions and
Monte-Carlo simulated predictions. Steps up until and including reconstruction
are required to transform the raw or simulated read-out of the detector into
what are known as physics objects. These physics objects correspond to
ultra-violet and infra-red safe descriptions particles and hadron showers e.g.
leptons and jets. Given the initial transverse energy of the collisions (zero)
any missing transverse energy ($E^T_{miss}$) is also reconstructed based on the
sum of the transverse energy of all objects in an event, this missing energy
indicates the presence of particles in the event that cannot be detected by any
of the ATLAS subsystems. The only particles in the Standard Model for which this
is expected are neutrinos. The files containing the reconstructed physics objects
adhere to the ATLAS Event Data Model and are referred to as Analysis Object Data
(xAOD). After reconstruction a part of Athena called the derivation framework is
used to produce skimmed and slimmed xAODs known as Derived xAODs (DxAODs). The
reduction of these files is carried out based on a loose selection criteria.

DxAODs are the usual starting point for analysis level software, in the case of
this analysis the CxAOD Framework. As in in figure~\ref{fig:cxaod-flow}
\input{05-reconstruction-and-selection/cxaod-flow}
The CxAOD Framework has two major modules the Maker and the Reader. The job of
the Maker is to further slim the data by performing pre-selection cuts and also
to apply calibrations which will be detailed below, the output of the Maker is
called a Calibrated xAOD (CxAOD). The Reader takes a CxAOD as input and performs
the analysis event selection, it can output histograms or $N$-tuples.

The Maker applies the selections of each of the three analysis channels, defined
in section~\ref{sec:selection}. CxAODs are produced separately for each of the
three channels as the different background compositions and signal signatures
require different optimisation. Pre-selection is performed on jets based on
requirements of transverse momentum and pseudo-rapidity. A tool known as the Jet
Vertex Tagger (JVT) is used to remove jets resulting from pileup from events.

\section{Leptons}%
\label{sec:lepton}

The channels of the $VH \to b\bar{b}$ analysis are defined by the number of
observed charged leptons ($e$ or $\mu$) in the decay of the vector boson. There
is one channel for the study of $WH \to b\bar{b}$ decays where the leptonic
decay $W \rightarrow \ell\nu$ yields a single charged lepton, the 1--lepton
channel. There are two channels for the study of $ZH \to b\bar{b}$ decays, the
0--lepton channel where $Z \rightarrow \nu\nu$, and the 2--lepton channel where
$Z \rightarrow \ell\ell$.

Two classifications of lepton are defined in order to categorise events into the
individual channels of the analysis, these are called $VH$-loose and $VH$-signal
leptons, channels are defined as disjoint sets by requiring different numbers of
both lepton categories. These classifications are defined in
table~\ref{tab:vh-leptons}.
\begin{table}
  \centering
  \resizebox{\textwidth}{!}{%
  \begin{tabular}{ l l l l l l l}
    \toprule
    \bfseries{Name} & $\bm{p_T}$ &  $\bm{\lvert \eta \rvert}$ & \bfseries{ID} & $\bm{d_0^{sig}}$ & $\bm{\lvert \Delta z_0\sin{\theta} \rvert}$
    & \bfseries{Isolation}\\
    \midrule
    \multicolumn{7}{l}{\bfseries{electrons}}\\
    VH-loose & > 7 \GeV  & < 2.47 & LH Loose & < 5 & < 0.5 mm & FCLoose \\
    ZH-signal & > 27 \GeV & < 2.47 & LH Loose & < 5 & < 0.5 mm & FCLoose \\
    WH-signal & > 27 \GeV & < 2.47 & LH Tight & < 5 & < 0.5 mm & FixedCutHighPtCaloOnly \\
    &&\\
    \multicolumn{7}{l}{\bfseries{muons}}\\
    VH-loose & > 7 \GeV & < 2.7 & Loose quality & < 3 & < 0.5 mm & FixecdCutLoose \\
    ZH-signal & > 27 \GeV & < 2.5 & Loose quality & < 3 & < 0.5 mm & FixecdCutLoose \\
 WH-signal & > 27 \GeV & < 2.5 & Medium quality & < 3 & < 0.5 mm & FixedCutHighPtCaloOnly \\
    \bottomrule
  \end{tabular}
  }
  \caption{Definitions of the VH leptons used to define and select events for
    the three analysis channels where $d_0^{sig}$ is the measured with respect to
    the beam line.}
  \label{tab:vh-leptons}
\end{table}
The characteristics of the fake lepton background from QCD multi-jet processes
differs between the 1-- and 2--lepton channels hence the reason for two
different categorisations. In general to suppress this kind of background
leptons are required to be isolated from other detector activity.

\subsection{Electrons}
\label{subsec:electrons}

As mentioned in chapter~\ref{ch:detector} electrons leave tracks in the ID and
energy deposits in the ECAL. Reconstructing electrons requires clustering the
energy deposits in the ECAL, this is achieved with a sliding window
algorithm~\cite{Delmastro:1747242}. Clusters must then be associated to tracks
in the ID, a Gaussian Sum Filter~\cite{ATLAS-CONF-2012-047} is used to account
for energy loses due to bremsstrahlung radiation. The energy for electron
candidates must be calibrated before it can be used in order to account for
things such as non-uniformity in the detector response. Calibration is achieved
by using simulated cluster activity from single particles to train a BDT
regression model designed to regress the measured energy in the ECAL to the
simulated energy. An in-situ data driven correction is applied to normalise the
response between data and simulation~\cite{ATL-PHYS-PUB-2016-015}.

Reconstruction alone is not enough to find electrons, other particles may leave
similar signatures in the ATLAS sub-detectors and therefore electron
identification must also be performed. Identification is performed using a
likelihood-based method. Variables which have power to discriminate between
electrons and other particles are used in the likelihood such as shower
profiles, track quality, how closely track and cluster positions match in $\eta$
and $\phi$, and the presence of a high-threshold TRT hit. This is one of the
main benefits of the TRT. Performance of this method is well
studied~\cite{ATLAS-CONF-2016-024,EgammaEffTWiki}.

\subsection{Muons}

Finding muons in the detector requires consideration of the coverage of the
different ATLAS sub-detectors, especially in general are not stopped in the
detector. Muons leave charged tracks in the ID and the muon spectrometers which
have coverages of $\lvert \eta \rvert < 2.7$ and $\lvert \eta \rvert < 2.5$
respectively. For the region $\lvert \eta \rvert > 2.5$ a stand-alone algorithm
which doesn't use ID tracks can be used. All muons within the coverage of the ID
require good quality ID tracks~\cite{muonTWiki, MuonSelectionToolTwiki}. A
combined algorithm is used in the majority of cases. For the region $\lvert
\eta  \rvert < 0.1$ two specialised algorithms SegmentTagged and CaloTagged are
used which require only muon segment and calorimeter deposits respectively. All
aforementioned algorithms are used together in what is known as a unified
chain~\cite{Aad:2014rra,MuonChainTWiki,ATL-PHYS-PUB-2015-037} to reconstruct and
identify muons.

\subsection{Taus}

As mentioned the only charged leptons that are considered in the analysis are
electrons and muons, leptonically decaying taus will have electrons and muons as
the only visible decay products however hadronically decaying taus must be
considered differently. Decays are considered as one or three pronged based on
the number of charged decay products, pions, with neutrinos a
nd neutral pions
also present. These decays are reconstructed in the calorimeters like jets with
the anti-$k_t$ algorithm with $\Delta R = 0.4$~\cite{ATL-PHYS-PUB-2015-045} but
the $p_T$ of the tau is set to the total energy of the TopoClusters within
$\Delta I goR < 0.2$, more on TopoClusters in section~\ref{sec:jets}. Tau
candidates must have $p_T > 20 \GeV$, $\lvert  \eta \rvert < 2.5$ excluding
$1.37 \lvert \eta \rvert < 1.52$, and either exactly 1 or 3 tracks. A BDT based
method for tau identification is used to reject fakes. Medium quality taus are
counted for each event~\cite{TauRecommendation2015,TauRecommendation2016}.

\section{Triggers}
\label{sec:triggers}

As stated at the beginning of this chapter the decay products of the vector
boson candidate are used to trigger the recording of events for this analysis.
Therefore important triggers for the 0--lepton channel are $E_T^{miss}$ triggers,
and for the 1--lepton and 2--lepton channels the single electron, or single muon
triggers. Note that it is not necessary to trigger on both charged leptons
coming from the Z boson in the 2--lepton channel, the presence of one lepton
allows the triggering to occur and the requirement for 2--leptons can be imposed
at a later stage. Some events will be missed by not using a di-lepton trigger,
however these amount to approximately only 5\% of the total. The list of
triggers used as they appear in the ATLAS trigger menu are shown in
table~\ref{tab:triggers}, for the $E_T^{miss}$, electron and muon triggers
respectively.
\begin{table}[ht]
  \begin{center}
    \resizebox{0.95\textwidth}{!}{
      \begin{tabular}{ l l l p{7cm} }
        \toprule
        \textbf{ Trigger Name } & \textbf{Period} & \textbf{Threshold (GeV)} & \textbf{Description} \\
        \midrule
        \multicolumn{4}{l}{\bfseries{$\bm{E_T^{miss}}$}}\\
        HLT\_xe70\_L1XE50 & 2015 & 70 GeV & \multirow{4}{*}{\parbox{6.5cm}{Seeded using the level L1\_XE50 (L1\_XE55) LAr and Tile calorimeter triggers, calibrated at the EM scale, with a threshold of 50(55) GeV.}} \\
        HLT\_xe90\_mht\_L1XE50 & 2016 (A-D3) & 90 GeV &  \\
        HLT\_xe110\_mht\_L1XE50  & 2016 ($\geq$ D4) & 110 GeV & \\
        HLT\_xe110\_pufit\_L1XE55 & 2017 & 110 GeV &  \\
        HLT\_xe110\_pufit\_xe70\_L1XE50 & 2018 & 110 GeV & \\
                                &&&\\
        \multicolumn{4}{l}{\bfseries{electrons}}\\
        HLT\_e24\_lhmedium\_L1EM20VH & 2015 & 24 GeV & Seeded using L1EM20VH level 1 trigger calibrated at the EM scale with a threshold of 20 GeV, and require medium ID quality.\\
                                &&&\\
        HLT\_e60\_lhmedium & 2015 & 60 GeV  & Seeded using L1EM20VH level 1 trigger calibrated at the EM scale with a threshold of 20 GeV, and require medium ID quality.\\
                                &&&\\
        HLT\_e120\_lhloose & 2015 & 120 GeV & Seeded using L1EM20VH level 1 trigger calibrated at the EM scale with a threshold of 20 GeV, and require loose ID quality.\\
                                &&&\\
        HLT\_e26\_lhtight\_nod0\_ivarloose & 2016 -- 2018 & 26 GeV & Tight likelihood ID required, and variable loose isolation required\\
        HLT\_e60\_lhmedium(\_nod0) & 2016 -- 2018 & 60 GeV & Medium ID likelihood required\\
        HLT\_e140\_lhloose(\_nod0) & 2016 -- 2018 & 140 GeV & Loose ID likelihood required\\
        HLT\_e300\_etcut & 2018 & 300 GeV & No ID requirements. \\
                                &&&\\
        \multicolumn{4}{l}{\bfseries{muons}}\\
        HLT\_mu20\_iloose\_L1MU15 & 2015 & 20 GeV & Seeded using L1MU15 level 1 trigger with a threshold of 15 GeV, and requiring loose isolation requirements.\\
                                &&&\\
        HLT\_mu50\ & 2015 -- 2018 & 60 GeV  & No isolation requirements. \\
        HLT\_mu26\_ivarmedium & 2016 -- 2018 & 26 GeV & Variable cone medium isolation requirements \\
        \bottomrule
      \end{tabular}
    }
    \caption{ Triggers used during the 2015, 2016, 2017 and 2018 data collection
      periods, notation like A or D3 denote periods during the year. }
    \label{tab:triggers}
  \end{center}
\end{table}

\subsection{0--Lepton Channel Triggers}
The events in the 0--lepton channel should have a $qq\nu\nu$ final state. We use
the $E_T^{miss}$ triggers listed in table~\ref{tab:triggers} as the final state
will manifest in the detector as $E_T^{miss}$ with the presence of jets. At the
stage of triggering $E_T^{miss}$ is only calculated from energy measured in the
calorimeters. As muons do not deposit much energy in the calorimeters the $W
\rightarrow \mu \nu + \text{jets}$ process is used to study the trigger
efficiency and derive an appropriate scale factor.


\subsection{1-- and 2--Lepton Channel Triggers}
The 1 and 2--lepton channels both contain charged leptons in the final state,
their final states are specifically $qq\ell\nu$ and $qq\ell\ell$ respectively.
As the two lepton channel is not expected to contain significant $E_T^{miss}$
and inefficiencies in the muon trigger are mitigated by the presence of two
muons in the events which wish to record the single lepton triggers listed in
table~\ref{tab:triggers} are simply used without any $E_T^{miss}$ triggers.

In the 1--lepton channel there is significant $E_T^{miss}$ expected due to the
presence of the neutrino. Single lepton triggers are used for events with $75
\GeV < p_T^{V} < 150 \GeV$ where the $E_T^{miss}$ triggers have yet to turn on
fully, for events with $p_T^{V} > 150 \GeV$ in order to mitigate inefficiencies
in the single muon triggers similar $E_T^{miss}$ triggers to the 0--lepton
triggers are used in conjunction with the single lepton triggers.


\section{Jets}
\label{sec:jets}

As mentioned in chapter~\ref{ch:theory} jets are the roughly conical structure
of detector activity resulting from the ultimate hadronisation of a QCD parton.
Two categories of jets are considered, signal jets and forward jets, when the
number of total jets is referred to it is equal to the sum of signal and forward
jets. As the Higgs candidate in every channel of the analysis is two b-tagged
jets, both the reconstruction of jets and the b-tagging strategy have huge
impacts on the final measurements. In this section the way jets are found and
reconstructed will be introduced, b-tagging will be explained in general and
then the specific tagging strategy of the analysis will be detailed.

\subsection{Topological Calorimeter Cluster Anti-$k_t$ Jets}
The jets that are found with a given algorithm are referred to as a jet
collection. The jet collection relevant to this analysis uses topological
calorimeter cell clusters to reconstruct jets~\cite{CALO2008}. These cluster are
then passed to the anti-$k_t$ jet finding algorithm~\cite{anti-kt}. This
algorithm takes a radius parameter which governs the size of jets, in the ATLAS
coordinates a radius parameter of $R=0.4$ is chosen.

As mentioned in chapter~\ref{ch:detector} pileup can cause issues with
reconstruction. In general there is a desire to suppress any jets which arise
from pileup. The Jet Vertex Tagger (JVT) is a likelihood-based discriminant
which is used to achieve this. The primary vertex location, jet $p_T$ and the

$p_T$ of tracks associated to a given jet, serve as inputs to the JVT which
outputs a 2-D likelihood that the jet arises from pileup. The likelihood is
resilient to bias arising from the jet flavour. The tool is applied only to jets
in region $\lvert  \eta \rvert < 2.5$ and $p_T > 120$ \GeV. A cut of JVT = 0.59,
is applied to all jets in the collection, this cut has an average efficiency of
92~\%. The definitions of signal and forward jets can be found in
table~\ref{tab:jet-cats}.
\begin{table}
  \centering
  \begin{tabular}{l l}
    \toprule
   Jet Category & Selection Requirements \\
    \midrule
    Forward Jets & jet cleaning \\
    & $p_{\mathrm{T}} > 30\,\mathrm{GeV}$ \\
    & $2.5 \leq \left|\eta\right| < 4.5$ \\ 
    %\hdashline \\[1ex]
    &\\
    Signal Jets & jet cleaning \\
    & $p_{\mathrm{T}} > 20\,\mathrm{GeV}$ \\
    & $ \left|\eta\right| < 2.5$ \\ 
    & JVT medium for $p_{\mathrm{T}} < 120\,\mathrm{GeV}$ \\
    \bottomrule
  \end{tabular}
  \caption{Jet selection requirements. Jet cleaning refers to the quality
    criteria interfaced through the JetCleaningTool CP tool. This tool removes
    jets in regions where either coherent/sporadic calorimeter noise or
    beam-induced background fake jets
    \cite{ATLAS-CONF-2015-029,Gonski:2272136}.}
  
  \label{tab:jet-cats}
\end{table}

\subsection{b-tagging}
\label{sec:btagging}

It is important to distinguish jets originating from b-quarks, which form our
Higgs candidate, from c-jets and $\tau$-jets, as well as jets originating from
quarks lighter than c-quark which are categorised together as light-jets. The
calculation of a discriminant which ought to separate b-jets from other jets is
known as b-tagging. In order to develop such a discriminant one must use
simulation in order to be able to know truly which parton initiated a jet such
that the performance of the discriminant can be validated. In simulation a jet
and the parton that initiated that jet are distinctly separate objects and so a
set of rules must be defined in order to decide which jet is b-jet and likewise
for other types of jets. Those rules are as follows:
\begin{enumerate}
\item  If a weakly decaying $b$-hadron is found within $\Delta R<R_{\mathrm max}$ of the
  jet axis, the jet is labeled a $b$-jet.
\item  If a $b$-hadron isn't found, but a weakly decaying $c$-hadron is
  found within $\Delta R<R_{\mathrm max}$ of the jet axis, then the jet is labeled as a $c$-jet.
\item  Otherwise, if a $\tau$-lepton is found within
  $\Delta R<R_{\mathrm max}$ of the jet axis, the jet is labeled a $\tau$-jet.
\item If any one hadron or $\tau$-lepton matches more than one jet, the closest jet
  is chosen as its parent.
\item All unlabeled jets after steps 1 through 4 are labeled as light-jets.
\end{enumerate}

The algorithm used to tag b-jets is the MV2c10 algorithm, this is a BDT which is
trained on kinematic and structural information about each jet. It is setup to
categorise between b-jets (signal) and a mixture of light-jets and c-jets
(background). The events in the training sample are simulated $t\bar{t}$ events
that have at least one lepton coming from a leptonically decaying W boson, and
hadronically decaying Z$^\prime$ events. The training sample has 5 million
$t\bar{t}$ events and 3 million Z$^\prime$ events.

The kinematic training variables that enter into the MV2c10 algorithm are simply
the jet $p_T$ and $\eta$. The structural information is more complicated, IP2D
and IP3D are two algorithms based on a log-likelihood ratio discriminant of
impact parameters (see chapter~\ref{ch:detector}). IP3D is defined as
\begin{align}
  \text{IP3D} &=\sum_{i=1}^{N}\log{\frac{p_b}{p_u}} \\
  \text{where}& \notag \\
  p_b &= P\Big(\text{is b-jet} \Big\vert \frac{d_0}{\sigma_{d_0}}, \frac{z_0 \sin{\theta}}{\sigma_{z_0\sin{\theta}}}\Big),\\
  p_u &= P\Big(\text{is light-jet} \Big\vert \frac{d_0}{\sigma_{d_0}}, \frac{z_0 \sin{\theta}}{\sigma_{z_0\sin{\theta}}}\Big),\\
  \text{and} & \notag \\
  N &= \text{the number of tracks for a given jet.} \notag
\end{align}
IP2D has the same definition but the probabilities $p_b$ and $p_u$ are
conditional only on the transverse impact parameter $d_0$ and have no dependence
on the longitudinal $z_0$. The output of two algorithms designed to find
secondary vertices, SV1 and JetFitter, also enter into the MV2c10 training.

A jet is defined b-tagged if its MV2c10 score exceeds a certain threshold. This
threshold is defined as the cut that gives a pre-determined efficiency value for
b-jets when applied to a $t\bar{t}$ sample. Calibrations are available for a
number of these so-called working points, these working points are shown in
table~\ref{tab:b-tag}.
\begin{table}[ht]
  \centering
  \resizebox{\textwidth}{!}{%
  \begin{tabular}{l l l r r}
    \toprule
    Name & MV2c10 cut & $b$-tagging efficiency (\%) & $c$-jet rejection & light-jet rejection \\
    \midrule
    FixedCutBEff\_60 & 0.94 & 61.14 & 22 & 1204 \\
    FixedCutBEff\_70 & 0.83 & 70.84 & 8 & 313 \\
    FixedCutBEff\_77 & 0.64 & 77.52 & 4 & 113 \\
    FixedCutBEff\_85 & 0.11 & 85.23 & 2 & 28 \\
    \bottomrule
  \end{tabular}
  }
  \caption[MV2c10 $b-tagging$ Working Points]{$b$-tagging working points
    available in this analysis, rejection is in the inverse of efficiency.}
  \label{tab:b-tag}
\end{table}


\subsection{Pseudo-continuous b-tagging}
The working points defined in table~\ref{tab:b-tag} are used in a so-called
pseudo-continuous mode. In this mode the MV2c10 distribution is binned with bin
edges corresponding to the working points listed in the table, and as can be
seen in figure~\ref{fig:PCBT_quant}. This figure shows the discrimination between
events with different jet flavours. 
\begin{figure}[!htbp]
  \centering
  \includegraphics[width=0.7\textwidth]{PCBT}
  \caption{Shapes of the PCBT quantiles for b-,c-, and light-jets.}
  \label{fig:PCBT_quant}
\end{figure}
Events that fall in the bins with ranges 70~--~60~\% and 60~--~0~\% have their
MV2c10 score stored for use in the analysis. Events in the other bins are not
tagged as b-jets.

\subsection{Truth Tagging}
\label{subsec:truth-tagging}

Removing all events that fail the cut associated with the 70~\% working point
results in loss of many events. This leaves us with low statistics samples for
analysis where we would rather have high statistics samples, as many of the
techniques used for categorising events have performance conditional on the
available statistics. Instead of simply throwing these events away it is
possible to keep all events that were simulated to have a b-hadron decay and
instead weight distributions by a factor which represents the probability that
an event would in fact be b-tagged and enter into the final distribution. This
procedure is known as Truth-Tagging, since the information as to whether a
b-hadron is truly in a simulated event is utilised, this does not use any
information relating to the true nature of the data as this is unknowable. In
order to draw easy comparisons, tagging directly with the MV2c10 algorithm is
referred to as direct tagging.

Given that our events contain more than on jet in all cases it is required to
develop of way of generating Truth-Tag weights for events based on all the jets
in the event. This weight is calculated as the product of the b-tagging
efficiency for each b-tagged jet, multiplied by the complement of the b-tagging
efficiency for each non b-tagged jet. In case the number of jets in a given
event ($m$) exceeds the number of required tagged jets in the analysis ($n=2$)
all possible combinations of jets which satisfy the analysis selection are
considered.

Given an event with $m$ jets and $n$ tagged jets required, the possible
combinations of tagged and non-tagged jets are $\binom{m}{n}$. For a given
tagged configuration, referred to as the $i^{th}$ combination and denoted as
$\binom{m}{n}_i$, the total number of remaining configurations is
$\overline{\binom{m}{n}} = \binom{m}{n} - \binom{m}{n}_i$. The efficiency and
inefficiency products are
\begin{align}                             
  \varepsilon \Bigg( \binom{m}{n}_i, x, f \Bigg)  & = \prod_{j \in n} \epsilon_x^f(j) \\
  \varepsilon_{in} \Bigg( \overline{\binom{m}{n}}_i, x, f \Bigg) & = \prod_{j \in m-n} (1 - \epsilon_x^f(j))
\end{align}
with $ \epsilon_x^f(j)$ the tagging efficiency of jet $j$ of flavour $f$ at an
efficiency working point $x$, and where $j \in m-n $ refers to the pool of non tagged
jets. The total event weight is
\begin{equation}
  w_{TT}(x) = \sum_i^{\lvert \binom{m}{n} \rvert} \varepsilon \Bigg(  \binom{m}{n}_i, x \Bigg)  \cdot \varepsilon_{in} \Bigg(  \overline{\binom{m}{n}}_i, x \Bigg),
\end{equation}
with a probability to choose a specific combination equal to
\begin{equation}
  P_{i}(x) = \frac{\varepsilon \big(  \binom{m}{n}_i, x \big) \cdot \varepsilon_{in} \big(  \overline{ \binom{m}{n}}_i, x \big) }{w_{TT}}.
\end{equation}
A single truth tagged combination can be selected based on this probability, and
the event is then scaled by the factor $w_{\mathrm {TT}}$. In practice due to
differences between the cumulative and pseudo-continuous b-tagging efficiency
distributions a modified version of Truth-Tagging must be applied in the
analysis, the modifications are described in appendix~\ref{app:truth-tag-pcbt}.

\subsection{Hybrid Tagging}
\label{subsec:hybrid-tagging}
There exists non-closure between direct-tagged and truth-tagged events which
means that truth tagging cannot directly be applied to events in the analysis.
This is because direct tagging is the only strategy which can be used on data
and so we can only assume that direct-tagged distributions in simulation will
describe the shape and normalisation of distributions in the data. This problem
is solved by implementing so-called hybrid tagging. The hybrid tagging strategy
involves the following steps:
\begin{enumerate}
\item Divide the jets of each event in two groups, depending on the truth tag
  flavour: a group with only true b-jets and the other with non b-jets.
  
\item All b-jets in the first group are direct tagged.
  
\item  The remaining group of c- and light-jets is truth tagged imposing a
  number of tagged jets proportional to the difference between the number of
  required jets in the signal region and the number of true b-jets in the first
  group passing the b-tag requirement.
\end{enumerate}
Distributions of $t\bar{t}$ and W + jets are shown in
figures~\ref{fig:truth_tag_validation_tt}~and~\ref{fig:truth_tag_validation_wjets}.
Which show direct, truth and hybrid-tagged distributions. It can be seen that
the closure of hybrid-tagged events is much better than that of truth-tagged
events with respect to that of the direct-tagged events. Hybrid tagging is
therefore chosen as the analysis strategy.
\begin{figure}[!htbp]
\centering
\subfloat[]{\includegraphics[width=0.48\textwidth]{TRF_2tag3jet_150ptv_WhfSR_ttbarbb_dRBB}}
\subfloat[]{\includegraphics[width=0.48\textwidth]{TRF_2tag3jet_150ptv_WhfSR_ttbarbc_j1VecCorrEta}}
\\
\subfloat[]{\includegraphics[width=0.48\textwidth]{TRF_2tag3jet_150ptv_WhfSR_ttbarbl_j1VecCorrPt}}
\caption{(a) $ \Delta R(\text{jet, jet})$  distribution in $t\overline{t} (bb)$ events, (b)  $\eta$ distribution in $t\overline{t} (bc)$ events and (c) $p_T$ distribution of the leading jet in $t\overline{t} (bl)$ events, using (black) direct tagging, (red) truth tagging and (green) hybrid tagging with the new customized maps.}
\label{fig:truth_tag_validation_tt}
\end{figure}

\begin{figure}[!htbp]
\centering
\subfloat[]{\includegraphics[width=0.48\textwidth]{TRF_2tag2jet_150ptv_WhfSR_Wbb_mBB}}
\subfloat[]{\includegraphics[width=0.48\textwidth]{TRF_2tag2jet_150ptv_WhfSR_Wbc_j2VecCorrPt}}
\\
\subfloat[]{\includegraphics[width=0.48\textwidth]{TRF_2tag3jet_150ptv_WhfSR_Wbl_dRBB}}
\caption{(a) $m_{bb}$ distribution in  \textit{W+bb} events, (b) $p_T$ of the sub-leading jet for \textit{W+bc} events and (c) $\Delta R(\text{jet, jet})$ distribution for \textit{W+bl} events using (black) direct tagging, (red) truth tagging and (green) hybrid tagging with the new customized maps.}
\label{fig:truth_tag_validation_wjets}
\end{figure}


\section{Missing Transverse Momentum}
\label{sec:met}
This section describes how $E_T^{miss}$ is calculated. The calculation is based
on the assumption that partons entering into the hard scatter interaction have
zero transverse momentum when they collide. This is not true due them having
Fermi temperature, but the transverse momentum induced by this phenomenon is
minimal and therefore is ignored. Given this assumption $E_T^{miss}$ is
calculated as the negative vector sum of the $p_T$ of photons, electrons, muons,
taus, and jets. There is also an additional soft term that enters into the sum
which is made up of all good quality tracks that aren't associated with any of
the aforementioned objects. These tracks must be associated with the primary
vertex and therefore are robust against pileup. A second formulation of
$E_T^{miss}$ called $E_{T, Trk}^{miss}$ is formulated just using inner detector
tracks, it is even more robust to pileup but cannot account for neutral
particles which do no leave tracks in the inner detector. One more variable
known as $E_T^{miss}$--significance is also calculated as
\begin{equation}
  E_T^{miss}\text{--significance} = \frac{E_T{miss}}{\sqrt{\Sigma p_T^e + \Sigma p_T^\mu + \Sigma p_T^{jet}}},
  \label{eq:metsig}
\end{equation}
where the denominator is clearly proportional to the $E_T^{miss}$ resolution and
therefore making cuts on this variable simulates making harder cuts on
$E_T^{miss}$ with better resolution and looser cuts on $E_T^{miss}$ with poorer
resolution.

\section{Overlap Removal}
Given all of the reconstruction methods described so far for different physics
objects there has yet been any mention of what happens if overlapping detector
activity could be reconstructed into more than one physics object. The treatment
of overlap depends on which physics objects are involved. Below is summary of
how different pairs of overlapping objects which are relevant to the analysis
are handled, in the summary a combined muon is one which has been reconstructed
with the combined muon reconstruction algorithm, that is to say it has tracks in
the ID and energy deposits in the ECAL.\\
$\bullet$ \textbf{tau-electron:} If $\Delta R(\tau,e)<0.2$, the $\tau$ lepton is
removed.\\
%
$\bullet$ \textbf{tau-muon:} If $\Delta R(\tau,\mu)<0.2$, the $\tau$ lepton is
removed, unless the $\tau$ lepton has $p_T>50$ \GeV and the muon is not a combined
muon, then the $\tau$ lepton is not removed.\\
%
$\bullet$ \textbf{electron-muon:} If a combined muon shares an ID track with an
electron, the electron is removed. If a calo-tagged muon shares an ID track with
an electron, the muon is removed.\\
%
$\bullet$ \textbf{electron-jet:} If $\Delta R(\mathrm{jet},e)<0.2$ the jet is
removed. For any remaining jets, if $\Delta R({\mathrm jet},e)<0.4$, the
electron is removed.\\
%
$\bullet$ \textbf{muon-jet} If $\Delta R({\mathrm jet},\mu)<0.2$ or the muon ID
track is ghost associated to the jet, then the jet is removed if the following
also holds. The jet has less than three associated tracks with $p_T > 500$ MeV
or the $p_T$ ratio of the muon and jet is larger than 0.5 and the ratio of the
muon $p_T$ to the sum of $p_T$ of tracks with $p_T > 500$ MeV associated to the
jet is larger than 0.7. For any remaining jets, if $\Delta
R(\mathrm{jet},\mu)<\min(0.4, 0.04 + 10\ \mathrm{\GeV} / p_T^{\mu})$, the muon is
removed from the jet.\\
%
$\bullet$ \textbf{tau-jet:} If $\Delta R(\tau,\mathrm{jet})<0.2$, the jet is
removed.


\section{Final Selection}
\label{sec:selection}

The final analysis selection varies between lepton channels. There are however
some common selections to all three channels as the Higgs candidate does not
differ between channels, after all this is what we aim to measure. All events must
have at least two signal jets. These signal jets must be b-tagged, the b-tagging
algorithm used is the MV2c10 algorithm with the 70\% efficiency working point
used, events with between 0 or 1 b-tags are considered for study and are not
used in the final analysis, events with $\ge3$ b-tags are rejected entirely. The
leading b-tagged jet my have $p_T > 45$ \GeV. A key variable in the analysis, the
di-jet mass, $m_{jj}$ is reconstructed from the two leading jets, in the case
where both jets are b-tagged $m_{bb} \equiv m_{jj}$ and represent the mass of
the Higgs candidate.

Reconstruction of the momentum of b-tagged jets can be enhanced with better
resolution with respect to other jets. This is achieved with so-called
muon-in-jet and $p_T$-reco corrections detailed in~\ref{app:b-tag-mom-res}.
Corrections are applied after events pass the full analysis selection, but
before other stages of the analysis. These corrections are not used in the
calculation of any $E_T^{miss}$ related variables.

\subsection{0--Lepton Channel Selection}
\label{sec:0lep-selection}

The vector boson candidate in the 0--lepton channel is a Z boson decaying to two
neutrinos. In order to select events for this candidate a large amount of
$E_T^{miss}$ is required (> 150 \GeV) and there must be exactly 0 $VH$-loose
leptons in the event. Whilst events with lower $E_T^{miss}$ may come from the
physical process we desire to measure the $E_T^{miss}$ trigger thresholds
require setting the cut at it's value, the efficiency is ~90\% for events
with $E_T^{miss} = 150$ \GeV and efficiency plateaus at $E_T^{miss} \approx 180$
\GeV.

There exists a non-trivial dependence of the trigger efficiency on activity of
jets in detector. This arises due to the fact that the calculation of
$E_T^{miss}$ is simply the total transverse energy of the event minus the
transverse energy of all objects in the event. These effects are hard to model
in certain phase spaces and so a requirement is put on $S_T$ such that it must
be greater than 120 \GeV for events with two total jets, and greater than 150 \GeV
for events with three total jets, where $S_T$ is defined as,
\begin{equation}
  S_T = \sum_i p_T^i, \text{for } i \text{ jets in the event.}
\end{equation}

Due to the lack of charged leptons in the $Z \rightarrow \nu\nu$ decay there is
nothing to trigger on to suppress multi-jet background processes arising from
QCD. This background is also enhanced due to limitations of the calorimeter
performance. A set of so-called anti-QCD cuts are applied to deal with this
multi-jet background, they are as follows:
\begin{itemize}
\item $\lvert \Delta \Phi ( E_T^{miss} , E_{t, trk}^{miss} ) \rvert <$ 90 $^\circ$
\item $\lvert \Delta \Phi ( j_1 , j_2 ) \rvert <$ 140 $^\circ$
\item $\lvert \Delta \Phi ( E_T^{miss} , h ) \rvert >$ 120 $^\circ$
\item $\text{min} ( \lvert \Delta \Phi ( E_T^{miss} , \text{pre-sel. jets}) \rvert ) >$ 20 $^\circ$ for  2 jets, $>$ 30 $^\circ$ for 3 jets.
\end{itemize}
Where $E_{t, trk}^{miss}$ is defined as the missing transverse momentum
calculated from the negative vector sum of the transverse momenta of tracks
reconstructed in the inner detector and identified as originating from the
primary vertex. The leading and sub-leading jets are denoted $j_1$ and $j_2$.
These cuts reduce the multi-jet background to approximately 1\% of the total
background in this channel.


\subsection{1--Lepton Channel Selection}
\label{sec:1lep-selection}

The vector boson candidate in the 1--lepton channel is a W boson decaying to one
charged lepton and one neutrino. In order to select for events of this signature
exactly one WH-signal lepton is required. At low $p_T$ there is increased
contribution from multi-jet processes, therefore the extra requirements of
$p_T^{W} > 150$ \GeV and $E_T^{miss} > 30$ \GeV. The latter cut is only applied in
the electron channel.  

\subsection{2--Lepton Channel Selection}
\label{sec:2lep-selection}

The vector boson candidate in the 1--lepton channel is a W boson decaying to two
charged leptons. Exactly two $VH$-loose leptons of the same lepton flavour are
required, additionally one of the leptons must pass the ZH-signal requirements.
For events with two muons, it is required that the muons are of opposite charge.
Electron reconstruction suffers from a higher rate of charge misidentification
and so this requirement is not applied to events with two electrons. The
di-lepton invariant mass is confined to be around the Z boson mass peak, events
require $81 < m_{ll} < 101$ \GeV. These cuts reduce multi-jet backgrounds to
negligible levels. 

The presence of two visible leptons in this channel allows for the enhancement
to b-tagged jets momentum resolution to be further improved. By inspection of
the decay in the channel it is clear that the momentum of the Higgs and vector
boson candidates ought to be balanced. The momentum resolution of the two
charged leptons forming the Z boson candidate is higher than that of the
b-tagged jets forming the Higgs candidate (even after corrections). The b-tagged
jets momentum can therefore be corrected with a kinematic fit. The kinematic fit
improves resolution by 20-30\% compared with the muon-in-jet corrected
quantities. This correction is only applied to events with 2 or 3 total jets as
the presence of more jets results in smearing of the effect over those
additional jets.

\begin{table}[ht]
  \centering
  \begin{tabular}{l l} 
    \toprule
    \multicolumn{2}{l}{\textbf{Common Selections}}\\
    Jets & $\geq$  2 signal jets  \\
    $b$-jets &  2 $b$-tagged signal jets \\
    Leading $b$-tagged-jet $p_{\mathrm{T}}$\  & $>$ 45 \GeV \\
         &\\
    \multicolumn{2}{l}{\textbf{0 Lepton}} \\
    Trigger & lowest un-prescaled $E_{\mathrm{T}}^{\text{miss}}$ triggers \\
    Leptons & 0 \VH-loose lepton \\
    $E_{\mathrm{T}}^{\text{miss}}$ & $>$ 150~\GeV  \\
    
    $S_{\mathrm{T}}$ & $>$ 120 (2 jets), $>$150 \GeV (3 jets)  \\
    $\lvert \text{min} \Delta \phi (E_{\mathrm{T}}^{\text{miss}}, \text{jet}) \rvert$ & $> 20\ensuremath{^\circ}$ (2 jets) , $> 30\ensuremath{^\circ}$(3 jets) \\
    $\lvert \Delta\phi(E_{\mathrm{T}}^{\text{miss}}, h) \rvert$ & $> 120\ensuremath{^\circ}$ \\
    $\lvert \Delta\phi(j_1, j_2) \rvert$ & $< 140\ensuremath{^\circ}$ \\
    $\lvert \Delta\phi(E_{\mathrm{T}}^{\text{miss}}, E_{T, \text{trk}}^{\text{miss}}) \rvert$ & $< 90\ensuremath{^\circ}$ \\
    $p_{\mathrm{T}}^V$ regions & [150, 250]~\GeV, [250, $\infty$]~\GeV  \\
         &\\
    \multicolumn{2}{l}{\textbf{1 Lepton}} \\
    Trigger &  $e$ channel: un-prescaled single electron \\
         & Tables 6 and 7 of Ref.~\cite{VHobjectsupportnote}\\
         & $\mu$ channel: see 0--lepton triggers \\
    Leptons & 1 \WH-signal\ lepton \\
         &  $>1$~\VH-loose\ lepton veto \\
    $E_{\mathrm{T}}^{\text{miss}}$   & $>$ 30~\GeV ($e$ channel) \\
    $p_{\mathrm{T}}^{V}$ regions & [150, 250]~\GeV, [250, $\infty$]~\GeV  \\ 
         &\\
    \multicolumn{2}{l}{\textbf{2 Lepton}}\\
    Trigger &  un-prescaled single lepton\\
         & Tables 6 and 7 of Ref.~\cite{VHobjectsupportnote}\\
    Leptons & 2 \VH-loose\ leptons \\
         & ($\ge$ 1 \ZH-signal\ lepton) \\
         &  Same flavor, opposite-charge for $\mu\mu$ \\
    $m_{\ell\ell}$   & 81 $< m_{\ell\ell} <$ 101~\GeV \\
    $p_{\mathrm{T}}^{V}$ regions & [75,150], [150, 250], [250, $\infty$]~\GeV  \\
    \bottomrule
  \end{tabular}
  \caption[The analysis event selection.]{Summary of the signal event selection
    in the 0--, 1-- and 2--lepton analyses. The tables describing single
    electron and muon triggers in Ref.~\cite{VHobjectsupportnote} are only
    available internally to ATLAS and so have been reproduced in
    appendix~\ref{app:recon}.}
  \label{tab:event-selection}
\end{table}