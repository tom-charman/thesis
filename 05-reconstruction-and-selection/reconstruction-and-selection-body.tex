\chapter{Reconstruction and Selection}%
\label{sec:method}

At the top level events are required to have one Higgs boson candidate and one
vector boson candidate. In all cases a Higgs boson candidate is comprised of two
b-tagged jets. More on the jet collection and b-tagging strategy used in
section~\ref{sec:jets}. Vector boson candidates are defined by a number of
different decay products defined in section~\ref{sec:channels-and-selections},
these decay products are triggered on, specific triggers used are discussed in
section~\ref{sec:triggers}. Reconstruction of basic quantities as well as higher
level algorithms such as overlap removal are handled by Athena and the CxAOD
Framework, more on these in section~\ref{sec:cxaod}. Finally with all quantities
reconstructed events are categorised in different analysis regions, these are
described in section~\ref{sec:ana-regions}.

\section{Channel Definitions and Lepton Selections}%
\label{sec:channels-and-selections}

The channels of the VH(bb) analysis are defined by the number of observed
charged leptons ($e$ or $\mu$) in the decay of the vector boson. There is one
channel for the study of WH(bb) decays where the leptonic decay $W \rightarrow
\ell\nu$ yields a single charged lepton. There are two channels for the study of
ZH(bb), the zero lepton channel, $Z \rightarrow \nu\nu$, and the two lepton
channel, $Z \rightarrow \ell\ell$. 

Two classifications of lepton are defined in order to categorise events into the
individual channels of the analysis, these are called VH-loose and VH-signal
leptons, channels are kept orthogonal by requiring different numbers of both
lepton categories. These classifications are defined in
table~\ref{tab:vh-leptons}. The characteristics of the fake lepton background
from QCD multi-jet processes differs between the 1 and 2 lepton channels hence
the reason for two different categorisations. In general to suppress this kind
of background leptons are required to be isolated from other detector activity.
\begin{table}
  \centering
  \resizebox{\textwidth}{!}{%
  \begin{tabular}{ l l l l l l l}
    \toprule
    \bfseries{Name} & $\bm{p_T}$ &  $\bm{\lvert \eta \rvert}$ & \bfseries{ID} & $\bm{d_0^{sig}}$ & $\bm{\lvert \Delta z_0\sin{\theta} \rvert}$
    & \bfseries{Isolation}\\
    \midrule
    \multicolumn{7}{l}{\bfseries{electrons}}\\
    VH-loose & > 7 \GeV  & < 2.47 & LH Loose & < 5 & < 0.5 mm & FCLoose \\
    ZH-signal & > 27 \GeV & < 2.47 & LH Loose & < 5 & < 0.5 mm & FCLoose \\
    WH-signal & > 27 \GeV & < 2.47 & LH Tight & < 5 & < 0.5 mm & FixedCutHighPtCaloOnly \\
    &&\\
    \multicolumn{7}{l}{\bfseries{muons}}\\
    VH-loose & > 7 \GeV & < 2.7 & Loose quality & < 3 & < 0.5 mm & FixecdCutLoose \\
    ZH-signal & > 27 \GeV & < 2.5 & Loose quality & < 3 & < 0.5 mm & FixecdCutLoose \\
 WH-signal & > 27 \GeV & < 2.5 & Medium quality & < 3 & < 0.5 mm & FixedCutHighPtCaloOnly \\
    \bottomrule
  \end{tabular}
  }
  \caption{Definitions of the VH leptons used to define and select events for
    the three analysis channels where $d_0^{sig}$ is the measured with respect to
    the beam line.}
  \label{tab:vh-leptons}
\end{table}

\subsection{Electrons}
\label{subsec:electrons}

As mentioned in chapter~\ref{sec:detector} electrons leave tracks in the ID and
energy deposits in the ECAL. Reconstructing electrons requires clustering the
energy deposits in the ECAL, this is achieved with a sliding window
algorithm~\cite{slide}. Clusters must then be associated to tracks in the ID, a
Gaussian Sum Filter~\cite{GSF} is used to account for energy loses due to
bremsstrahlung radiation. The energy for electron candidates must be calibrated
before it can be used in order to account for things such as non-uniformity in
the detector response. Calibration is achieved by using simulated cluster
activity from single particles to train a BDT regression model designed to
regress the measured energy in the ECAL to the simulated energy. An in-situ data
driven correction is applied to normalise the response between data and
simulation~\cite{elec-calibration}.

Reconstruction alone is not enough to find electrons, other particles may leave
similar signatures in the ATLAS sub-detectors and therefore electron
identification must also be performed. Identification is performed using a
likelihood-based method. Variables which have power to discriminate between
electrons and other particles are used in the likelihood such as shower
profiles, track quality, how closely track and cluster positions match in $\eta$
and $\phi$, and the presence of a high-threshold TRT hit. This is one of the
main benefits of the TRT. Performance of this method is well
studied~\cite{2likelihood-electron-id-notes}.

\subsection{Muons}

Finding muons in the detector requires consideration of the coverage of the
different ATLAS sub-detectors, especially in general are not stopped in the
detector. Muons leave charged tracks in the ID and the muon spectrometers which
have coverages of $\lvert \eta \rvert < 2.7$ and $\lvert \eta \rvert < 2.5$
respectively. For the region $\lvert \eta \rvert > 2.5$ a stand-alone algorithm
which doesn't use ID tracks can be used. All muons within the coverage of the ID
require good quality ID tracks~\cite{}. A combined algorithm is used in the
majority of cases. For the region $\lvert  \eta  \rvert < 0.1$ two specialised
algorithms SegmentTagged and CaloTagged are used which require only muon segment
and calorimeter deposits respectively. All aforementioned algorithms are used
together in what is known as a unified chain~\cite{muon-reco-id} to reconstruct and identify
muons.

\subsection{Taus}

As mentioned the only charged leptons that are considered in the analysis are
electrons and muons, leptonically decaying taus will have electrons and muons as
the only visible decay products however hadronically decaying taus must be
considered differently. Decays are considered as one or three pronged based on
the number of charged decay products, pions, with neutrinos a
nd neutral pions
also present. These decays are reconstructed in the calorimeters like jets with
the anti-$k_t$ algorithm with $\Delta R = 0.4$~\cite{tau-reco} but the $p_T$ of
the tau is set to the total energy of the TopoClusters within $\Delta I goR < 0.2$,
more on TopoClusters in section~\ref{sec:jets}. Tau candidates must have $p_T >
20 GeV$, $\lvert  \eta \rvert < 2.5$ excluding $1.37 \lvert \eta \rvert < 1.52$,
and either exactly 1 or 3 tracks. A BDT based method for tau identification is
used to reject fakes. Medium quality taus are counted for each event~\cite{med-taus2}.

\section{Triggers}
\label{sec:triggers}

As stated at the beginning of this chapter the decay products of the vector
boson candidate are used to trigger the recording of events for this analysis.
Therefore important triggers for the 0-lepton channel are $E_T^{miss}$ triggers,
and for the 1-lepton and 2-lepton channels the single electron, or single muon
triggers. Note that it is not necessary to trigger on both charged leptons
coming from the Z boson in the 2-lepton channel, the presence of one lepton
allows the triggering to occur and the requirement for 2-leptons can be imposed
at a later stage. Some events will be missed by not using a di-lepton trigger,
however these amount to approximately only 5\% of the total. The list of
triggers used as they appear in the ATLAS trigger menu are shown in
tables~\ref{tab:mettrig, tab:electrontrig, tab:muontrig}, for the $E_T^{miss}$,
electron and muon triggers respectively.

\begin{table}[ht]
  \begin{center}
    \resizebox{0.95\textwidth}{!}{
      \begin{tabular}{ l l l p{7cm} }
        \toprule
        \textbf{ Trigger Name } & \textbf{Period} & \textbf{Threshold (GeV)} & \textbf{Description} \\
        \midrule
        \multicolumn{4}{l}{\bfseries{$\bm{E_T^{miss}}$}}\\
        HLT\_xe70\_L1XE50 & 2015 & 70 GeV & \multirow{4}{*}{\parbox{6.5cm}{Seeded using the level L1\_XE50 (L1\_XE55) LAr and Tile calorimeter triggers, calibrated at the EM scale, with a threshold of 50(55) GeV.}} \\
        HLT\_xe90\_mht\_L1XE50 & 2016 (A-D3) & 90 GeV &  \\
        HLT\_xe110\_mht\_L1XE50  & 2016 ($\geq$ D4) & 110 GeV & \\
        HLT\_xe110\_pufit\_L1XE55 & 2017 & 110 GeV &  \\
        HLT\_xe110\_pufit\_xe70\_L1XE50 & 2018 & 110 GeV & \\
                                &&&\\
        \multicolumn{4}{l}{\bfseries{electrons}}\\
        HLT\_e24\_lhmedium\_L1EM20VH & 2015 & 24 GeV & Seeded using L1EM20VH level 1 trigger calibrated at the EM scale with a threshold of 20 GeV, and require medium ID quality.\\
                                &&&\\
        HLT\_e60\_lhmedium & 2015 & 60 GeV  & Seeded using L1EM20VH level 1 trigger calibrated at the EM scale with a threshold of 20 GeV, and require medium ID quality.\\
                                &&&\\
        HLT\_e120\_lhloose & 2015 & 120 GeV & Seeded using L1EM20VH level 1 trigger calibrated at the EM scale with a threshold of 20 GeV, and require loose ID quality.\\
                                &&&\\
        HLT\_e26\_lhtight\_nod0\_ivarloose & 2016 -- 2018 & 26 GeV & Tight likelihood ID required, and variable loose isolation required\\
        HLT\_e60\_lhmedium(\_nod0) & 2016 -- 2018 & 60 GeV & Medium ID likelihood required\\
        HLT\_e140\_lhloose(\_nod0) & 2016 -- 2018 & 140 GeV & Loose ID likelihood required\\
        HLT\_e300\_etcut & 2018 & 300 GeV & No ID requirements. \\
                                &&&\\
        \multicolumn{4}{l}{\bfseries{muons}}\\
        HLT\_mu20\_iloose\_L1MU15 & 2015 & 20 GeV & Seeded using L1MU15 level 1 trigger with a threshold of 15 GeV, and requiring loose isolation requirements.\\
                                &&&\\
        HLT\_mu50\ & 2015 -- 2018 & 60 GeV  & No isolation requirements. \\
        HLT\_mu26\_ivarmedium & 2016 -- 2018 & 26 GeV & Variable cone medium isolation requirements \\
        \bottomrule
      \end{tabular}
    }
    \caption{ Triggers used during the 2015, 2016, 2017 and 2018 data collection
      periods, notation like A or D3 denote periods during the year. }
    \label{tab:triggers}
  \end{center}
\end{table}
%\begin{table}[h]
  \begin{center}
    \resizebox{\textwidth}{!}{
      \begin{tabular}{l l l p{7cm}}
        \toprule
        \textbf{ Trigger Name } & \textbf{Period} & \textbf{Threshold (GeV)} & \textbf{Description} \\
        \midrule
        HLT\_e24\_lhmedium\_L1EM20VH & 2015 & 24 GeV & Seeded using L1EM20VH level 1 trigger calibrated at the EM scale with a threshold of 20 GeV, and require medium ID quality.\\
        &&&\\
        HLT\_e60\_lhmedium & 2015 & 60 GeV  & Seeded using L1EM20VH level 1 trigger calibrated at the EM scale with a threshold of 20 GeV, and require medium ID quality.\\
        &&&\\
        HLT\_e120\_lhloose & 2015 & 120 GeV & Seeded using L1EM20VH level 1 trigger calibrated at the EM scale with a threshold of 20 GeV, and require loose ID quality.\\
        &&&\\
        HLT\_e26\_lhtight\_nod0\_ivarloose & 2016 -- 2018 & 26 GeV & Tight likelihood ID required, and variable loose isolation required\\
        HLT\_e60\_lhmedium(\_nod0) & 2016 -- 2018 & 60 GeV & Medium ID likelihood required\\
        HLT\_e140\_lhloose(\_nod0) & 2016 -- 2018 & 140 GeV & Loose ID likelihood required\\
        HLT\_e300\_etcut & 2018 & 300 GeV & No ID requirements. \\
        \bottomrule
      \end{tabular}
    }
    \caption{ Single Electron triggers used during the 2015-2018 data collection
      period.  Please note that the notation, (A,~D3) refer to specific ATLAS
      collection periods in 2016. }
    \label{tab:electrontrigs}
  \end{center}
\end{table}
%\begin{table}[h]
  \begin{center}
    \resizebox{\textwidth}{!}{
      \begin{tabular}{l l l p{7cm}}
        \toprule
        \textbf{ Trigger Name } & \textbf{Period} & \textbf{Threshold (GeV)} & \textbf{Description} \\
        \midrule
        HLT\_mu20\_iloose\_L1MU15 & 2015 & 20 GeV & Seeded using L1MU15 level 1 trigger with a threshold of 15 GeV, and requiring loose isolation requirements.\\
                                &&&\\
        HLT\_mu50\ & 2015 -- 2018 & 60 GeV  & No isolation requirements. \\
        HLT\_mu26\_ivarmedium & 2016 -- 2018 & 26 GeV & Variable cone medium isolation requirements \\
        \bottomrule
      \end{tabular}
    }
    \caption{ Single muon triggers used during the 2015-2018 data collection period.}
    \label{tab:muontrigs}
  \end{center}
\end{table}

\section{Jets}
\label{sec:jets}


\section{Athena and the CxAOD Framework}
\label{sec:cxaod}
Data recorded by the ATLAS detector is passed through the central collaboration
software framework Athena before entering the analysis level data processing.
\begin{figure}[ht]
  \centering
  %Define block styles
  \tikzstyle{decision} = [diamond, draw, fill=blue!20, 
  text width=4.5em, text badly centered, node distance=3cm, inner sep=0pt]
  \tikzstyle{block} = [rectangle, draw, fill=blue!20, text width=12em, text
  centered, rounded corners, minimum height=3em]
  \tikzstyle{line} = [draw, -latex']
  \tikzstyle{cloud} = [draw, ellipse,fill=red!20, node distance=3cm, minimum
  height=2em]
  
    
  \begin{tikzpicture}[node distance = 2cm, auto]
    % Place nodes
    \node [block] (gen) {Generation};
    
    \node[below=0.75cm of gen] (gen-half) {};
    \node[right=0.75cm of gen-half] (evnt) {EVNT};
    
    \node [block, below=1.5cm of gen] (sim) {Simulation};

    \node[below=0.75cm of sim] (sim-half) {};
    \node[right=0.75cm of sim-half] (hits) {HITS};
    
    \node [block, below=1.5cm of sim] (digi) {Digitisation};

    \node[below=0.75cm of digi] (digi-half) {};
    \node[right=0.75cm of digi-half] (rdo) {RDO};
    
    \node [block, below=1.5cm of digi] (mc-reco) {Reconstruction};

    \node[below=0.75cm of mc-reco] (mc-reco-half) {};
    \node[right=0.75cm of mc-reco-half] (xaod) {xAOD};
    
    \node [block, below=1.5cm of mc-reco] (mc-der) {Derivation};

    \node[below=0.75cm of mc-der] (mc-der-half) {};
    \node[right=0.75cm of mc-der-half] (dxaod) {DxAOD}; 
    
    \node [block, left=3cm of digi, fill=red!20] (trig) {Trigger};

    \node[below=0.75cm of trig] (trig-half) {};
    \node[left=0.75cm of trig-half] (dxaod) {RAW};
    
    \node [block, below=1.5cm of trig, fill=red!20] (data-reco) {Reconstruction};

    \node[below=0.75cm of data-reco] (data-reco-half) {};
    \node[left=0.75cm of data-reco-half] (xaod) {xAOD};
    
    \node [block, below=1.5cm of data-reco, fill=red!20] (data-der) {Derivation};

    \node[below=0.75cm of data-der] (data-der-half) {};
    \node[left=0.75cm of data-der-half] (dxaod) {DxAOD};

    \node[left=1.5cm of mc-der] (ghost) {};
    \node [block, below=1.5cm of ghost, fill=purple!40] (ana) {Analysis};
    
    % Draw edges
    \path [line] (gen) -- (sim);
    \path [line] (sim) -- (digi);o
    \path [line] (digi) -- (mc-reco);
    \path [line] (mc-reco) -- (mc-der);

    \path [line] (trig) -- (data-reco);
    \path [line] (data-reco) -- (data-der);

    \path [line] (mc-der) -- (ana);
    \path [line] (data-der) -- (ana);
  \end{tikzpicture}
  \caption[Data flow central to ATLAS.]{A flow chart showing the Athena data
    processing chain. Red nodes indicate the presence of data recorded from
    collisions and blue nodes indicate the presence of Monte-Carlo simulated
    events.}
  \label{fig:data-flow}
\end{figure}

Athena is responsible for the steps shown in figure~\ref{fig:data-flow}. As can
be seen in the figure Athena processes both data recorded from collisions and
Monte-Carlo simulated predictions. Steps up until and including reconstruction
are required to transform the raw or simulated read-out of the detector into
what are known as physics objects. These physics objects correspond to UV and IR
safe descriptions particles and hadron showers e.g. leptons and jets. Given the
initial transverse energy of the collisions (zero) any missing transverse energy
($E^T_{miss}$) is also reconstructed based on the sum of the transverse energy
of all objects in an event, this missing energy indicates the presence of
particles in the event that cannot be detector by any of ATLAS subsystems. The
only particles in the Standard Model for which this is expected are neutrinos.
The files containing the reconstructed physics objects adhere to the ATLAS Event
Data Model (EDM) and are referred to as Analysis Object Data (xAOD). After
reconstruction a part of Athena called the derivation framework is used to
produce skimmed and slimmed xAODs known as Derived xAODs (DxAODs). The reduction
of these files is carried out based on a loose selection criteria, the criteria
used in the VH(bb) analysis are shown in table~\ref{tab:derivations}.

DxAODs are the usual starting point for analysis level software, in the case of
this analysis the CxAOD Framework. Of course xAODs can also be used as the
starting point for analysis they are just larger.
\input{05-reconstruction-and-selection/cxaod-flow}
As in in figure~\ref{fig:cxaod-flow} The CxAOD Framework has two major
components the Maker and the Reader. The job of the maker is to further slim the
data by performing pre-selection cuts and also to apply calibrations which will
be detailed below, the output of the Maker is called a Calibrated xAOD (CxAOD).
The Reader takes a CxAOD as input and performs the analysis event selection, it
can output histograms or nTuples.

The Maker applies selections the selections of each of the three analysis
channels, defined in section~\ref{sec:channels-and-selections}. CxAODs are
produced separately for each of the three channels as the different background
compositions and signal signatures require different optimisation. Pre-selection is
performed on jets based on application of requirements on transverse momentum and
pseudo-rapidity. A tool known as the Jet Vertex Tagger (JVT) is used to remove
jets resulting from pileup from events.

\section{Overlap Removal}

\section{Missing Transverse Momentum}

\section{Categorisation into Analysis Regions}
\label{sec:ana-regions}

\subsection{Top \texorpdfstring{$e \mu$}{e mu} control region}%

\label{sec:topemucr}
\subsection{\texorpdfstring{$\Delta R(b,b)$}{DRbb} Control Regions}%
\label{sec:control-region-defintions}


