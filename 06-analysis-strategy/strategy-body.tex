\chapter{Analysis Strategy}%
\label{ch:strategy}
In the previous chapter the process of reconstruction and selection was
described. With the objects required for this analysis reconstructed and an
event selection in place an analysis strategy is formed in order to maximise the
signal strength of the VH(bb) process and yield a robust result that is well
understood in terms of modelling and systematic errors. In this chapter that
strategy will be detailed, first the categorisation into analysis regions will
be defined, next the multi-variate algorithm that is used to generate some of
the distributions entering into the profile-likelihood fit will be explained.
Plots of the data versus Monte-Carlo prediction will then be shown in order to
build a picture of the pre-fit status of the agreement. Finally a series of
cross-checks which are used to validate the methods of the analysis will be
explained.

The author's contributions include studying the behaviour of the
profile-likelihood fit with the inclusion of these regions, which are new in
this round of the analysis and making comparisons between the 80~\invfb and
140~\invfb datasets). Contributions also include the training of the
multi-variate classification algorithm with the inclusion of new input variables
with respect to the previous version of the algorithm.

\section{Categorisation into Analysis Regions}
\label{sec:ana-regions}

Events which pass the selection detailed in table~\ref{tab:event-selection} are
categorised into several regions for analysis. Firstly they are split into what
are known as medium ($75 - 150 \GeV$), high ($150 - 250 \GeV$) and extreme ($ >
250 \GeV$) $p_T^{V}$.


Choice of pTV and nJets per channel.
Which backgrounds are present in which channels.
Which samples are used to simulate which backgrounds.
\begin{figure}[h]
  \centering
  \begin{tabular}{cc}
    \subfloat[]{\includegraphics[width=0.505\linewidth]{1lep_qqWH_2tag_2jet.png}}
    \subfloat[]{\includegraphics[width=0.49\linewidth]{1lep_qqWH_2tag_3jet.png}}\\
  \end{tabular}
  \caption[A 2--dimensional histogram of signal events in the $p_{\mathrm{T}}^V$, $\Delta
  R(b, \bar{b})$ plane.]{Signal distribution of $\Delta R$ between the two
    selected jets as function of $p_{\mathrm{T}}^{V}$ in the 1--lepton channel are shown
    in the 2-tag 2-jet (a) and 2-tag 3-jet (b) categories. The black lines
    demonstrate the upper and lower continuous cuts used to categorise the
    events into the signal and control regions. The colour scale shows the
    fraction of signal events in each bin of these 2D histograms.}
  \label{fig:drbb-crs}
\end{figure}


\subsection{Top \texorpdfstring{$e \mu$}{e mu} control region}%
\label{sec:topemucr}
Plots of top e-mu control region
Rationale behind usage

\section{Composition of Analysis Regions}
\begin{table}[tbph]
\centering
\resizebox{\textwidth}{!}{
  \begin{tabular}{llrS[table-format=3.2]S[table-format=3.2]S[table-format=3.2]}
    \toprule
    {\bfseries Process} & {\bfseries Generator} & \bfseries{$\bm{\sigma \times BR}$ [pb]} & \multicolumn{3}{c}{{\bfseries $\bm{N_{\text{events}}}$ in millions}}\\
                        &&&\bfseries{mc16a}&\bfseries{mc16d}&\bfseries{mc16e}\\
    \midrule

    $qq \to ZH \to \nu\nu b\bar{b}$ & \textsc{Powheg MiNLO} + \textsc{Pythia 8 } (NNPDF3)& $153.05\times0.582$ & 2  & 2  & 3.3  \\
    $qq \to WH \to l^+\nu b\bar{b}$ & \textsc{Powheg MiNLO} + \textsc{Pythia 8 } (NNPDF3)& $282.78\times0.582$ & 4  & 4  & 6.6  \\
    $qq \to WH \to l^-\nu b\bar{b}$ & \textsc{Powheg MiNLO} + \textsc{Pythia 8 } (NNPDF3) & $179.49\times0.582$ & 2  & 2  & 3.3  \\
    $qq \to ZH \to ll b\bar{b}$ & \textsc{Powheg MiNLO} + \textsc{Pythia 8 } (NNPDF3)& $77.04\times0.582$ & 3  & 3  & 5  \\
    $gg \to ZH \to \nu\nu b\bar{b}$ & \textsc{Powheg} + \textsc{Pythia 8}  (NNPDF3) & $24.57\times0.582$ & 0.5  & 0.5  & 0.5  \\
    $gg \to ZH \to l^{-}l^{+} b\bar{b}$ & \textsc{Powheg} + \textsc{Pythia 8} (NNPDF3) & $12.42\times0.582$ & 0.75  & 0.75  & 0.75  \\
\bottomrule
\end{tabular}
}
\caption[The samples of Monte-Carlo simulated predictions of background
processes.]{Monte Carlo samples used for the signal processes and the cross
  section and branching ratio (BR) used to normalise the different processes at
  $\sqrt{s}=13$~TeV. Branching ratios correspond to the $H\rightarrow b\bar{b}$
  decay while $V$ branching ratio are still included in the cross-section. This
  was made to make easier the comparison with the reference tables computed
  including $V$ decays in~\cite{twikiCrossSections}. $l$ corresponds to all $e$,
  $\mu$ and $\tau$ leptons together.}
\label{tab:sigMC}
\end{table}

\begin{table}[tbph]
\centering
\resizebox{\textwidth}{!}{
  \begin{tabular}{lllrrrr}
    \toprule
    \multicolumn{2}{l}{{\bfseries Process}} & {\bfseries Generator} & \bfseries{$\bm{\sigma \times BR}$ [pb]} & \multicolumn{3}{c}{{\bfseries $\bm{N_{\text{events}}}$ in millions}}\\
                                            &&&&\bfseries{mc16a}&\bfseries{mc16d}&\bfseries{mc16e}\\
    \midrule
    \multicolumn{2}{l}{\bfseries{Vector boson + jets}} & & & & & \\
    \multicolumn{2}{l}{$Z \to \nu\nu$} & \textsc{Sherpa} $2.2.1$ & $56280\times0.200$ & 150 & 160 & 140 \\
    \multicolumn{2}{l}{$W \to \ell\nu$} & \textsc{Sherpa} $2.2.1$ & $183600\times0.325$ & 340 & 400 & 540 \\
    \multicolumn{2}{l}{$Z/\gamma^{*} \to \ell\ell$} & \textsc{Sherpa} $2.2.1$ & $61940\times0.101$ & 120 & 160 & 210 \\
    \multicolumn{2}{l}{\bfseries{Top-quark}} & & & & & \\
    $t\bar{t}$ & non-full-had & \textsc{Powheg} + \textsc{Pythia 8} & $831.76\times0.543$ & 120(55) & 150(55) & 200(61) \\
                                            & (plus ET/pTW extensions)&&&&&\\
                                            & di-leptonic & \textsc{Powheg} + \textsc{Pythia 8} & $831.76\times0.105$ & $-$ & 45 & 100 \\
    Single-top & $s$~-~channel (leptonic-top) & \textsc{Powheg} + \textsc{Pythia 8} & $10.32\times0.325$ & 4 & 5 & 7 \\
                                            & $t$~-~channel (leptonic-top) & \textsc{Powheg} + \textsc{Pythia 8} & $216.96\times0.325$ & 10 & 12 & 17 \\
                                            & $Wt$~-~channel & \textsc{Powheg} + \textsc{Pythia 8} & $71.7\times1$ & 20 & 24(24) & 33(33) \\
                                            &\newline (plus di-lepton extension)&&&&&\\
    \multicolumn{2}{l}{\bfseries{Diboson}} & & & & & \\
    $qq\rightarrow WW$ & $\rightarrow qqlv$ & \textsc{Sherpa 2.2.1} & $112.6\times0.439$ & 14 & 50 & 24 \\
    $qq\rightarrow WZ$ & $\rightarrow lvqq$ (with $Z\rightarrow b\bar{b}$ extension) & \textsc{Sherpa 2.2.1} & $50.3\times0.227$ & 7(6) & 36(7) & 12(10) \\
                                            & $\rightarrow qqvv$ & \textsc{Sherpa 2.2.1} & $50.3\times0.135$ & 6 & 6 & 10 \\
                                            & $\rightarrow qqll$ & \textsc{Sherpa 2.2.1} & $50.3\times0.0683$ & 5 & 27 & 9 \\
    $qq\rightarrow ZZ$ & $\rightarrow qqll$ (with $Z\rightarrow b\bar{b}$ extension) & \textsc{Sherpa 2.2.1} & $15.57\times0.140$ & 5(5)  & 5(6) & 9(4) \\
                                            & $\rightarrow qqvv$ (with $Z\rightarrow b\bar{b}$ extension) & \textsc{Sherpa 2.2.1} & $15.57\times0.280$ & 5(5)  & 5(6) & 9(8) \\
    $gg\rightarrow WW$ & $\rightarrow qqlv$ & \textsc{Sherpa 2.2.2} & $4.8\times0.439$ & 0.8 & 0.9 & 1.1 \\
    $gg\rightarrow ZZ$ & $\rightarrow qqll$ or $qqvv$ & \textsc{Sherpa 2.2.2} & $1.57\times0.420$ & 4 & 6 & 8 \\
    \bottomrule
  \end{tabular}
}
\caption{Monte Carlo samples used for the background processes and the cross
  section times branching ratio (BR) used to normalise the different processes
  at $\sqrt{s}=13$ TeV. The last column shows the number of simulated events for
  the {mc16a+d+e} production. Branching ratios correspond to the decays shown.
  In this table $\ell$ is inclusive of $e$, $\mu$, $\tau$ leptons. For
  $Z/\gamma^{*} \to \ell\ell$ events the requirement of $m_{\ell\ell}>40$~\GeV
  was imposed.
%For the samples $WZ$ and $ZZ$ the number in brackets shows the number of additional events generated with the decay $Z \rightarrow b\bar{b}$.
}
\label{tab:bkgMC}
\end{table}


\section{Multi-variate Event Classification}%
\label{sec:mva}
Explaintation of algorithm, refer to ML chapter
- Classifier signal (VH) vs background (all other backgrounds)
- What regions is it trained on?
- What is the performance like?
- Transformation
\begin{table}[htbp]
\begin{center}
  \begin{tabular}{lllll}
    \toprule
    {\bfseries Variable} & {\bfseries Name} & {\bfseries 0--lepton} & {\bfseries 1--lepton} & {\bfseries 2--lepton} \\
    \midrule
    $m_{jj}$ & mBB & $\checkmark$ & $\checkmark$ & $\checkmark$ \\
    $\Delta R(jet_{1}, jet_{2})$ & dRBB & $\checkmark$ & $\checkmark$ & $\checkmark$ \\
    $p_{\mathrm{T}^{\text{jet1}}$ & pTB1 & $\checkmark$ & $\checkmark$ & $\checkmark$ \\
    $p_{\mathrm{T}^{\text{jet2}}$ & pTB2 & $\checkmark$ & $\checkmark$ & $\checkmark$ \\
    $p_{\mathrm{T}^{V}$ & pTV & \checkmark & $\checkmark$ & $\checkmark$ \\
    $\Delta \phi(V, H)$ & dPhiVBB & $\checkmark$ & $\checkmark$ & $\checkmark$ \\
    binned MV2c10(jet$_{1}$) & bin\_MV2c10B1 & $\checkmark$ & $\checkmark$ &  \\
    binned MV2c10(jet$_{2}$) & bin\_MV2c10B2 & $\checkmark$ & $\checkmark$ & \\
    $|\Delta \eta(jet_{1}, jet_{2})|$ & dEtaBB & $\checkmark$ &  &  \\
    $M_{\text{eff}}$ & MEff & $\checkmark$ & & \\
    track based soft MET term & softMET & $\checkmark$ & & \\
    MET & MET & $\equiv p_{\mathrm{T}^{V}$ & $\checkmark$ &  \\
    $\min(\Delta\phi(\ell,jet))$ & dPhiLBmin &  & $\checkmark$ & \\
    mTW\ & mTW &  & $\checkmark$ &  \\
    $\Delta Y(W,H)$ & dYWH & & $\checkmark$ &  \\
    $m_{\text{top}}$ & mTop & & $\checkmark$ & \\
    MET significance & METSig & & & $\checkmark$ \\
    $\Delta \eta(V, H)$ & dEtaVBB & &  & $\checkmark$ \\
    $m_{\ell\ell}$ & mLL & & & $\checkmark$ \\
    $\cos{\theta(\ell^-,Z)}$ & cosThetaLep & & & $\checkmark$ \\
    \multicolumn{5}{l}{\bfseries Only in 3 Jet Events} \\
    $p_{\mathrm{T}^{\text{jet}_3}$ & pTJ3 & $\checkmark$ & $\checkmark$ & $\checkmark$ \\
    $m_{jjj}$ & mBBJ & $\checkmark$ & $\checkmark$ & $\checkmark$ \\
    \bottomrule
  \end{tabular}
  \caption{Variables used to train the multi-variate discriminant.}
  \label{tab:MVAinputs}
\end{center}
\end{table}

\begin{table}[htbp]
  \begin{center}
    \begin{tabular}{llp{0.4\textwidth}}
      \toprule
      \textsc{tmva} Setting & Value & Definition \\
      \midrule
      BoostType & gradient boosting & Boost procedure \\
      Shrinkage & 0.5 & Learning rate \\
      SeparationType & Gini index & Node separation gain \\
      PruneMethod & No Pruning & Pruning method \\
      NTrees & 200 (600 for 1--lepton \VH) & Number of trees \\
      MaxDepth & 4 (2 for 1--lepton diboson) & Maximum tree depth \\
      nCuts & 100 & Number of equally spaced cuts tested per variable per node \\
      nEventsMin & 5\% & Minimum number of events in a node (\% of total events) \\
      \bottomrule
    \end{tabular}
    \caption[Hyperparameter choices used in the multi-variate
    analysis.]{Hyperparameters used for the BDT trainings. Exceptions for the
      1--lepton \VH\ and diboson trainings are given in brackets.}
    \label{tab:BDTSetup}
  \end{center}
\end{table}
 \begin{figure}[htbp]
  \centering
  \begin{tabular}{cccc}
    \subfloat[]{\includegraphics[width=0.33\linewidth]{0-lep-mva/Distr_SignalBackground_bin_MV2c10B1_BDT_0L_2J_150ptv_1of2_optimised_PCBTmode_SRCR-eps-converted-to}}
    \subfloat[]{\includegraphics[width=0.33\linewidth]{0-lep-mva/Distr_SignalBackground_bin_MV2c10B2_BDT_0L_2J_150ptv_1of2_optimised_PCBTmode_SRCR-eps-converted-to}}
     \subfloat[]{\includegraphics[width=0.33\linewidth]{0-lep-mva/Distr_SignalBackground_dEtaBB_BDT_0L_2J_150ptv_1of2_optimised_PCBTmode_SRCR-eps-converted-to}}\\
    \subfloat[]{\includegraphics[width=0.33\linewidth]{0-lep-mva/Distr_SignalBackground_dPhiVBB_BDT_0L_2J_150ptv_1of2_optimised_PCBTmode_SRCR-eps-converted-to}}
    \subfloat[]{\includegraphics[width=0.33\linewidth]{0-lep-mva/Distr_SignalBackground_dRBB_BDT_0L_2J_150ptv_1of2_optimised_PCBTmode_SRCR-eps-converted-to}}
     \subfloat[]{\includegraphics[width=0.33\linewidth]{0-lep-mva/Distr_SignalBackground_mBB_BDT_0L_2J_150ptv_1of2_optimised_PCBTmode_SRCR-eps-converted-to}}\\
    \subfloat[]{\includegraphics[width=0.33\linewidth]{0-lep-mva/Distr_SignalBackground_MEff_BDT_0L_2J_150ptv_1of2_optimised_PCBTmode_SRCR-eps-converted-to}}
     \subfloat[]{\includegraphics[width=0.33\linewidth]{0-lep-mva/Distr_SignalBackground_MET_BDT_0L_2J_150ptv_1of2_optimised_PCBTmode_SRCR-eps-converted-to}}          
    \subfloat[]{\includegraphics[width=0.33\linewidth]{0-lep-mva/Distr_SignalBackground_pTB1_BDT_0L_2J_150ptv_1of2_optimised_PCBTmode_SRCR-eps-converted-to}} \\
    \subfloat[]{\includegraphics[width=0.33\linewidth]{0-lep-mva/Distr_SignalBackground_pTB2_BDT_0L_2J_150ptv_1of2_optimised_PCBTmode_SRCR-eps-converted-to}}   
    \subfloat[]{\includegraphics[width=0.33\linewidth]{0-lep-mva/Distr_SignalBackground_softMET_BDT_0L_2J_150ptv_1of2_optimised_PCBTmode_SRCR-eps-converted-to}}\\    
    \end{tabular}
    \caption[Inputs to the multi-variate analysis in the 0--lepton 2--jet
    region.]{Inputs to the multi-variate analysis in the 0--lepton 2--jet
      region. Signal events are shown in blue and background events are shown in
      red. The signal and background histograms have been normalised to the same
      area. The distributions only include events with $E_T^{\text{miss}}$ > 150
      \GeV.}
    \label{fig:bdtinputs-0lep}
\end{figure}

 \begin{figure}[htbp]
  \centering
  \begin{tabular}{cccc}
    \subfloat[]{\includegraphics[width=0.3\linewidth]{1-lep-mva/Distr_SignalBackground_bin_MV2c10B1_BDT_1L_2J_150ptv_1of2-eps-converted-to}}
    \subfloat[]{\includegraphics[width=0.3\linewidth]{1-lep-mva/Distr_SignalBackground_bin_MV2c10B2_BDT_1L_2J_150ptv_1of2-eps-converted-to}}
     \subfloat[]{\includegraphics[width=0.3\linewidth]{1-lep-mva/Distr_SignalBackground_dPhiLBmin_BDT_1L_2J_150ptv_1of2-eps-converted-to}}\\
    \subfloat[]{\includegraphics[width=0.3\linewidth]{1-lep-mva/Distr_SignalBackground_dPhiVBB_BDT_1L_2J_150ptv_1of2-eps-converted-to}}
    \subfloat[]{\includegraphics[width=0.3\linewidth]{1-lep-mva/Distr_SignalBackground_dRBB_BDT_1L_2J_150ptv_1of2-eps-converted-to}}
     \subfloat[]{\includegraphics[width=0.3\linewidth]{1-lep-mva/Distr_SignalBackground_dYWH_BDT_1L_2J_150ptv_1of2-eps-converted-to}}\\
    \subfloat[]{\includegraphics[width=0.3\linewidth]{1-lep-mva/Distr_SignalBackground_mBB_BDT_1L_2J_150ptv_1of2-eps-converted-to}}
     \subfloat[]{\includegraphics[width=0.3\linewidth]{1-lep-mva/Distr_SignalBackground_MET_BDT_1L_2J_150ptv_1of2-eps-converted-to}}          
    \subfloat[]{\includegraphics[width=0.3\linewidth]{1-lep-mva/Distr_SignalBackground_Mtop_BDT_1L_2J_150ptv_1of2-eps-converted-to}} \\
    \subfloat[]{\includegraphics[width=0.3\linewidth]{1-lep-mva/Distr_SignalBackground_mTW_BDT_1L_2J_150ptv_1of2-eps-converted-to}}   
    \subfloat[]{\includegraphics[width=0.3\linewidth]{1-lep-mva/Distr_SignalBackground_pTB1_BDT_1L_2J_150ptv_1of2-eps-converted-to}}
    \subfloat[]{\includegraphics[width=0.3\linewidth]{1-lep-mva/Distr_SignalBackground_pTB2_BDT_1L_2J_150ptv_1of2-eps-converted-to}}\\   
    \subfloat[]{\includegraphics[width=0.3\linewidth]{1-lep-mva/Distr_SignalBackground_pTV_BDT_1L_2J_150ptv_1of2-eps-converted-to}} 
    \end{tabular}
    \caption[Inputs to the multi-variate analysis in the 1--lepton 2--jet
    region.]{Inputs to the multi-variate analysis in the 1--lepton 2--jet
      region. Signal events are shown in blue and background events are shown in
      red. The signal and background histograms have been normalised to the same
      area.The distributions only include events with $p_T^{W}$ > 150
      \GeV.}
    \label{fig:bdtinputs-1lep}
\end{figure}

 \begin{figure}[htbp]
  \centering
  \begin{tabular}{cccc}
    \subfloat[]{\includegraphics[width=0.33\linewidth]{2-lep-mva/Distr_SignalBackground_cosThetaLep_BDT_2L_2J_150ptv_1of2_optimised_PCBTmode_SRCR_extensions-eps-converted-to}}
    \subfloat[]{\includegraphics[width=0.33\linewidth]{2-lep-mva/Distr_SignalBackground_dEtaVBB_BDT_2L_2J_150ptv_1of2_optimised_PCBTmode_SRCR_extensions-eps-converted-to}}
     \subfloat[]{\includegraphics[width=0.33\linewidth]{2-lep-mva/Distr_SignalBackground_dPhiVBB_BDT_2L_2J_150ptv_1of2_optimised_PCBTmode_SRCR_extensions-eps-converted-to}}\\
    \subfloat[]{\includegraphics[width=0.33\linewidth]{2-lep-mva/Distr_SignalBackground_dRBB_BDT_2L_2J_150ptv_1of2_optimised_PCBTmode_SRCR_extensions-eps-converted-to}}
    \subfloat[]{\includegraphics[width=0.33\linewidth]{2-lep-mva/Distr_SignalBackground_mBB_BDT_2L_2J_150ptv_1of2_optimised_PCBTmode_SRCR_extensions-eps-converted-to}}
     \subfloat[]{\includegraphics[width=0.33\linewidth]{2-lep-mva/Distr_SignalBackground_METSig_BDT_2L_2J_150ptv_1of2_optimised_PCBTmode_SRCR_extensions-eps-converted-to}}\\
    \subfloat[]{\includegraphics[width=0.33\linewidth]{2-lep-mva/Distr_SignalBackground_mLL_BDT_2L_2J_150ptv_1of2_optimised_PCBTmode_SRCR_extensions-eps-converted-to}}
     \subfloat[]{\includegraphics[width=0.33\linewidth]{2-lep-mva/Distr_SignalBackground_pTB1_BDT_2L_2J_150ptv_1of2_optimised_PCBTmode_SRCR_extensions-eps-converted-to}}          
    \subfloat[]{\includegraphics[width=0.33\linewidth]{2-lep-mva/Distr_SignalBackground_pTB2_BDT_2L_2J_150ptv_1of2_optimised_PCBTmode_SRCR_extensions-eps-converted-to}} \\
    \subfloat[]{\includegraphics[width=0.33\linewidth]{2-lep-mva/Distr_SignalBackground_pTV_BDT_2L_2J_150ptv_1of2_optimised_PCBTmode_SRCR_extensions-eps-converted-to}}   
    \end{tabular}
    \caption[Inputs to the multi-variate analysis in the 2--lepton 2--jet
    region.]{Inputs to the multi-variate analysis in the 2--lepton 2--jet
      region. Signal events are shown in blue and background events are shown in
      red. The signal and background histograms have been normalised to the same
      area.The distributions only include events with $p_T^{Z}$ > 150
      \GeV.}
    \label{fig:bdtinputs-2lep}
\end{figure}



\subsection{Pre-fit plots}
Summary of everything
\input{06-analysis-strategy/0lep-2jet-prefit}
\begin{figure}
  \centering
  \begin{tabular}{cc}
    % top row
    \includegraphics[width=.3\textwidth]{final_fit_mva/prefit/Region_BMax250_BMin150_Y6051_DCRHigh_{\mathrm{T}}2_L0_distMET_J3_Prefit}%
    & \includegraphics[width=.3\textwidth]{final_fit_mva/prefit/Region_BMin250_Y6051_DCRHigh_{\mathrm{T}}2_L0_distMET_J3_Prefit} \\

    % middle row
    \includegraphics[width=.3\textwidth]{final_fit_mva/prefit/Region_BMax250_BMin150_Y6051_DSR_{\mathrm{T}}2_L0_distmva_J3_Prefit}%
    & \includegraphics[width=.3\textwidth]{final_fit_mva/prefit/Region_BMin250_Y6051_DSR_{\mathrm{T}}2_L0_distmva_J3_Prefit} \\

    % bottom row
    \includegraphics[width=.3\textwidth]{final_fit_mva/prefit/Region_BMax250_BMin150_Y6051_DCRLow_{\mathrm{T}}2_L0_distMET_J3_Prefit}%
    & \includegraphics[width=.3\textwidth]{final_fit_mva/prefit/Region_BMin250_Y6051_DCRLow_{\mathrm{T}}2_L0_distMET_J3_Prefit} \\
  \end{tabular}
  \caption{Pre-fit distributions in the 0--lepton channel in the 3--jet region.}
  \label{fig:0lep-3jet-prefit}
\end{figure}
\begin{figure}
  \centering
  \begin{tabular}{cc}
    % top row
    \includegraphics[width=.3\textwidth]{final_fit_mva/prefit/Region_BMax250_BMin150_Y6051_DCRHigh_T2_L1_distpTV_J2_Prefit}%
    & \includegraphics[width=.3\textwidth]{final_fit_mva/prefit/Region_BMin250_Y6051_DCRHigh_T2_L1_distpTV_J2_Prefit} \\

    % middle row
    \includegraphics[width=.3\textwidth]{final_fit_mva/prefit/Region_BMax250_BMin150_Y6051_DSR_T2_L1_distmva_J2_Prefit}%
    & \includegraphics[width=.3\textwidth]{final_fit_mva/prefit/Region_BMin250_Y6051_DSR_T2_L1_distmva_J2_Prefit} \\

    % bottom row
    \includegraphics[width=.3\textwidth]{final_fit_mva/prefit/Region_BMax250_BMin150_Y6051_DCRLow_T2_L1_distpTV_J2_Prefit}%
    & \includegraphics[width=.3\textwidth]{final_fit_mva/prefit/Region_BMin250_Y6051_DCRLow_T2_L1_distpTV_J2_Prefit} \\
  \end{tabular}
  \caption{Pre-fit distributions in the 1 lepton 2 jet channel.}
\end{figure}
\begin{figure}
  \centering
  \begin{tabular}{cc}
    % top row
    \includegraphics[width=.3\textwidth]{final_fit_mva/prefit/Region_BMax250_BMin150_Y6051_DCRHigh_{\mathrm{T}}2_L1_distpTV_J3_Prefit}%
    & \includegraphics[width=.3\textwidth]{final_fit_mva/prefit/Region_BMin250_Y6051_DCRHigh_{\mathrm{T}}2_L1_distpTV_J3_Prefit} \\

    % middle row
    \includegraphics[width=.3\textwidth]{final_fit_mva/prefit/Region_BMax250_BMin150_Y6051_DSR_{\mathrm{T}}2_L1_distmva_J3_Prefit}%
    & \includegraphics[width=.3\textwidth]{final_fit_mva/prefit/Region_BMin250_Y6051_DSR_{\mathrm{T}}2_L1_distmva_J3_Prefit} \\

    % bottom row
    \includegraphics[width=.3\textwidth]{final_fit_mva/prefit/Region_BMax250_BMin150_Y6051_DCRLow_{\mathrm{T}}2_L1_distpTV_J3_Prefit}%
    & \includegraphics[width=.3\textwidth]{final_fit_mva/prefit/Region_BMin250_Y6051_DCRLow_{\mathrm{T}}2_L1_distpTV_J3_Prefit} \\
  \end{tabular}
  \caption{Pre-fit distributions in the 1--lepton channel in the 3--jet region.}
  \label{fig:1lep-3jet-prefit}
\end{figure}
\begin{figure}
  \centering
  \begin{tabular}{cc}
    % top row
    \includegraphics[width=.3\textwidth]{final_fit_mva/prefit/Region_BMax150_BMin75_Y6051_DCRHigh_{\mathrm{T}}2_L2_distpTV_J2_Prefit}%
    \includegraphics[width=.3\textwidth]{final_fit_mva/prefit/Region_BMax250_BMin150_Y6051_DCRHigh_{\mathrm{T}}2_L2_distpTV_J2_Prefit}%
    & \includegraphics[width=.3\textwidth]{final_fit_mva/prefit/Region_BMin250_Y6051_DCRHigh_{\mathrm{T}}2_L2_distpTV_J2_Prefit} \\

    % middle row
    \includegraphics[width=.3\textwidth]{final_fit_mva/prefit/Region_BMax150_BMin75_Y6051_DSR_{\mathrm{T}}2_L2_distmva_J2_Prefit}%
    \includegraphics[width=.3\textwidth]{final_fit_mva/prefit/Region_BMax250_BMin150_Y6051_DSR_{\mathrm{T}}2_L2_distmva_J2_Prefit}%
    & \includegraphics[width=.3\textwidth]{final_fit_mva/prefit/Region_BMin250_Y6051_DSR_{\mathrm{T}}2_L2_distmva_J2_Prefit} \\

    % bottom row
    \includegraphics[width=.3\textwidth]{final_fit_mva/prefit/Region_BMax150_BMin75_Y6051_DCRLow_{\mathrm{T}}2_L2_distpTV_J2_Prefit}%
    \includegraphics[width=.3\textwidth]{final_fit_mva/prefit/Region_BMax250_BMin150_Y6051_DCRLow_{\mathrm{T}}2_L2_distpTV_J2_Prefit}%
    & \includegraphics[width=.3\textwidth]{final_fit_mva/prefit/Region_BMin250_Y6051_DCRLow_{\mathrm{T}}2_L2_distpTV_J2_Prefit} \\
  \end{tabular}
  \caption{Pre-fit distributions in the 2--lepton channel in the  2--jet
    region.}
  \label{fig:2lep-2jet-prefit}
\end{figure}
\begin{figure}
  \centering
  \begin{tabular}{cc}
    % top row
    \includegraphics[width=.33\textwidth]{final_fit_mva/prefit/Region_BMax150_BMin75_incJet1_Y6051_DCRHigh_T2_L2_distpTV_J3_Prefit}%
    \includegraphics[width=.33\textwidth]{final_fit_mva/prefit/Region_BMax250_BMin150_incJet1_Y6051_DCRHigh_T2_L2_distpTV_J3_Prefit}%
    & \includegraphics[width=.33\textwidth]{final_fit_mva/prefit/Region_BMin250_incJet1_Y6051_DCRHigh_T2_L2_distpTV_J3_Prefit} \\

    % middle row
    \includegraphics[width=.33\textwidth]{final_fit_mva/prefit/Region_BMax150_BMin75_incJet1_Y6051_DSR_T2_L2_distmva_J3_Prefit}%
    \includegraphics[width=.33\textwidth]{final_fit_mva/prefit/Region_BMax250_BMin150_incJet1_Y6051_DSR_T2_L2_distmva_J3_Prefit}%
    & \includegraphics[width=.33\textwidth]{final_fit_mva/prefit/Region_BMin250_incJet1_Y6051_DSR_T2_L2_distmva_J3_Prefit} \\

    % bottom row
    \includegraphics[width=.33\textwidth]{final_fit_mva/prefit/Region_BMax150_BMin75_incJet1_Y6051_DCRLow_T2_L2_distpTV_J3_Prefit}%
    \includegraphics[width=.33\textwidth]{final_fit_mva/prefit/Region_BMax250_BMin150_incJet1_Y6051_DCRLow_T2_L2_distpTV_J3_Prefit}%
    & \includegraphics[width=.33\textwidth]{final_fit_mva/prefit/Region_BMin250_incJet1_Y6051_DCRLow_T2_L2_distpTV_J3_Prefit} \\
  \end{tabular}
  \caption{Pre-fit distributions in the 2--lepton channel in the  $\geq$3--jet
    region.}
  \label{fig:2lep-3pjet-prefit}
\end{figure}
\section{Analysis Cross-checks}
Mention VZ and cut-based cross checks so I can refer to them in the systematics
chapter.





