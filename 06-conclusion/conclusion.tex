\chapter{Conclusion and improvements}%
\label{sec:conclusion}

A summary of the sub detectors that make up the ATLAS detector at the LHC has been given and referred
to throughout the detailing of the Higgs boson analysis in which the Higgs is produced in association
with a vector boson and decays into two b quarks. The relevant physics theory was described and the
Brout-Englert-Higgs mechanism was described in basic detail in order to motivate the importance of
Higgs boson measurements. The analysis strategy has been described in some detail with references to
the full procedures provided. The main stages of the analysis are as follows. Object reconstruction,
which relies heavily on the tools and studies carried about the ATLAS collaboration. Event selection
which is motivated by the physics theory that underlies the entire analysis. The multivariate analysis
which takes the outputs of the object reconstruction and event selection and uses powerful statistical
techniques to exploit as much information that these variables contain as possible in forming a
discriminant on the signal and background. The treatment of statistical uncertainties, which is crucial
to bring the result into the real world by accounting for limitations of the detector and theoretical
techniques used to generate simulations. Overall when everything is taken into consideration the
observed significance of the signal process was over 5 standard deviations, the threshold for a
discovery.

One way that an improvement might be made to the analysis has been discussed. The use of a new bespoke
algorithm to perform high dimensional re-weighting could be used in order to quantify systematic
uncertainties. Similarly studies will be performed in the future to determine if any advancements in
machine learning (a rapidly growing field of research in itself) can be used in order to improve any of
the areas where multivariate techniques are used, namely the classification BDT used in the main
analysis and in the b-tagging algorithms used in event selection.
