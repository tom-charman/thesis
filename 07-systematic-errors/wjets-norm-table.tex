\begin{table}[!htb]
  \centering
  \resizebox{\textwidth}{!}{
    \begin{tabular}{llllllll}
      \toprule
      & & \multicolumn{6}{ c }{Prior in Region} \\
      Process & Name & CRLow 2-Jets & SR 2-Jets & CRHigh 2-Jets &  CRLow 3-Jets & SR 3-Jets & CRHigh 3-Jets  \\
      \midrule
      {\bfseries 1-Lepton} & & & & & & & \\
      W+l    &  {\texttt{SysWclNorm}}             &  \multicolumn{6}{ c }{   32\% } \\
      W+cl   &  {\texttt{SysWlNorm}}              &  \multicolumn{6}{ c }{  37\%  }\\
      W+hf   & {\texttt{norm\_Wbb\_J2}}           &  \multicolumn{6}{ c }{ Floating Normalisation (2-jet events only)} \\
      W+hf   & {\texttt{norm\_Wbb\_J3}}           &  \multicolumn{6}{ c }{ Floating Normalisation (3-jet events only)} \\
      W+hf   & {\texttt{SysWbbCRSRextrap}}        &   -11.5\% &  -- & +14.8\%  &  -3.6\% & -- & +5.1\%    \\
      {\bfseries 0-Lepton} & & & & & & & \\
      W+hf   & {\texttt{SysWbbNorm\_L0}}    &   5\% &  5\% & --  &  -- & -- & --    \\
      W+hf   & {\texttt{SysWbbCRSRextrap}}        &   -7.7\% &  -- & +14.9\%  &  -5.6\% & -- & +7.0\%    \\
      \bottomrule
    \end{tabular}
  }
  \caption{Summary of the nuisance parameters used to account for the
    uncertainty on W + jets background predictions. Systematic uncertainties
    that are implemented with a prior have the value corresponding to a
    1--$\sigma$ shift shown broken down by lepton channel and analysis category.
    As can be seen by comparing the process and name columns uncertainties named
    with \texttt{Wbb} are applied to all W + hf events.}
  \label{tab:wjetsnorm}
\end{table}