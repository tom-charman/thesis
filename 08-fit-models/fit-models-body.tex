\chapter{Fit Models}
\label{ch:fit-models}
\section{Profile Likelihood Fits}%
\label{sec:plf}
\subsection{Likelihood function definition}
\label{sec:lhoodDef}
The statistical analysis of the data uses a binned likelihood function which
maximum correspond to the best description of data. It is defined as the product
over all bins of the Poisson probability to observe $N^{\text{obs}}_b$ data
events given a prediction of
$N^{\text{exp}}_b(\mu,\mathbf{k},{\bm\theta})$ events in a certain bin
$i$:

\begin{equation}
  L(\mu,{\bm{k},\bm{\theta}}) = \prod_{i\in\,\text{bins}} \frac{\left( N_{i}^{\text{exp}}(\mu,{\bm{k,\theta}}) \right)^{N_{i}^{\text{data}}}}{N_{i}^{\text{data}}\,!} \cdot e^{-N_{i}^{\text{exp}}(\mu,{\bm{k,\theta}})}
  
\end{equation}

In this likelihood, the number of predicted events is made dependent on three
sets of parameters: the signal strength $\mu$, the scale factors
$\mathbf{k}=\left\{k_1, ...,k_j\right\}$, and the nuisance parameters
$\bm{\theta} = \left\{\theta_1,...,\theta_l\right\}$, as follows

\begin{equation}
  N_{i}^{\text{exp}}(\mu,\mathbf{k},\bm{\theta}) = \mu \cdot N_{i,\text{sig}}^{\text{exp}}(\bm{\theta}) + \sum_{b\in\,\text{bkg}} k_b\cdot N_{i,b}^{\text{exp}}(\bm{\theta})
  
\end{equation}
The parameter of interest $\mu=\sigma/\sigma_{\text{SM}}$ is common to all
channels and category and is the ratio between the measured and the expected
signal cross-sections.

As each scale factor does for it associated background component, the signal
strength scales the amount of signal linearly without any prior constraint, or
penalty in the likelihood function.


The nuisance parameters (NP) $\theta_i$ encode the dependence of the prediction
on systematic uncertainties into continuous parameters in the likelihood.

The prior knowledge on these parameters is reflected by a Gaussian penalty term
$\text{Gauss}(0\,|\,\theta_i,1)$ added to the likelihood for each NP, rending
displacement of these parameters depreciated.

The parameters $\theta_i$ are therefore expressed in standard deviation in the
following.

It results in a log-normal (normal) dependence of the predicted rates (shapes)
on the displayed parameter values.


The nominal fit result in terms of $\mu$ and $\sigma_{\mu}$ is obtained by
maximizing the likelihood function with respect to all parameters.  This is
referred to as the maximized log-likelihood value, MLL. The profile likelihood
ratio test statistic, $q_\mu$, is then constructed as follows:

\begin{equation}
  q_\mu = - 2\; \ln \left[ \mathcal{L} (\mu, \hat{\hat{\mathbf{k}}}, \hat{\hat{\bm\theta}}_{\mu})\, / \, \mathcal{L} (\hat{\mu}, \hat{\mathbf{k}}, \hat{\bm\theta}) \right]
  
\end{equation}
where $\hat{\mu}$ and $\hat{\theta}$ are the parameters that maximise the
likelihood (with the constraint $0 \leq \hat{\mu} \leq \mu$), and
$\hat{\hat{\theta}}_\mu$ are the nuisance parameter values that maximise the
likelihood for a given $\mu$. This test statistic is used to measure the
compatibility of the background-only model with the observed data, extracting
the local $p_0$ value, and, if no hint of a signal is found in this procedure,
for the derivation of exclusion intervals using the $CL_s$
method~\cite{Cowan:2010js,Read:2002hq}.

\begin{table}[h!tbp]
  \setlength{\extrarowheight}{2pt} 
  \begin{center}
\resizebox{\textwidth}{!}{
    \begin{tabular}{| l | l | C{1.7cm} | C{1.7cm} | C{1.9cm} | C{1.9cm} | C{1.3cm} | C{1.3cm} |}

\hline
\hline
\multirow{3}{*}{Channel}   & \multirow{3}{*}{SR/CR}  & \multicolumn{6}{c|}{Categories}   \\
\cline{3-8}
                           &                         & \multicolumn{2}{c|}{$75$~\\GeV$ < p_{\text{T}}^V < 150$~\\GeV} & \multicolumn{2}{c|}{$150$~\\GeV$ < p_{\text{T}}^V < 250$~\\GeV} & \multicolumn{2}{c|}{$p_{\text{T}}^V > 250$~\\GeV} \\
\cline{3-8}
                           &                         & $2$-jets & $3$-jets & $2$-jets & $3$-jets & $2$-jets & $3$-jets \\
\hline
\hline
$0$-lepton                 & CRLow        & $-$      & $-$      & yields    & yields    & yields    & yields      \\
                           & SR           & $-$      & $-$      & BDT       & BDT       & BDT       & BDT      \\
                           & CRHigh       & $-$      & $-$      & yields    & yields    & yields    & yields      \\
\hline
$1$-lepton                 & CRLow        & $-$ *    & $-$ *    & yields    & yields    & yields    & yields      \\
                           & SR           & $-$ *    & $-$ *    & BDT       & BDT       & BDT       & BDT      \\
                           & CRHigh       & $-$ *    & $-$ *    & yields    & yields    & yields    & yields      \\
\hline
$2$-lepton                 & CRLow        & yields   & yields   & yields    & yields    & yields    & yields      \\
                           & SR           & BDT      & BDT      & BDT       & BDT       & BDT       & BDT      \\
                           & CRHigh       & yields   & yields   & yields    & yields    & yields    & yields      \\
\hline
\multicolumn{8}{|l|}{* previously considered, removed from main result} \\
\hline
\hline

    \end{tabular}
}
  \end{center}
\caption{Event categories and discriminants fitted to data in the 0-, 1- and
  2-lepton channels.}
\label{tabular:FitRegions}
\end{table}
%\section{Fit Models}%
%\label{sec:fit-models}
\section{VH(b,b) multi-variate discriminant fit}%
\label{sec:mva-fit}
\section{Di-jet mass fit}%
\label{sec:mbb-fit}
\section{VZ Cross-check fit}%
\label{sec:mvadiboson-fit}