\chapter{Results}%
\label{ch:results}
\section{\texorpdfstring{\VHbb}{VH->bb} Multi-Variate Discriminant Fit Results}%
\label{sec:mva-results}

\subsection{Nuisance Parameter Pulls}
Describe anything in the pulls and correlations that is particularly relevant to
the parts of the analysis that I worked on, V + jets systematics in particular.

Omit pulls of little significance referring to the internal note.

(+ add in chapter 6 brief description of the treatment of systematics in the
fit, i.e. what do we prune, state that there is smoothing and cite)
% \begin{figure}[hb]
% \centering
% \includegraphics[angle=270]{final_fit_mva/pullComparisons/NP_allExceptGammas.pdf}
% \caption{caption}
% % Nuisance parameter pulls and the free parameter scale factors corresponding to
% % an unconditional combined fit performed to the Asimov dataset (red) and an
% % unconditional combined fit to the \RunTwo data (black).
% \label{fig:nppulls_012L_MVAVH} 
% \end{figure}
%
\begin{figure}[hb]
\centering
%\includegraphics[width=0.49\linewidth]{final_fit_mva/pullComparisons/NP_VH.pdf}
\includegraphics[width=0.49\linewidth]{final_fit_mva/pullComparisons/NP_FloatNorm.pdf}
\includegraphics[width=0.49\linewidth]{final_fit_mva/pullComparisons/NP_Zjets.pdf}
\includegraphics[width=0.49\linewidth]{final_fit_mva/pullComparisons/NP_Wjets.pdf}
\includegraphics[width=0.49\linewidth]{final_fit_mva/pullComparisons/NP_Top.pdf}
\includegraphics[width=0.49\linewidth]{final_fit_mva/pullComparisons/NP_Diboson.pdf}
\includegraphics[width=0.49\linewidth]{final_fit_mva/pullComparisons/NP_MJ.pdf}
\caption{Nuisance parameter pulls and free parameter scale factors relating
  to the backgrounds of the analysis, where an Asimov dataset conditional on
  $\mu=1$ in red is compared with the data in black.}
\label{fig:nppulls_012L_MVAVH_a} 
\end{figure}
%
\begin{figure}[hb]
\centering
\includegraphics[width=0.49\linewidth]{final_fit_mva/pullComparisons/NP_Jet.pdf}
\includegraphics[width=0.49\linewidth]{final_fit_mva/pullComparisons/NP_BTag.pdf}
\includegraphics[width=0.49\linewidth]{final_fit_mva/pullComparisons/NP_Lepton.pdf}
\includegraphics[width=0.49\linewidth]{final_fit_mva/pullComparisons/NP_MET.pdf}
\includegraphics[width=0.49\linewidth]{final_fit_mva/pullComparisons/NP_LUMI.pdf}
\caption{Nuisance parameter pulls and free parameter scale factors relating to the
  experimental sources of uncertainty in the the analysis, where an Asimov
  dataset conditional on $\mu=1$ in red is compared with the data in black.}
\label{fig:nppulls_012L_MVAVH_b}
\end{figure}
%
% \begin{figure}[hb]
% \centering
% \includegraphics[width=0.49\linewidth]{final_fit_mva/pullComparisons/NP_GammasL0.pdf}
% \includegraphics[width=0.49\linewidth]{final_fit_mva/pullComparisons/NP_GammasL1.pdf}
% \includegraphics[width=0.49\linewidth]{final_fit_mva/pullComparisons/NP_GammasL2.pdf}
% \includegraphics[width=0.49\linewidth]{final_fit_mva/pullComparisons/NP_DDttbar.pdf}
% \caption{caption}
% MC stat. nuisance parameter pulls and the free parameter scale factors
% corresponding to $0$-lepton, $1$-lepton and $2$-lepton as well as the Gamma
% parameter pulls for the data driven top template in the $2$-lepton channel
% corresponding to a conditional combined fit to the Asimov dataset (red) and an
% unconditional combined fit to the \RunTwo data (black)
% \label{fig:nppulls_012L_MVAVH_c} 
% \end{figure}

\subsection{Breakdown and Ranking of Uncertainties}
Show the breakdown and ranking of the uncertainties, explain briefly the physics
reasons why things appear where they do. Then later in conc. refer to how to
improve the analysis with reference to the breakdown.
\begin{table}[h]
  \centering
  \begin{tabular}{lrr}
    {\bfseries POI: Central Value} & & \\
    {\bfseries Signal Strength $\bm{\mu=\sigma/\sigma_{\text{SM}}=1.02}$} & & \\ 
    \toprule
    {\bfseries Nuisance Parameter Set} & \multicolumn{2}{l}{\bfseries Impact on error}  \\
                                  & {\bfseries Signed impact} & {\bfseries Un-signed impact}  \\
    \midrule
    Total                    & +0.184 / -0.170 & $ \pm 0.177 $ \\
    & & \\
    DataStat                 & +0.116 / -0.114 & $ \pm 0.115 $ \\
    & & \\
    Data stat only           & +0.109 / -0.107 & $ \pm 0.108 $ \\
    Top $e\mu$ CR stat       & +0.014 / -0.014 & $ \pm 0.014 $ \\
    Floating normalizations  & +0.035 / -0.033 & $ \pm 0.034 $ \\
    & & \\
    FullSyst                 & +0.143 / -0.126 & $ \pm 0.134 $ \\
    Modelling: \VH           & +0.083 / -0.061 & $ \pm 0.072 $ \\
    Modelling: Background    & +0.068 / -0.064 & $ \pm 0.066 $ \\
    Multi Jet                & +0.004 / -0.006 & $ \pm 0.005 $ \\
    Modelling: single top    & +0.020 / -0.019 & $ \pm 0.019 $ \\
    Modelling: $t\bar{t}$    & +0.021 / -0.020 & $ \pm 0.021 $ \\
    Modelling: $W+$jets      & +0.041 / -0.038 & $ \pm 0.040 $ \\
    Modelling: $Z+$jets      & +0.032 / -0.032 & $ \pm 0.032 $ \\
    Modelling: Diboson       & +0.034 / -0.031 & $ \pm 0.033 $ \\
    MC stat                  & +0.031 / -0.030 & $ \pm 0.031 $ \\
    & & \\
    Experimental Syst        & +0.079 / -0.069 & $ \pm 0.074 $ \\
    Detector: lepton         & +0.004 / -0.005 & $ \pm 0.004 $ \\
    Detector: MET            & +0.015 / -0.014 & $ \pm 0.015 $ \\
    Detector: JET            & +0.048 / -0.038 & $ \pm 0.043 $ \\
    Detector: FTAG (b-jet)   & +0.047 / -0.042 & $ \pm 0.045 $ \\
    Detector: FTAG (c-jet)   & +0.037 / -0.033 & $ \pm 0.035 $ \\
    Detector: FTAG (l-jet)   & +0.009 / -0.009 & $ \pm 0.009 $ \\
    Detector: FTAG (extrap)  & +0.000 / -0.000 & $ \pm 0.000 $ \\
    Detector: PU             & +0.002 / -0.003 & $ \pm 0.003 $ \\
    Luminosity               & +0.019 / -0.013 & $ \pm 0.016 $ \\
    \bottomrule
  \end{tabular}
\caption{caption}
  % The breakdown of the uncertainties coming from data statistics (``Data
  % Stat.''), systematic uncertainties together with MC statistical uncertainties
  % (``Full Syst.''), MC statistics only (``MC Stat.''), and different sub-groups
  % of systematic uncertainties, is shown in the three combined channels fit to
  % the Run 2 data to extract $\mu_{VH}$.
\label{tab:breakdown_012L_MVAVH}
\end{table}
\begin{figure}[ht]
  \centering
  %\includegraphics[width=0.49\linewidth]{final_fit_mva/ranking/pulls_SigXsecOverSM_prefit_125_NOSignalUnc.pdf} \\
  \includegraphics[width=0.75\linewidth]{final_fit_mva/ranking/pulls_SigXsecOverSM_125_SignalUnc.pdf}
  \caption{The top 15 nuisance parameters are shown ranked based on their impact
    on the \VHbb\ signal strength $\mu$ as determined by the combined
    unconditional fit to data.}
  \label{fig:Rank_012L_MVAVH}
\end{figure}
\begin{figure}[hb]
  \centering
  % \includegraphics[width=0.49\linewidth]{final_fit_mva/ranking/pulls_SigXsecOverSM_prefit_125_NOSignalUnc.pdf} \\
  \includegraphics[width=0.75\linewidth]{final_fit_mva/ranking/pulls_SigXsecOverSM_125_NOSignalUnc.pdf}
  \caption{The top 15 nuisance parameters are shown, omitting signal
    uncertainties, ranked based on their impact on the \VHbb\ signal strength
    $\mu$ as determined by the combined unconditional fit to data.}
  \label{fig:Rank_012L_MVAVH_nosig}
\end{figure}


\subsection{Post-fit Data Versus Predictions}
Describe agreement here.
\begin{figure}
  \centering
  \begin{tabular}{cc}
    % top row
    \includegraphics[width=.3\textwidth]{final_fit_mva/postfit/Region_BMax250_BMin150_Y6051_DCRHigh_{\mathrm{T}}2_L0_distMET_J2_GlobalFit_unconditionnal_mu1}%
    & \includegraphics[width=.3\textwidth]{final_fit_mva/postfit/Region_BMin250_Y6051_DCRHigh_{\mathrm{T}}2_L0_distMET_J2_GlobalFit_unconditionnal_mu1} \\

    % middle row
    \includegraphics[width=.3\textwidth]{final_fit_mva/postfit/Region_BMax250_BMin150_Y6051_DSR_{\mathrm{T}}2_L0_distmva_J2_GlobalFit_unconditionnal_mu1}%
    & \includegraphics[width=.3\textwidth]{final_fit_mva/postfit/Region_BMin250_Y6051_DSR_{\mathrm{T}}2_L0_distmva_J2_GlobalFit_unconditionnal_mu1} \\

    % bottom row
    \includegraphics[width=.3\textwidth]{final_fit_mva/postfit/Region_BMax250_BMin150_Y6051_DCRLow_{\mathrm{T}}2_L0_distMET_J2_GlobalFit_unconditionnal_mu1}%
    & \includegraphics[width=.3\textwidth]{final_fit_mva/postfit/Region_BMin250_Y6051_DCRLow_{\mathrm{T}}2_L0_distMET_J2_GlobalFit_unconditionnal_mu1} \\
  \end{tabular}
  \caption{Post-fit distributions in the 0--lepton 2--jet channel.}
\end{figure}
\begin{figure}
  \centering
  \begin{tabular}{cc}
    % top row
    \includegraphics[width=.3\textwidth]{final_fit_mva/postfit/Region_BMax250_BMin150_Y6051_DCRHigh_T2_L0_distMET_J3_GlobalFit_unconditionnal_mu1}%
    & \includegraphics[width=.3\textwidth]{final_fit_mva/postfit/Region_BMin250_Y6051_DCRHigh_T2_L0_distMET_J3_GlobalFit_unconditionnal_mu1} \\

    % middle row
    \includegraphics[width=.3\textwidth]{final_fit_mva/postfit/Region_BMax250_BMin150_Y6051_DSR_T2_L0_distmva_J3_GlobalFit_unconditionnal_mu1}%
    & \includegraphics[width=.3\textwidth]{final_fit_mva/postfit/Region_BMin250_Y6051_DSR_T2_L0_distmva_J3_GlobalFit_unconditionnal_mu1} \\

    % bottom row
    \includegraphics[width=.3\textwidth]{final_fit_mva/postfit/Region_BMax250_BMin150_Y6051_DCRLow_T2_L0_distMET_J3_GlobalFit_unconditionnal_mu1}%
    & \includegraphics[width=.3\textwidth]{final_fit_mva/postfit/Region_BMin250_Y6051_DCRLow_T2_L0_distMET_J3_GlobalFit_unconditionnal_mu1} \\
  \end{tabular}
  \caption{Post-fit distributions in the 0--lepton 3--jet channel.}
\end{figure}
\begin{figure}
  \centering
  \begin{tabular}{cc}
    % top row
    %\includegraphics[width=.3\textwidth]{final_fit_mva/postfit/Region_BMin0_Y6051_DCRHigh_T2_L1_distMET_J2_GlobalFit_unconditionnal_mu1}%
    \includegraphics[width=.3\textwidth]{final_fit_mva/postfit/Region_BMax250_BMin150_Y6051_DCRHigh_T2_L1_distpTV_J2_GlobalFit_unconditionnal_mu1}%
    & \includegraphics[width=.3\textwidth]{final_fit_mva/postfit/Region_BMin250_Y6051_DCRHigh_T2_L1_distpTV_J2_GlobalFit_unconditionnal_mu1} \\

    % middle row
    %\includegraphics[width=.3\textwidth]{}%
    \includegraphics[width=.3\textwidth]{final_fit_mva/postfit/Region_BMax250_BMin150_Y6051_DSR_T2_L1_distmva_J2_GlobalFit_unconditionnal_mu1}%
    & \includegraphics[width=.3\textwidth]{final_fit_mva/postfit/Region_BMin250_Y6051_DSR_T2_L1_distmva_J2_GlobalFit_unconditionnal_mu1} \\

    % bottom row
    %\includegraphics[width=.3\textwidth]{it}%
    \includegraphics[width=.3\textwidth]{final_fit_mva/postfit/Region_BMax250_BMin150_Y6051_DCRLow_T2_L1_distpTV_J2_GlobalFit_unconditionnal_mu1}%
    & \includegraphics[width=.3\textwidth]{final_fit_mva/postfit/Region_BMin250_Y6051_DCRLow_T2_L1_distpTV_J2_GlobalFit_unconditionnal_mu1} \\
    
    % (a) first & (b) second \\[6pt]
    %(c) third & (d) fourth \\[6pt]
    %\multicolumn{2}{c}{\includegraphics[width=65mm]{it} }\\
    %\multicolumn{2}{c}{(e) fifth}
  \end{tabular}
  \caption{caption}
\end{figure}
\begin{figure}
  \centering
  \begin{tabular}{cc}
    % top row
    \includegraphics[width=.3\textwidth]{final_fit_mva/postfit/Region_BMax250_BMin150_Y6051_DCRHigh_T2_L1_distpTV_J3_GlobalFit_unconditionnal_mu1}%
    & \includegraphics[width=.3\textwidth]{final_fit_mva/postfit/Region_BMin250_Y6051_DCRHigh_T2_L1_distpTV_J3_GlobalFit_unconditionnal_mu1} \\

    % middle row
    \includegraphics[width=.3\textwidth]{final_fit_mva/postfit/Region_BMax250_BMin150_Y6051_DSR_T2_L1_distmva_J3_GlobalFit_unconditionnal_mu1}%
    & \includegraphics[width=.3\textwidth]{final_fit_mva/postfit/Region_BMin250_Y6051_DSR_T2_L1_distmva_J3_GlobalFit_unconditionnal_mu1} \\

    % bottom row
    \includegraphics[width=.3\textwidth]{final_fit_mva/postfit/Region_BMax250_BMin150_Y6051_DCRLow_T2_L1_distpTV_J3_GlobalFit_unconditionnal_mu1}%
    & \includegraphics[width=.3\textwidth]{final_fit_mva/postfit/Region_BMin250_Y6051_DCRLow_T2_L1_distpTV_J3_GlobalFit_unconditionnal_mu1} \\
  \end{tabular}
  \caption{Post-fit distributions in the 1 lepton 3 jet channel.}
\end{figure}
\begin{figure}
  \centering
  \begin{tabular}{cc}
    % top row
    \includegraphics[width=.33\textwidth]{final_fit_mva/postfit/Region_BMax150_BMin75_Y6051_DCRHigh_T2_L2_distpTV_J2_GlobalFit_unconditionnal_mu1}%
    \includegraphics[width=.33\textwidth]{final_fit_mva/postfit/Region_BMax250_BMin150_Y6051_DCRHigh_T2_L2_distpTV_J2_GlobalFit_unconditionnal_mu1}%
    & \includegraphics[width=.33\textwidth]{final_fit_mva/postfit/Region_BMin250_Y6051_DCRHigh_T2_L2_distpTV_J2_GlobalFit_unconditionnal_mu1} \\

    % middle row
    \includegraphics[width=.33\textwidth]{final_fit_mva/postfit/Region_BMax150_BMin75_Y6051_DSR_T2_L2_distmva_J2_GlobalFit_unconditionnal_mu1}%
    \includegraphics[width=.33\textwidth]{final_fit_mva/postfit/Region_BMax250_BMin150_Y6051_DSR_T2_L2_distmva_J2_GlobalFit_unconditionnal_mu1}%
    & \includegraphics[width=.33\textwidth]{final_fit_mva/postfit/Region_BMin250_Y6051_DSR_T2_L2_distmva_J2_GlobalFit_unconditionnal_mu1} \\

    % bottom row
    \includegraphics[width=.33\textwidth]{final_fit_mva/postfit/Region_BMax150_BMin75_Y6051_DCRLow_T2_L2_distpTV_J2_GlobalFit_unconditionnal_mu1}%
    \includegraphics[width=.33\textwidth]{final_fit_mva/postfit/Region_BMax250_BMin150_Y6051_DCRLow_T2_L2_distpTV_J2_GlobalFit_unconditionnal_mu1}%
    & \includegraphics[width=.33\textwidth]{final_fit_mva/postfit/Region_BMin250_Y6051_DCRLow_T2_L2_distpTV_J2_GlobalFit_unconditionnal_mu1} \\
  \end{tabular}
  \caption{Post-fit distributions in the 2--lepton 2--jet channel.}
\end{figure}
\begin{figure}
  \centering
  \begin{tabular}{cc}
    % top row
    \includegraphics[width=.3\textwidth]{final_fit_mva/postfit/Region_BMax150_BMin75_incJet1_Y6051_DCRHigh_T2_L2_distpTV_J3_GlobalFit_unconditionnal_mu1}%
    \includegraphics[width=.3\textwidth]{final_fit_mva/postfit/Region_BMax250_BMin150_incJet1_Y6051_DCRHigh_T2_L2_distpTV_J3_GlobalFit_unconditionnal_mu1}%
    & \includegraphics[width=.3\textwidth]{final_fit_mva/postfit/Region_BMin250_incJet1_Y6051_DCRHigh_T2_L2_distpTV_J3_GlobalFit_unconditionnal_mu1} \\

    % middle row
    \includegraphics[width=.3\textwidth]{final_fit_mva/postfit/Region_BMax150_BMin75_incJet1_Y6051_DSR_T2_L2_distmva_J3_GlobalFit_unconditionnal_mu1}%
    \includegraphics[width=.3\textwidth]{final_fit_mva/postfit/Region_BMax250_BMin150_incJet1_Y6051_DSR_T2_L2_distmva_J3_GlobalFit_unconditionnal_mu1}%
    & \includegraphics[width=.3\textwidth]{final_fit_mva/postfit/Region_BMin250_incJet1_Y6051_DSR_T2_L2_distmva_J3_GlobalFit_unconditionnal_mu1} \\

    % bottom row
    \includegraphics[width=.3\textwidth]{final_fit_mva/postfit/Region_BMax150_BMin75_incJet1_Y6051_DCRLow_T2_L2_distpTV_J3_GlobalFit_unconditionnal_mu1}%
    \includegraphics[width=.3\textwidth]{final_fit_mva/postfit/Region_BMax250_BMin150_incJet1_Y6051_DCRLow_T2_L2_distpTV_J3_GlobalFit_unconditionnal_mu1}%
    & \includegraphics[width=.3\textwidth]{final_fit_mva/postfit/Region_BMin250_incJet1_Y6051_DCRLow_T2_L2_distpTV_J3_GlobalFit_unconditionnal_mu1} \\
  \end{tabular}
  \caption{Post-fit distributions in the 2 lepton, 3+ jets channel.}
\end{figure}

\subsection{Signal Strength and STXS Measurements}
Show the signal strength measurements and STXS measurements, everything agrees
with the standard model so there is not much to state beyond that.

(+ add in chapter 6 mention the multiple POI version of the fit which yields the
STXS measurement, briefly describe what this is defining the Njet - NHjet
procedure to get the STXS bins, i.e. define the fiducial space)
\begin{figure}[!htbp]
	\centering
	\includegraphics[width=0.7\linewidth]{final_fit_mva/results/Global_SoverB_Y6051_details_pulls.pdf}
  \caption{caption}
  % Data events from the full Run 2 dataset distributed as a function of
  % $\log_{10}(S/B)$ and overlaid on MC predictions after applying the pulls from
  % data.
  \label{fig:sobplot}
\end{figure}

\subsubsection{\VH\ Signal Strength Measurement}
Table~\ref{tab:sig_channels} shows the expected and observed significances for
the background plus signal hypothesis broken down by analysis channel.
\begin{table}[h]
  \centering
  \begin{tabular}{lrr}
    \toprule
    {\bfseries Channel}  & {\bfseries Expected sig. (cond. data)} & {\bfseries Observed sig.} \\
    \midrule
    0L         & 4.1               &	4.4	    \\
    1L         & 4.1               &	4.0	    \\
    2L         & 4.3               &	4.2	    \\
    0L+1L+2L   & 6.7               &	6.7	    \\
    \bottomrule
  \end{tabular}
  \caption{caption}
  % Expected significances for a single signal strength, or three
  % corresponding to the three leptonic channel (3-$\mu$ fit), in the combined
  % fits of the mva discriminant in the three channels to the cond. data Asimov
  % dataset (data fit with $\mu=1$) and to Run 2 data.
  \label{tab:sig_channels}
\end{table}
The expected significances were determined using an Asimov dataset conditional
on $\mu=1$. All channels are approaching the 5$\sigma$ threshold to claim
discovery and the combined fit is well above the discovery threshold in
agreement with the previous result~\cite{vhbb-obs}. Figure~\ref{fig:channel-mus}
shows the best fit values for the signal strength in each channel and in the
combined paradigm.
\begin{figure}[ht]
	\centering
	\includegraphics[width=0.7\linewidth]{final_fit_mva/results/Plot_mu_0_DecorrPOI_VH.pdf}
  \caption[Best fit values for signal strength broken down by analysis
  channel.]{Best fit values of the signal strength for the 0--, 1-- and 2--
    channels, as well as the combined (Comb.) measurement. Uncertainties are
    broken down into either statistical (Stat.) or systematic (Syst.), with
    those due to scale factors on background processes counted as statistical.}
  \label{fig:channel-mus}
\end{figure}
All measurements are compatible with the standard model prediction of $\mu=1$,
the composition and values of the uncertainties affecting each channel are
similar with the 2--lepton channel being slightly more impacted by statistical
rather than systematic uncertainty.
%\begin{figure}[!htbp]
	\centering
	\includegraphics[width=0.47\linewidth]{final_fit_mva/results/Plot_mu_1_DecorrPOI_J_VH.pdf}
	\includegraphics[width=0.47\linewidth]{final_fit_mva/results/Plot_mu_0_DecorrBMin_VH.pdf}
	\includegraphics[width=0.47\linewidth]{final_fit_mva/results/Plot_mu_15_DecorrPOI_J_L_VH.pdf}
	\includegraphics[width=0.47\linewidth]{final_fit_mva/results/Plot_mu_15_DecorrPOI_BMin_L_VH.pdf}
  \caption{caption}
  % Best values of the signal strength $\mu_{\VH}^{b\bar{b}}$ uncorrelated
  % (multi-$\mu$) between $N($jet$)$ regions (top left), \pTV regions (top right),
  % $N($jet$)$ regions and lepton channels (bottom left), \pTV regions and lepton
  % channels (bottom right) and their combination in the combined unconditional
  % fit to the Run 2 data. The (black) total observed uncertainty is quoted
  % together with its decomposition in the (green) statistical component, and
  % systematic component. In this plot the uncertainty due to background scale
  % factors is included in the statistical component.
\label{fig:channels-mus-uncorr}
\end{figure}

\subsubsection{\WH\ and \ZH\ Signal Strength Measurements}
Table~\ref{tab:sig_WZH} shows the expected and observed significances for
the background plus signal hypothesis for the \WH\, \ZH\ and \VH measurements.
\begin{table}[ht]
  \centering
  \begin{tabular}{lrr}
    \toprule
    {\bfseries Channel}  & {\bfseries Expected} & {\bfseries Observed} \\
    \midrule
    \WH         & 4.1       &	4.0	\\
    \ZH         & 5.1       &	5.3	    \\
    \VH         & 6.7      	&	6.7	    \\
    \bottomrule
  \end{tabular}
  \caption{Statistical significances for the background plus signal hypothesis
    for an Asimov dataset conditional on $\mu=1$, called expected and for the
    data, called observed. All values are given standard deviations
    ($\sigma$s). Significances are shown the \WH\, \ZH\ and \VH measurements.}
  \label{tab:sig_WZH}
\end{table}
The expected significances were determined using an Asimov dataset conditional
on $\mu=1$. The \WH\ stand alone measurement is approaching the 5$\sigma$
threshold to claim discovery. The \ZH\ stand alone measurement has surpassed the
threshold and the paper associated with this measurement is first proof of
discovery published. Figure~\ref{fig:WZH-mus}
shows the best fit values for the signal strength for each measurement.
\begin{figure}[hb]
	\centering
	\includegraphics[width=0.7\linewidth]{final_fit_mva/results/Plot_mu_1_STXSFitScheme2_VH.pdf}
  \caption{caption}
  % Best values of the signal strengths associated to the $WH$ and $ZH$
  % production modes and their combination in the combined fit on the
  % three channels to the Run 2 data. The (black) total observed
  % uncertainty is quoted together with its decomposition in the (green)
  % statistical component, and systematic component. In this plot the
  % uncertainty due to background scale factors is included in the
  % statistical component.}
  \label{fig:WZH-mus}
\end{figure}
All measurements are compatible with the standard model prediction of $\mu=1$,
the composition and values of the uncertainties affecting each channel are very
similar.


\subsubsection{Simplified Template Cross-section Measurements}
\begin{figure}[hb]
	\centering
	\includegraphics[width=0.7\linewidth]{final_fit_mva/results/Plot_mu_5POI.pdf}
  \caption[Best fit values for cross-section times branching ratio in each STXS
  bin.]{Best fit values of the cross-section times branching ratio ($\sigma
    \times B$) in each of the STXS bins. Uncertainties quoted are broken down by
    source, either statistical (Stat.) or systematic (Syst.). Uncertainties are
    broken down into either statistical (Stat.) or systematic (Syst.), with
    those due to scale factors on background processes counted as statistical.
    The theoretical uncertainty on the SM prediction of is shown in grey.}
  \label{fig:5POI-mus}
\end{figure}
\begin{figure}[hb]
  \centering
  \includegraphics[width=0.7\linewidth]{final_fit_mva/results/corr_XS_5POI.pdf}
  %% plot already updated to the 5 POI XS
  \caption{caption}
  % Correlations in the measure of the ratio of the
  % $\sigma\times\text{BR}$ over the SM for the $\mathrm{WH}_{150-250}$,
  % $\mathrm{WH}_{\geq 250}$, $\mathrm{ZH}_{75-150}$, $\mathrm{ZH}_{150-250}$,
  % and $\mathrm{ZH}_{\geq 250}$ processes in the unconditional fit to data.
  \label{fig:5POI-corr}
\end{figure}
\begin{table}[!htbp]
  \setlength{\extrarowheight}{4pt}
  \centering
    \resizebox{\textwidth}{!}{
      \begin{tabular}{lrrrrrr}
        \toprule
        Measurement region & SM prediction fb  & Result fb & Stat. unc. fb & \multicolumn{3}{c}{Syst. unc. fb} \\
                           &                   &           &               & Th. sig. & Th. bkg.  & Exp. \\
        \midrule
        $W\rightarrow \ell\nu$; $150<p_{\text{T}}^{V}<250$~\GeV\         & $24.0 \pm 1.1$ & $19.0^{+12.2}_{-12.0} $ & $^{+7.7}_{-7.7}$ & $^{+1.2}_{-0.5}$ & $^{+5.5}_{-5.5}$  & $^{+6.0}_{-6.0}$ \\
        $W\rightarrow \ell\nu$; $p_{\text{T}}^{V}>250$~\GeV\             & $7.1 \pm 0.3$ & $7.2^{+2.3}_{-2.1}  $   & $^{+1.9}_{-1.8}$  & $^{+0.4}_{-0.3}$ & $^{+0.8}_{-0.8}$  & $^{+0.7}_{-0.6}$ \\
        $Z\rightarrow \ell\ell,\nu\nu$; $75<p_{\text{T}}^{V}<150$~\GeV\     & $50.6 \pm 4.1$ & $42.5^{+36.4}_{-35.4}$   & $^{+25.3}_{-25.3}$   &  $^{+6.6}_{-4.6}$ & $^{+17.2}_{-17.2}$ & $^{+20.2}_{-19.2}$ \\
        $Z\rightarrow \ell\ell,\nu\nu$; $150<p_{\text{T}}^{V}<250$~\GeV\    & $18.8 \pm 2.4$        & $20.5^{+6.4}_{-6.0}$   & $^{+5.1}_{-4.9}$   & $^{+2.3}_{-2.3}$ &  $^{+2.4}_{-2.3}$ & $^{+2.4}_{-2.1}$ \\
        $Z\rightarrow \ell\ell,\nu\nu$; $p_{\text{T}}^{V}>250$~\GeV\        & $4.9 \pm 0.5$        & $5.4^{+1.7}_{-1.6} $   & $^{+1.5}_{-1.4}$    & $^{+0.6}_{-0.3}$  & $^{+0.5}_{-0.5}$ & $^{+0.3}_{-0.3}$ \\
        \bottomrule
      \end{tabular}
    }
    \caption{caption, ($|y_H|<2.5$, $H\to b\bar{b}$)}
    % Best-fit values and uncertainties for the $VH$,
    % $V\to$~leptons for the cross-section
    % times the $H\to b\bar{b}$ branching ratio, in the reduced STXS scheme.
    % The SM predictions for each region, computed using the inclusive
    % cross-section
    % calculations and the simulated event samples are also shown.
    % The contributions to the total uncertainty in the measurements from
    % statistical (Stat.~unc.) or systematic uncertainties (Syst.~unc.) in
    % the signal modelling (Th.~sig.), background modelling (Th.~bkg.),
    % and in experimental performance (Exp.) are given separately.
    % The total systematic uncertainty, equal to the difference in quadrature between
    % the total uncertainty and the statistical uncertainty, differs from the
    % sum in quadrature of the Th. Sig., Th. Bkg., and Exp. systematic
    % uncertainties due to correlations.
    % All leptonic decays of the $V$ bosons (including those to $\tau$-leptons, $\ell=e,\mu,\tau$) are considered.
    \label{tab:XS3and5POI}
\end{table}
\begin{figure}[!htbp]
	\centering
	\includegraphics[width=0.7\linewidth]{final_fit_mva/results/Fancy_SM_5POI.pdf}
  \caption{caption}
  % Best values of the $\sigma\times \text{BR}$ for
  % $\mathrm{WH}_{150-250}$, $\mathrm{WH}_{\geq 250}$,
  % $\mathrm{ZH}_{75-150}$, $\mathrm{ZH}_{150-250}$, and
  % $\mathrm{ZH}_{\geq 250}$ in the unconditional fit to data. The
  % (black) total observed uncertainty is quoted together with the
  % (green) statistical component. The observation is compared to the
  % SM prediction, marked with the red lines, and its uncertainty,
  % represented by the red area.
  \label{fig:5POI-fancyMU}
\end{figure}
\begin{table}[h]
  \centering
  \begin{tabular}{lrr}
    \toprule
    {\bfseries Channel}  & {\bfseries Expected} & {\bfseries Observed} \\
    \midrule
    \WH, $150~\GeV<p_{\mathrm{T}}^{\text{miss}}<250~\GeV$     & 2.0               &	1.6	    \\
    \WH, $p_{\mathrm{T}}^{\text{miss}} > 250~\GeV$            & 3.4               &	3.6	    \\
    \ZH, $75~\GeV<p_{\mathrm{T}}^{\text{miss}}<150~\GeV$      & 1.4               &	1.2	    \\
    \ZH, $150~\GeV<p_{\mathrm{T}}^{\text{miss}}<250~\GeV$     & 3.4               &	3.6	    \\
    \ZH, $p_{\mathrm{T}}^{\text{miss}} > 250~\GeV$            & 3.5               &	3.6	    \\
    \VH                                       & 6.7               &	6.7	    \\
    \bottomrule
  \end{tabular}
  \caption{Statistical significances on the \VHbb\ cross-section measurement in
    each STXS bin for an Asimov dataset conditional on $\mu=1$, called expected
    and for the data, called observed. All values are given standard deviations
    ($\sigma$s). Significances are shown alongside the result from the single
    parameter combined fit, denoted \VH.}
  \label{tab:sig_stxs}
\end{table}