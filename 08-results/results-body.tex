\chapter{Results}%
\label{ch:results}
In this chapter the results of the \VHbb\ multi-variate discriminant fit are
shown. Other results such as those for the cross-checks of the analysis can be
found in~\cite{VHMainNote2019}. The fit is performed in accordance with the
asymptotic approximation~\cite{Cowan:2010js} for calculating the $p_0$ values,
which are converted to statistical significances expressed in terms of standard
deviations. An Asimov dataset conditional on $\mu=1$  is generated for use in
the above approximation and is used in a fit whose nuisance parameter pulls are
compared with those coming from the unconditional fit to the data that is used
to generate the final results. As per the prescription of the asymptotic
approximation the results will be discussed in the context of a hypothesis test
comparing the null hypothesis, the background only model, with the alternative
hypothesis, the background plus signal model.

My contributions to the results are as already stated in other chapters plus
running the combined, and 2--lepton channel fits to generate and study the
results at many key milestones in the publication timeline including the final
results in the paper~\cite{final-paper}.

\section{Nuisance Parameter Pulls}
Figure~\ref{fig:nppulls_012L_MVAVH_a} shows the nuisance parameter pulls coming
from the combined profile-likelihood fit for the $W+$jets, $Z+$jets and top
process backgrounds.
% \begin{figure}[hb]
% \centering
% \includegraphics[angle=270]{final_fit_mva/pullComparisons/NP_allExceptGammas.pdf}
% \caption{caption}
% % Nuisance parameter pulls and the free parameter scale factors corresponding to
% % an unconditional combined fit performed to the Asimov dataset (red) and an
% % unconditional combined fit to the \RunTwo data (black).
% \label{fig:nppulls_012L_MVAVH} 
% \end{figure}
%
\begin{figure}[hb]
\centering
%\includegraphics[width=0.49\linewidth]{final_fit_mva/pullComparisons/NP_VH.pdf}
\includegraphics[width=0.49\linewidth]{final_fit_mva/pullComparisons/NP_FloatNorm.pdf}
\includegraphics[width=0.49\linewidth]{final_fit_mva/pullComparisons/NP_Zjets.pdf}
\includegraphics[width=0.49\linewidth]{final_fit_mva/pullComparisons/NP_Wjets.pdf}
\includegraphics[width=0.49\linewidth]{final_fit_mva/pullComparisons/NP_Top.pdf}
\includegraphics[width=0.49\linewidth]{final_fit_mva/pullComparisons/NP_Diboson.pdf}
\includegraphics[width=0.49\linewidth]{final_fit_mva/pullComparisons/NP_MJ.pdf}
\caption{Nuisance parameter pulls and free parameter scale factors relating
  to the backgrounds of the analysis, where an Asimov dataset conditional on
  $\mu=1$ in red is compared with the data in black.}
\label{fig:nppulls_012L_MVAVH_a} 
\end{figure}
%
\begin{figure}[hb]
\centering
\includegraphics[width=0.49\linewidth]{final_fit_mva/pullComparisons/NP_Jet.pdf}
\includegraphics[width=0.49\linewidth]{final_fit_mva/pullComparisons/NP_BTag.pdf}
\includegraphics[width=0.49\linewidth]{final_fit_mva/pullComparisons/NP_Lepton.pdf}
\includegraphics[width=0.49\linewidth]{final_fit_mva/pullComparisons/NP_MET.pdf}
\includegraphics[width=0.49\linewidth]{final_fit_mva/pullComparisons/NP_LUMI.pdf}
\caption{Nuisance parameter pulls and free parameter scale factors relating to the
  experimental sources of uncertainty in the the analysis, where an Asimov
  dataset conditional on $\mu=1$ in red is compared with the data in black.}
\label{fig:nppulls_012L_MVAVH_b}
\end{figure}
%
% \begin{figure}[hb]
% \centering
% \includegraphics[width=0.49\linewidth]{final_fit_mva/pullComparisons/NP_GammasL0.pdf}
% \includegraphics[width=0.49\linewidth]{final_fit_mva/pullComparisons/NP_GammasL1.pdf}
% \includegraphics[width=0.49\linewidth]{final_fit_mva/pullComparisons/NP_GammasL2.pdf}
% \includegraphics[width=0.49\linewidth]{final_fit_mva/pullComparisons/NP_DDttbar.pdf}
% \caption{caption}
% MC stat. nuisance parameter pulls and the free parameter scale factors
% corresponding to $0$-lepton, $1$-lepton and $2$-lepton as well as the Gamma
% parameter pulls for the data driven top template in the $2$-lepton channel
% corresponding to a conditional combined fit to the Asimov dataset (red) and an
% unconditional combined fit to the \RunTwo data (black)
% \label{fig:nppulls_012L_MVAVH_c} 
% \end{figure}
The other nuisance parameter pulls have been omitted but have been
studied~\cite{VHMainNote2019}, and the \texttt{Sys} has been dropped from the
names of the parameters displayed. Observation of the pull on \texttt{SysZMbb}
shows that it is pulled towards the $+1\sigma$ value which meets the
expectations based on the shapes derived in section~\ref{sec:zjets-shapes}. The
uncorrelated component of the shape, which is pulled in that direction but to a
lesser degree this also matches expectations. Several of the nuisance parameters
relating to the BDTr shape uncertainties are pulled to either $+1\sigma$ or
$-1\sigma$ as would be expected, whilst several have not been pulled much which
goes against naive expectations. There could be many explanations for this but
as the phase space is split up into many different regions limiting the
statistical power of the method it is best not to over-interpret. In order to
understand the behaviour of the uncertainties, the correlations between nuisance
parameter pulls, shown in figure~\ref{fig:np-corrs}, must be studied.
\begin{landscape}
  \begin{figure}[hb]
	\centering
	\includegraphics[width=0.8\paperheight]{final_fit_mva/pullComparisons/corr_HighCorrNoMCStat}
  \caption[Correlations between nuisance parameter pulls.]{Correlations between
    nuisance parameter pulls where any nuisance parameter with a correlation of
    at least 25~\% is shown and all other are omitted.}
  \label{fig:np-corrs}
\end{figure}
\end{landscape}
There are no correlations which indicate an unexpected relationship between the
relevant quantities.

\section{Breakdown and Ranking of Uncertainties}
Table~\ref{tab:breakdown_012L_MVAVH} shows a breakdown of the impact of the
systematic uncertainties on the signal strength measurement.
\begin{table}[h]
  \centering
  \begin{tabular}{lrr}
    {\bfseries POI: Central Value} & & \\
    {\bfseries Signal Strength $\bm{\mu=\sigma/\sigma_{\text{SM}}=1.02}$} & & \\ 
    \toprule
    {\bfseries Nuisance Parameter Set} & \multicolumn{2}{l}{\bfseries Impact on error}  \\
                                  & {\bfseries Signed impact} & {\bfseries Un-signed impact}  \\
    \midrule
    Total                    & +0.184 / -0.170 & $ \pm 0.177 $ \\
    & & \\
    DataStat                 & +0.116 / -0.114 & $ \pm 0.115 $ \\
    & & \\
    Data stat only           & +0.109 / -0.107 & $ \pm 0.108 $ \\
    Top $e\mu$ CR stat       & +0.014 / -0.014 & $ \pm 0.014 $ \\
    Floating normalizations  & +0.035 / -0.033 & $ \pm 0.034 $ \\
    & & \\
    FullSyst                 & +0.143 / -0.126 & $ \pm 0.134 $ \\
    Modelling: \VH           & +0.083 / -0.061 & $ \pm 0.072 $ \\
    Modelling: Background    & +0.068 / -0.064 & $ \pm 0.066 $ \\
    Multi Jet                & +0.004 / -0.006 & $ \pm 0.005 $ \\
    Modelling: single top    & +0.020 / -0.019 & $ \pm 0.019 $ \\
    Modelling: $t\bar{t}$    & +0.021 / -0.020 & $ \pm 0.021 $ \\
    Modelling: $W+$jets      & +0.041 / -0.038 & $ \pm 0.040 $ \\
    Modelling: $Z+$jets      & +0.032 / -0.032 & $ \pm 0.032 $ \\
    Modelling: Diboson       & +0.034 / -0.031 & $ \pm 0.033 $ \\
    MC stat                  & +0.031 / -0.030 & $ \pm 0.031 $ \\
    & & \\
    Experimental Syst        & +0.079 / -0.069 & $ \pm 0.074 $ \\
    Detector: lepton         & +0.004 / -0.005 & $ \pm 0.004 $ \\
    Detector: MET            & +0.015 / -0.014 & $ \pm 0.015 $ \\
    Detector: JET            & +0.048 / -0.038 & $ \pm 0.043 $ \\
    Detector: FTAG (b-jet)   & +0.047 / -0.042 & $ \pm 0.045 $ \\
    Detector: FTAG (c-jet)   & +0.037 / -0.033 & $ \pm 0.035 $ \\
    Detector: FTAG (l-jet)   & +0.009 / -0.009 & $ \pm 0.009 $ \\
    Detector: FTAG (extrap)  & +0.000 / -0.000 & $ \pm 0.000 $ \\
    Detector: PU             & +0.002 / -0.003 & $ \pm 0.003 $ \\
    Luminosity               & +0.019 / -0.013 & $ \pm 0.016 $ \\
    \bottomrule
  \end{tabular}
\caption{caption}
  % The breakdown of the uncertainties coming from data statistics (``Data
  % Stat.''), systematic uncertainties together with MC statistical uncertainties
  % (``Full Syst.''), MC statistics only (``MC Stat.''), and different sub-groups
  % of systematic uncertainties, is shown in the three combined channels fit to
  % the Run 2 data to extract $\mu_{VH}$.
\label{tab:breakdown_012L_MVAVH}
\end{table}
It can be seen that the total statistical uncertainty (DataStat) has a smaller
impact than the uncertainties due to modelling and experimental sources
(FullSyst). Considering the systematic uncertainties, those coming from
experimental sources have a greater impact than those coming from the modelling
of the signal, which in turn have a greater impact than those coming from the
modelling of the background. Further scrutiny of uncertainties on the signal
strength comes in the form of the ranking shown in
figures~\ref{fig:Rank_012L_MVAVH} and~\ref{fig:Rank_012L_MVAVH_nosig}.
\begin{figure}[ht]
  \centering
  %\includegraphics[width=0.49\linewidth]{final_fit_mva/ranking/pulls_SigXsecOverSM_prefit_125_NOSignalUnc.pdf} \\
  \includegraphics[width=0.75\linewidth]{final_fit_mva/ranking/pulls_SigXsecOverSM_125_SignalUnc.pdf}
  \caption{The top 15 nuisance parameters are shown ranked based on their impact
    on the \VHbb\ signal strength $\mu$ as determined by the combined
    unconditional fit to data.}
  \label{fig:Rank_012L_MVAVH}
\end{figure}
\begin{figure}[hb]
  \centering
  % \includegraphics[width=0.49\linewidth]{final_fit_mva/ranking/pulls_SigXsecOverSM_prefit_125_NOSignalUnc.pdf} \\
  \includegraphics[width=0.75\linewidth]{final_fit_mva/ranking/pulls_SigXsecOverSM_125_NOSignalUnc.pdf}
  \caption{The top 15 nuisance parameters are shown, omitting signal
    uncertainties, ranked based on their impact on the \VHbb\ signal strength
    $\mu$ as determined by the combined unconditional fit to data.}
  \label{fig:Rank_012L_MVAVH_nosig}
\end{figure}

The rankings are determined by change in $\mu$ resulting from shifting a given
nuisance parameter. The nuisance parameter in question is moved to its up and
down variation, based on post-fit uncertainty, represented by the blue boxes
that are empty and hatched respectively. The variations based on the pre-fit
uncertainty are represented by the yellow area.

It can be seen that the ranking is dominated by uncertainties on the signal. The
post-fit ranking which omits signal uncertainties is largely dominated by
flavour tagging uncertainties and scale factors for normalisation and acceptance
between analysis regions. The highest ranked uncertainty in both cases is the
$b$-tagging efficiency. This is unsurprising as $b$-tagging plays a crucial role
in the analysis selection. Notably the $Z+$jets shape uncertainties do not
appear in the top sources of uncertainty. Naively this is unexpected as the
background dominates in a sensitive channel of the analysis, however, other
modelling uncertainties which are described with multi-variate techniques have a
greater impact.

\section{Post-fit Data Versus Predictions}
Figures~\ref{fig:0lep2jet-postfit}, \ref{fig:0lep3jet-postfit},
\ref{fig:1lep2jet-postfit}, \ref{fig:1lep3jet-postfit},
\ref{fig:2lep2jet-postfit}, \ref{fig:2lep3pjet-postfit} show the post-fit data
versus prediction comparisons for each of the analysis channel and jet
multiplicity combinations. Each figure shows the individual analysis regions
based on $\Delta R(b, b)$ region and $p_{\mathrm{T}}^V$ bin. Across the board
good agreement between the data and prediction can be seen, supporting the
conclusion that our uncertainty model covers all of the differences between the
two datasets and the supporting the background plus signal hypothesis.
\begin{figure}
  \centering
  \begin{tabular}{cc}
    % top row
    \includegraphics[width=.3\textwidth]{final_fit_mva/postfit/Region_BMax250_BMin150_Y6051_DCRHigh_{\mathrm{T}}2_L0_distMET_J2_GlobalFit_unconditionnal_mu1}%
    & \includegraphics[width=.3\textwidth]{final_fit_mva/postfit/Region_BMin250_Y6051_DCRHigh_{\mathrm{T}}2_L0_distMET_J2_GlobalFit_unconditionnal_mu1} \\

    % middle row
    \includegraphics[width=.3\textwidth]{final_fit_mva/postfit/Region_BMax250_BMin150_Y6051_DSR_{\mathrm{T}}2_L0_distmva_J2_GlobalFit_unconditionnal_mu1}%
    & \includegraphics[width=.3\textwidth]{final_fit_mva/postfit/Region_BMin250_Y6051_DSR_{\mathrm{T}}2_L0_distmva_J2_GlobalFit_unconditionnal_mu1} \\

    % bottom row
    \includegraphics[width=.3\textwidth]{final_fit_mva/postfit/Region_BMax250_BMin150_Y6051_DCRLow_{\mathrm{T}}2_L0_distMET_J2_GlobalFit_unconditionnal_mu1}%
    & \includegraphics[width=.3\textwidth]{final_fit_mva/postfit/Region_BMin250_Y6051_DCRLow_{\mathrm{T}}2_L0_distMET_J2_GlobalFit_unconditionnal_mu1} \\
  \end{tabular}
  \caption{Post-fit distributions in the 0--lepton 2--jet channel.}
\end{figure}
\begin{figure}
  \centering
  \begin{tabular}{cc}
    % top row
    \includegraphics[width=.3\textwidth]{final_fit_mva/postfit/Region_BMax250_BMin150_Y6051_DCRHigh_T2_L0_distMET_J3_GlobalFit_unconditionnal_mu1}%
    & \includegraphics[width=.3\textwidth]{final_fit_mva/postfit/Region_BMin250_Y6051_DCRHigh_T2_L0_distMET_J3_GlobalFit_unconditionnal_mu1} \\

    % middle row
    \includegraphics[width=.3\textwidth]{final_fit_mva/postfit/Region_BMax250_BMin150_Y6051_DSR_T2_L0_distmva_J3_GlobalFit_unconditionnal_mu1}%
    & \includegraphics[width=.3\textwidth]{final_fit_mva/postfit/Region_BMin250_Y6051_DSR_T2_L0_distmva_J3_GlobalFit_unconditionnal_mu1} \\

    % bottom row
    \includegraphics[width=.3\textwidth]{final_fit_mva/postfit/Region_BMax250_BMin150_Y6051_DCRLow_T2_L0_distMET_J3_GlobalFit_unconditionnal_mu1}%
    & \includegraphics[width=.3\textwidth]{final_fit_mva/postfit/Region_BMin250_Y6051_DCRLow_T2_L0_distMET_J3_GlobalFit_unconditionnal_mu1} \\
  \end{tabular}
  \caption{Post-fit distributions in the 0--lepton 3--jet channel.}
\end{figure}
\begin{figure}
  \centering
  \begin{tabular}{cc}
    % top row
    %\includegraphics[width=.3\textwidth]{final_fit_mva/postfit/Region_BMin0_Y6051_DCRHigh_T2_L1_distMET_J2_GlobalFit_unconditionnal_mu1}%
    \includegraphics[width=.3\textwidth]{final_fit_mva/postfit/Region_BMax250_BMin150_Y6051_DCRHigh_T2_L1_distpTV_J2_GlobalFit_unconditionnal_mu1}%
    & \includegraphics[width=.3\textwidth]{final_fit_mva/postfit/Region_BMin250_Y6051_DCRHigh_T2_L1_distpTV_J2_GlobalFit_unconditionnal_mu1} \\

    % middle row
    %\includegraphics[width=.3\textwidth]{}%
    \includegraphics[width=.3\textwidth]{final_fit_mva/postfit/Region_BMax250_BMin150_Y6051_DSR_T2_L1_distmva_J2_GlobalFit_unconditionnal_mu1}%
    & \includegraphics[width=.3\textwidth]{final_fit_mva/postfit/Region_BMin250_Y6051_DSR_T2_L1_distmva_J2_GlobalFit_unconditionnal_mu1} \\

    % bottom row
    %\includegraphics[width=.3\textwidth]{it}%
    \includegraphics[width=.3\textwidth]{final_fit_mva/postfit/Region_BMax250_BMin150_Y6051_DCRLow_T2_L1_distpTV_J2_GlobalFit_unconditionnal_mu1}%
    & \includegraphics[width=.3\textwidth]{final_fit_mva/postfit/Region_BMin250_Y6051_DCRLow_T2_L1_distpTV_J2_GlobalFit_unconditionnal_mu1} \\
    
    % (a) first & (b) second \\[6pt]
    %(c) third & (d) fourth \\[6pt]
    %\multicolumn{2}{c}{\includegraphics[width=65mm]{it} }\\
    %\multicolumn{2}{c}{(e) fifth}
  \end{tabular}
  \caption{caption}
\end{figure}
\begin{figure}
  \centering
  \begin{tabular}{cc}
    % top row
    \includegraphics[width=.3\textwidth]{final_fit_mva/postfit/Region_BMax250_BMin150_Y6051_DCRHigh_T2_L1_distpTV_J3_GlobalFit_unconditionnal_mu1}%
    & \includegraphics[width=.3\textwidth]{final_fit_mva/postfit/Region_BMin250_Y6051_DCRHigh_T2_L1_distpTV_J3_GlobalFit_unconditionnal_mu1} \\

    % middle row
    \includegraphics[width=.3\textwidth]{final_fit_mva/postfit/Region_BMax250_BMin150_Y6051_DSR_T2_L1_distmva_J3_GlobalFit_unconditionnal_mu1}%
    & \includegraphics[width=.3\textwidth]{final_fit_mva/postfit/Region_BMin250_Y6051_DSR_T2_L1_distmva_J3_GlobalFit_unconditionnal_mu1} \\

    % bottom row
    \includegraphics[width=.3\textwidth]{final_fit_mva/postfit/Region_BMax250_BMin150_Y6051_DCRLow_T2_L1_distpTV_J3_GlobalFit_unconditionnal_mu1}%
    & \includegraphics[width=.3\textwidth]{final_fit_mva/postfit/Region_BMin250_Y6051_DCRLow_T2_L1_distpTV_J3_GlobalFit_unconditionnal_mu1} \\
  \end{tabular}
  \caption{Post-fit distributions in the 1 lepton 3 jet channel.}
\end{figure}
\begin{figure}
  \centering
  \begin{tabular}{cc}
    % top row
    \includegraphics[width=.33\textwidth]{final_fit_mva/postfit/Region_BMax150_BMin75_Y6051_DCRHigh_T2_L2_distpTV_J2_GlobalFit_unconditionnal_mu1}%
    \includegraphics[width=.33\textwidth]{final_fit_mva/postfit/Region_BMax250_BMin150_Y6051_DCRHigh_T2_L2_distpTV_J2_GlobalFit_unconditionnal_mu1}%
    & \includegraphics[width=.33\textwidth]{final_fit_mva/postfit/Region_BMin250_Y6051_DCRHigh_T2_L2_distpTV_J2_GlobalFit_unconditionnal_mu1} \\

    % middle row
    \includegraphics[width=.33\textwidth]{final_fit_mva/postfit/Region_BMax150_BMin75_Y6051_DSR_T2_L2_distmva_J2_GlobalFit_unconditionnal_mu1}%
    \includegraphics[width=.33\textwidth]{final_fit_mva/postfit/Region_BMax250_BMin150_Y6051_DSR_T2_L2_distmva_J2_GlobalFit_unconditionnal_mu1}%
    & \includegraphics[width=.33\textwidth]{final_fit_mva/postfit/Region_BMin250_Y6051_DSR_T2_L2_distmva_J2_GlobalFit_unconditionnal_mu1} \\

    % bottom row
    \includegraphics[width=.33\textwidth]{final_fit_mva/postfit/Region_BMax150_BMin75_Y6051_DCRLow_T2_L2_distpTV_J2_GlobalFit_unconditionnal_mu1}%
    \includegraphics[width=.33\textwidth]{final_fit_mva/postfit/Region_BMax250_BMin150_Y6051_DCRLow_T2_L2_distpTV_J2_GlobalFit_unconditionnal_mu1}%
    & \includegraphics[width=.33\textwidth]{final_fit_mva/postfit/Region_BMin250_Y6051_DCRLow_T2_L2_distpTV_J2_GlobalFit_unconditionnal_mu1} \\
  \end{tabular}
  \caption{Post-fit distributions in the 2--lepton 2--jet channel.}
\end{figure}
\begin{figure}
  \centering
  \begin{tabular}{cc}
    % top row
    \includegraphics[width=.3\textwidth]{final_fit_mva/postfit/Region_BMax150_BMin75_incJet1_Y6051_DCRHigh_T2_L2_distpTV_J3_GlobalFit_unconditionnal_mu1}%
    \includegraphics[width=.3\textwidth]{final_fit_mva/postfit/Region_BMax250_BMin150_incJet1_Y6051_DCRHigh_T2_L2_distpTV_J3_GlobalFit_unconditionnal_mu1}%
    & \includegraphics[width=.3\textwidth]{final_fit_mva/postfit/Region_BMin250_incJet1_Y6051_DCRHigh_T2_L2_distpTV_J3_GlobalFit_unconditionnal_mu1} \\

    % middle row
    \includegraphics[width=.3\textwidth]{final_fit_mva/postfit/Region_BMax150_BMin75_incJet1_Y6051_DSR_T2_L2_distmva_J3_GlobalFit_unconditionnal_mu1}%
    \includegraphics[width=.3\textwidth]{final_fit_mva/postfit/Region_BMax250_BMin150_incJet1_Y6051_DSR_T2_L2_distmva_J3_GlobalFit_unconditionnal_mu1}%
    & \includegraphics[width=.3\textwidth]{final_fit_mva/postfit/Region_BMin250_incJet1_Y6051_DSR_T2_L2_distmva_J3_GlobalFit_unconditionnal_mu1} \\

    % bottom row
    \includegraphics[width=.3\textwidth]{final_fit_mva/postfit/Region_BMax150_BMin75_incJet1_Y6051_DCRLow_T2_L2_distpTV_J3_GlobalFit_unconditionnal_mu1}%
    \includegraphics[width=.3\textwidth]{final_fit_mva/postfit/Region_BMax250_BMin150_incJet1_Y6051_DCRLow_T2_L2_distpTV_J3_GlobalFit_unconditionnal_mu1}%
    & \includegraphics[width=.3\textwidth]{final_fit_mva/postfit/Region_BMin250_incJet1_Y6051_DCRLow_T2_L2_distpTV_J3_GlobalFit_unconditionnal_mu1} \\
  \end{tabular}
  \caption{Post-fit distributions in the 2 lepton, 3+ jets channel.}
\end{figure}

\clearpage

\section{Signal Strength and STXS Measurements}
In this section the signal strength and STXS measurements will be detailed.

\subsection{\VH\ Signal Strength Measurement}
Table~\ref{tab:sig_channels} shows the expected and observed significances for
the background plus signal hypothesis broken down by analysis channel.
\begin{table}[h]
  \centering
  \begin{tabular}{lrr}
    \toprule
    {\bfseries Channel}  & {\bfseries Expected sig. (cond. data)} & {\bfseries Observed sig.} \\
    \midrule
    0L         & 4.1               &	4.4	    \\
    1L         & 4.1               &	4.0	    \\
    2L         & 4.3               &	4.2	    \\
    0L+1L+2L   & 6.7               &	6.7	    \\
    \bottomrule
  \end{tabular}
  \caption{caption}
  % Expected significances for a single signal strength, or three
  % corresponding to the three leptonic channel (3-$\mu$ fit), in the combined
  % fits of the mva discriminant in the three channels to the cond. data Asimov
  % dataset (data fit with $\mu=1$) and to Run 2 data.
  \label{tab:sig_channels}
\end{table}
The expected significances were determined using an Asimov dataset conditional
on $\mu=1$. All channels are approaching the 5$\sigma$ threshold to claim
discovery and the combined fit is well above the discovery threshold, in
agreement with the previous result~\cite{vhbb-obs}. Figure~\ref{fig:channel-mus}
shows the best fit values for the signal strength in each channel and in the
combined paradigm.
\begin{figure}[ht]
	\centering
	\includegraphics[width=0.7\linewidth]{final_fit_mva/results/Plot_mu_0_DecorrPOI_VH.pdf}
  \caption[Best fit values for signal strength broken down by analysis
  channel.]{Best fit values of the signal strength for the 0--, 1-- and 2--
    channels, as well as the combined (Comb.) measurement. Uncertainties are
    broken down into either statistical (Stat.) or systematic (Syst.), with
    those due to scale factors on background processes counted as statistical.}
  \label{fig:channel-mus}
\end{figure}
All measurements are compatible with the standard model prediction of $\mu=1$.
The composition and values of the uncertainties affecting each channel are
similar with the 2--lepton channel being slightly more impacted by statistical
rather than systematic uncertainty. The measurement of \VH signal strength is
also compatible with the CMS Collaboration result of $1.01 \pm
0.22$~\cite{CMS:hbb}, corresponding measurements of the \WH and \ZH processes as
well as the cross-section have not yet been published. 
% \begin{figure}[!htbp]
	\centering
	\includegraphics[width=0.47\linewidth]{final_fit_mva/results/Plot_mu_1_DecorrPOI_J_VH.pdf}
	\includegraphics[width=0.47\linewidth]{final_fit_mva/results/Plot_mu_0_DecorrBMin_VH.pdf}
	\includegraphics[width=0.47\linewidth]{final_fit_mva/results/Plot_mu_15_DecorrPOI_J_L_VH.pdf}
	\includegraphics[width=0.47\linewidth]{final_fit_mva/results/Plot_mu_15_DecorrPOI_BMin_L_VH.pdf}
  \caption{caption}
  % Best values of the signal strength $\mu_{\VH}^{b\bar{b}}$ uncorrelated
  % (multi-$\mu$) between $N($jet$)$ regions (top left), \pTV regions (top right),
  % $N($jet$)$ regions and lepton channels (bottom left), \pTV regions and lepton
  % channels (bottom right) and their combination in the combined unconditional
  % fit to the Run 2 data. The (black) total observed uncertainty is quoted
  % together with its decomposition in the (green) statistical component, and
  % systematic component. In this plot the uncertainty due to background scale
  % factors is included in the statistical component.
\label{fig:channels-mus-uncorr}
\end{figure}
\clearpage
\subsection{\WH\ and \ZH\ Signal Strength Measurements}
Table~\ref{tab:sig_WZH} shows the expected and observed significances for
the background plus signal hypothesis for the \WH\, \ZH\ and \VH measurements.
\begin{table}[ht]
  \centering
  \begin{tabular}{lrr}
    \toprule
    {\bfseries Channel}  & {\bfseries Expected} & {\bfseries Observed} \\
    \midrule
    \WH         & 4.1       &	4.0	\\
    \ZH         & 5.1       &	5.3	    \\
    \VH         & 6.7      	&	6.7	    \\
    \bottomrule
  \end{tabular}
  \caption{Statistical significances for the background plus signal hypothesis
    for an Asimov dataset conditional on $\mu=1$, called expected and for the
    data, called observed. All values are given standard deviations
    ($\sigma$s). Significances are shown the \WH\, \ZH\ and \VH measurements.}
  \label{tab:sig_WZH}
\end{table}
The expected significances were determined using an Asimov dataset conditional
on $\mu=1$. The \WH\ stand alone measurement is approaching the 5$\sigma$
threshold to claim discovery. The \ZH\ stand alone measurement has surpassed the
threshold and the paper associated with this measurement is first proof of
discovery~\cite{final-paper}. Figure~\ref{fig:WZH-mus} shows the best fit values
for the signal strength for each measurement.
\begin{figure}[hb]
	\centering
	\includegraphics[width=0.7\linewidth]{final_fit_mva/results/Plot_mu_1_STXSFitScheme2_VH.pdf}
  \caption{caption}
  % Best values of the signal strengths associated to the $WH$ and $ZH$
  % production modes and their combination in the combined fit on the
  % three channels to the Run 2 data. The (black) total observed
  % uncertainty is quoted together with its decomposition in the (green)
  % statistical component, and systematic component. In this plot the
  % uncertainty due to background scale factors is included in the
  % statistical component.}
  \label{fig:WZH-mus}
\end{figure}
All measurements are compatible with the standard model prediction of $\mu=1$,
the composition and values of the uncertainties affecting each channel are very
similar.
\clearpage

\subsection{Simplified Template Cross-section Measurements}
Table~\ref{tab:sig_stxs} shows the expected and observed significances for
the \VHbb\ cross-section measurement in the each STXS bin. All bins have an
observed statistical significance in line with what was expected.
Figure~\ref{fig:5POI-mus} shows the best fit values for the cross-section times
branching ratio ($\sigma \times B$) in each of the STXS bins.
\begin{table}[h]
  \centering
  \begin{tabular}{lrr}
    \toprule
    {\bfseries Channel}  & {\bfseries Expected} & {\bfseries Observed} \\
    \midrule
    \WH, $150~\GeV<p_{\mathrm{T}}^{\text{miss}}<250~\GeV$     & 2.0               &	1.6	    \\
    \WH, $p_{\mathrm{T}}^{\text{miss}} > 250~\GeV$            & 3.4               &	3.6	    \\
    \ZH, $75~\GeV<p_{\mathrm{T}}^{\text{miss}}<150~\GeV$      & 1.4               &	1.2	    \\
    \ZH, $150~\GeV<p_{\mathrm{T}}^{\text{miss}}<250~\GeV$     & 3.4               &	3.6	    \\
    \ZH, $p_{\mathrm{T}}^{\text{miss}} > 250~\GeV$            & 3.5               &	3.6	    \\
    \VH                                       & 6.7               &	6.7	    \\
    \bottomrule
  \end{tabular}
  \caption{Statistical significances on the \VHbb\ cross-section measurement in
    each STXS bin for an Asimov dataset conditional on $\mu=1$, called expected
    and for the data, called observed. All values are given standard deviations
    ($\sigma$s). Significances are shown alongside the result from the single
    parameter combined fit, denoted \VH.}
  \label{tab:sig_stxs}
\end{table}
All measurements are compatible with the standard model predictions and are
weakly limited by statistical uncertainty. The theoretical uncertainty on the
standard  model prediction is greater for \ZH\ bins than for \WH\ bins.
\begin{figure}[hb]
	\centering
	\includegraphics[width=0.7\linewidth]{final_fit_mva/results/Plot_mu_5POI.pdf}
  \caption[Best fit values for cross-section times branching ratio in each STXS
  bin.]{Best fit values of the cross-section times branching ratio ($\sigma
    \times B$) in each of the STXS bins. Uncertainties quoted are broken down by
    source, either statistical (Stat.) or systematic (Syst.). Uncertainties are
    broken down into either statistical (Stat.) or systematic (Syst.), with
    those due to scale factors on background processes counted as statistical.
    The theoretical uncertainty on the SM prediction of is shown in grey.}
  \label{fig:5POI-mus}
\end{figure}
Figure~\ref{fig:5POI-corr} shows the correlation between each STXS bin. It can
be seen that no two bins are strongly correlated with one another. The lack of
correlation is largely due to the analysis categorisations which serve as
independent measurements of each STXS bin.
\begin{figure}[hb]
  \centering
  \includegraphics[width=0.7\linewidth]{final_fit_mva/results/corr_XS_5POI.pdf}
  %% plot already updated to the 5 POI XS
  \caption{caption}
  % Correlations in the measure of the ratio of the
  % $\sigma\times\text{BR}$ over the SM for the $\mathrm{WH}_{150-250}$,
  % $\mathrm{WH}_{\geq 250}$, $\mathrm{ZH}_{75-150}$, $\mathrm{ZH}_{150-250}$,
  % and $\mathrm{ZH}_{\geq 250}$ processes in the unconditional fit to data.
  \label{fig:5POI-corr}
\end{figure}
%\begin{table}[!htbp]
  \setlength{\extrarowheight}{4pt}
  \centering
    \resizebox{\textwidth}{!}{
      \begin{tabular}{lrrrrrr}
        \toprule
        Measurement region & SM prediction fb  & Result fb & Stat. unc. fb & \multicolumn{3}{c}{Syst. unc. fb} \\
                           &                   &           &               & Th. sig. & Th. bkg.  & Exp. \\
        \midrule
        $W\rightarrow \ell\nu$; $150<p_{\text{T}}^{V}<250$~\GeV\         & $24.0 \pm 1.1$ & $19.0^{+12.2}_{-12.0} $ & $^{+7.7}_{-7.7}$ & $^{+1.2}_{-0.5}$ & $^{+5.5}_{-5.5}$  & $^{+6.0}_{-6.0}$ \\
        $W\rightarrow \ell\nu$; $p_{\text{T}}^{V}>250$~\GeV\             & $7.1 \pm 0.3$ & $7.2^{+2.3}_{-2.1}  $   & $^{+1.9}_{-1.8}$  & $^{+0.4}_{-0.3}$ & $^{+0.8}_{-0.8}$  & $^{+0.7}_{-0.6}$ \\
        $Z\rightarrow \ell\ell,\nu\nu$; $75<p_{\text{T}}^{V}<150$~\GeV\     & $50.6 \pm 4.1$ & $42.5^{+36.4}_{-35.4}$   & $^{+25.3}_{-25.3}$   &  $^{+6.6}_{-4.6}$ & $^{+17.2}_{-17.2}$ & $^{+20.2}_{-19.2}$ \\
        $Z\rightarrow \ell\ell,\nu\nu$; $150<p_{\text{T}}^{V}<250$~\GeV\    & $18.8 \pm 2.4$        & $20.5^{+6.4}_{-6.0}$   & $^{+5.1}_{-4.9}$   & $^{+2.3}_{-2.3}$ &  $^{+2.4}_{-2.3}$ & $^{+2.4}_{-2.1}$ \\
        $Z\rightarrow \ell\ell,\nu\nu$; $p_{\text{T}}^{V}>250$~\GeV\        & $4.9 \pm 0.5$        & $5.4^{+1.7}_{-1.6} $   & $^{+1.5}_{-1.4}$    & $^{+0.6}_{-0.3}$  & $^{+0.5}_{-0.5}$ & $^{+0.3}_{-0.3}$ \\
        \bottomrule
      \end{tabular}
    }
    \caption{caption, ($|y_H|<2.5$, $H\to b\bar{b}$)}
    % Best-fit values and uncertainties for the $VH$,
    % $V\to$~leptons for the cross-section
    % times the $H\to b\bar{b}$ branching ratio, in the reduced STXS scheme.
    % The SM predictions for each region, computed using the inclusive
    % cross-section
    % calculations and the simulated event samples are also shown.
    % The contributions to the total uncertainty in the measurements from
    % statistical (Stat.~unc.) or systematic uncertainties (Syst.~unc.) in
    % the signal modelling (Th.~sig.), background modelling (Th.~bkg.),
    % and in experimental performance (Exp.) are given separately.
    % The total systematic uncertainty, equal to the difference in quadrature between
    % the total uncertainty and the statistical uncertainty, differs from the
    % sum in quadrature of the Th. Sig., Th. Bkg., and Exp. systematic
    % uncertainties due to correlations.
    % All leptonic decays of the $V$ bosons (including those to $\tau$-leptons, $\ell=e,\mu,\tau$) are considered.
    \label{tab:XS3and5POI}
\end{table}
%\begin{figure}[!htbp]
	\centering
	\includegraphics[width=0.7\linewidth]{final_fit_mva/results/Fancy_SM_5POI.pdf}
  \caption{caption}
  % Best values of the $\sigma\times \text{BR}$ for
  % $\mathrm{WH}_{150-250}$, $\mathrm{WH}_{\geq 250}$,
  % $\mathrm{ZH}_{75-150}$, $\mathrm{ZH}_{150-250}$, and
  % $\mathrm{ZH}_{\geq 250}$ in the unconditional fit to data. The
  % (black) total observed uncertainty is quoted together with the
  % (green) statistical component. The observation is compared to the
  % SM prediction, marked with the red lines, and its uncertainty,
  % represented by the red area.
  \label{fig:5POI-fancyMU}
\end{figure}
