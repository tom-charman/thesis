\chapter{Conclusion}%
\label{ch:conclusion}
The \VHbb\ analysis has been presented with its many categorisations and
carefully chosen techniques to maximise the signal sensitivity and robustness of
the analysis. The analysis used 139~fb$^{-1}$ of data from proton-proton
collisions at a centre of mass energy of $\sqrt{S} = 13~\TeV$ recorded by the
ATLAS detector and provided by the LHC. Measurements of the \WH, \ZH, and \VH\
signal strengths have been made along with cross-section measurements using the
so-called simplified template cross-section paradigm.

The results of the measurements of signal strengths agree with the Standard
Model predictions within the uncertainties. Signal strengths are measured to be
$\mu_{W\!H}^{bb} = 0.95^{+0.27}_{-0.25}$, $\mu_{Z\!H}^{bb} =
1.08^{+0.25}_{-0.23}$, and $\mu_{V\!H}^{bb} = 1.02^{+0.18}_{-0.17}$ normalised
to the Standard Model prediction and for a $m_H = 125~\GeV$.

Cross-sections were measured for \WH\ and \ZH. For both processes measurements
were made in a number of different $p_{\mathrm{T}}^{V}$ bins, namely the
75--150~\GeV, 150--250~\GeV\ and >250~\GeV\ bins where the first bin includes
only a \ZH\ measurement. A measurement of \WH\ in the lowest
$p_{\mathrm{T}}^{V}$ bin would be desirable and could have been achieved with
the inclusion of a 75--150~\GeV\ bin in the 1--lepton channel. Such an inclusion
was planned but had to be dropped from the analysis due to a poor understanding
of the effects of systematic uncertainties in the necessary bin. Results in all
of the STXS bins agree with the Standard Model prediction for cross-section
times branching ratio within the uncertainties.

Modelling of background and signal processes as well as of estimation of
systematic uncertainties has been carried out using a wide variety of
techniques. All checks indicate that the uncertainty model covers all
discrepancies between data and the Monte-Carlo predictions. The breakdown of
uncertainties shows that the precision of the measurement on the signal strength
is limited more by the systematic than the statistical uncertainties. For this
reason it is more important than ever to consider how these uncertainties can be
better understood and reduced.

Efforts to model the $V+$jets and top processes include using a multi-variate
technique, which more completely captures the difference between two datasets
than the traditional univariate approach. Where a univariate approach has
been used, in the $Z+$jets modelling, comparisons have been made to data. These
comparisons to data are favourable to comparing two different Monte-Carlo
predictions as there is no guarantee that the true nature of the data lies
anywhere within the smooth interpolation between the two predictions. Both of
these improvements over the previous iteration of the analysis increase
confidence in the approach to estimating systematic uncertainties which
inherently relies on some prior assumptions and is therefore potentially the
area of the analysis in which confidence is most needed.

It is clear from the ranking of uncertainties that an improvement to the
procedure for $b$-tagging would enhance the precision with which the signal
strength could be measured.

\section{Future studies}%
\label{sec:future}

A combination of the multi-variate and data-driven approaches to modelling would
further enhance confidence in the estimation of $Z+$jets systematic
uncertainties. This would involve using either the $N$-dimensional
parametrisation or the hybrid $N-1$-dimensional parametrisation, discussed in
sections~\ref{sec:ND-reweight} and~\ref{sec:hybrid-reweight}, with the nominal
prediction being trained against the data in the classifier. In order for this
to be possible the gap in the phase space due to the 80--140~\GeV\ veto must be
addressed. A BDT is an inappropriate algorithm for training on data with any
gaps in the input features due to its inherently cut based nature, discussed in
section~\ref{sec:bdts}. Neural networks, discussed in
section~\ref{sec:neural-networks}, are more appropriate for this application as
they can smoothly interpolate along a single dimension of the input space. It
has been shown in the literature that neural networks can be trained to include
a parameter that allows the network to be predictive on datasets that differ
from those in the training by some choice of internal
parameters~\cite{param-hep, param-hep-2}.

Improvements to the truth tagging strategy discussed in
section~\ref{subsec:truth-tagging} could enhance the capabilities of the
analysis due to the key role of $b$-jets in the process being studied. It has
been shown that neural networks can provide a better efficiency map than the
current two dimensional maps that are used~\cite{nn-truth-tagging}, allowing for
better agreement between the truth and direct tagged samples.

Further measurements of the \VHbb\ process are currently underway by members of
the ATLAS collaboration. These measurements aim to combine the event selection
of this analysis, referred to as the resolved analysis (due to the inclusion of
two resolved $b$-jets in each event), with that of a similar analysis which
studies a higher $p_T$ phase space in which the two $b$-jets are merged into a
large radius jet, and a measurement of the \VHcc\ process.