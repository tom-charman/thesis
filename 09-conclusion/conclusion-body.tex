\chapter{Conclusion}%
\label{ch:conclusion}
The \VHbb\ analysis has been presented with it's many categorisations and
carefully chosen techniques to maximise the signal sensitivity and robustness of
the analysis. The results of the analysis agree with the Standard Model
predictions across the board and many efforts have been made to ensure that each
element of the analysis is well understood.

The $V+$jets and top process modelling is performed using a multi-variate
technique which more completely captures the difference between two datasets for
comparison than the traditional univariate approach. Where a univariate
approach has been used, in the $Z+$jets modelling, comparisons have been made to
data. These comparisons to data are favourable to comparing two different
predictions as there is no guarantee that the true nature of the data lies
anywhere within the smooth interpolation between the two predictions. Both of
these improvements over the previous iteration of the analysis increase
confidence in the approach to estimating systematic uncertainties which
inherently relies on some prior assumptions and is therefore potentially the
area of the analysis in which confidence is most needed.


% Conclude that the analysis has many moving parts that effect the result, most of
% these parts are quite mature and so the analysis is general well understood.

% Highlight again the areas that I worked on and their impact on the analysis.
% Make note that Zjets systematics are still significant but have a smaller impact
% on the analysis than in previous rounds due to a number of things... namely a
% higher stats sample in which to derive the uncertainty, a better method for
% deriving the uncertainty, the introduction of cos theta lep in the MVA which
% helps separate Zjets and Higgs events better.

\section{Future studies}%
\label{sec:future}

A combination of the multi-variate and data-driven approaches to modelling would
further enhance confidence in the estimation of $Z+$jets systematic
uncertainties. This would involve using either the $N$-dimensional
parametrisation or the hybrid $N-1$-dimensional parametrisation, discussed in
sections~\ref{sec:ND-reweight} and~\ref{sec:hybrid-reweight}, with the nominal
prediction being trained against the data in the classifier. In order for this
to possible the gap in the phase space due to the 80--140~\GeV\ veto must be
addressed. A BDT is an inappropriate algorithm for training on data with any
gaps in the input features due to it's inherently cut based nature discussed in
section~\ref{sec:bdts}. Neural networks, discussed in
section~\ref{sec:neural-networks} are more appropriate for this application as
they can smoothly interpolate along a single dimension of the input space. It
has been shown in the literature that neural networks can be trained to include
a parameter that allows the network to be predictive on datasets that differ
from those in the training by some choice of internal
parameters~\cite{param-hep, param-hep-2}.



% Refer to the method that can be used to address the gap in the zjets modelling
% phase space caused by the mbb veto and therefore would allow a mutli-variate
% data driven approach, essentially combining the current wjets and zjets
% modelling strategies.

% Refer to systematics that could be improved in order to get a better result.