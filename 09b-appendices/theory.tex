\chapter{Theory Supplement}%
\label{app:theory}

\section{Symmetries}
Gauge group $SU{(3)}_{c} \times SU{(2)}_{L} \times U{(1)}_{Y}$

Global Abelian U (1) transformation

\begin{equation}
  \phi(x) \rightarrow \phi^{\prime}(x) = e^{i\alpha}\phi(x)
  \label{eq:gabu(1)}
\end{equation}

Considering Lagrangian:
\begin{equation}
  \mathscr{L} = \partial_{\mu}\phi^{*}\partial_{\mu}\phi - m^{2}\phi^{*}\phi -
  \frac{\lambda}{2}{(\phi^{*}\phi)}^{2},
  \label{eq:u(1)lag}
\end{equation}
which is invariant under~\ref{eq:gabu(1)}, we can think about the consequences of the this symmetry.
Given that complex conjugation is applied to the field $\phi$ it is implicit that this field can be
expanded as $\phi = \frac{1}{\sqrt{2}}{(\phi_{1} + i\phi_{2})}$ then the middle term of~%
\ref{eq:u(1)lag} can be expanded as
\begin{equation}
  m^{2}\phi^{*}\phi = \frac{m^{2}}{2}{(\phi_{1}^{2} + \phi_{2}^{2})}
  \label{eq:u(1)lag_mid}
\end{equation}
which implies by comparison with the mass term in the real Klein-Gordon equation that $\phi_{1}$
and $\phi_{2}$ have equal mass.

\section{Gauge Bosons and the Higgs}
Starting with a complex doublet
\begin{equation}
  \Phi = \begin{pmatrix} \phi_{1} \\ \phi_{2} \end{pmatrix},
\end{equation}
the Lagrangian invariant under the required symmetry is
\begin{equation}
  \mathscr{L} = -\frac{1}{4}F_{\mu\nu}^{a}F^{a\mu\nu} -\frac{1}{4}B_{\mu\nu}B^{\mu\nu}
  + {(D_{\mu}\Phi)}^{\dagger}D_{\mu}\Phi - \lambda{\bigg(\Phi^{\dagger}\Phi - \frac{v^{2}}{2}\bigg)}^{2},
\end{equation}
with
\begin{align}
  F_{\mu\nu}^{a} = &~\partial_{\mu}A^{a}_{\nu} - \partial_{\nu}A_{\mu}^{a} +
  g\epsilon^{abc}A^{b}_{\mu}A^{c}_{\nu}, \\
  B_{\mu\nu} = &~\partial_{\mu}B_{\nu} - \partial_{\nu}B_{\mu},
\end{align}
and
\begin{equation}
  D_{\mu}\Phi = \partial_{\mu}\Phi -
  i\frac{g}{2}\tau^{a}A^{a}_{\mu}{\big(hi - i\frac{g^{\prime}}{2}B_{\mu}\Phi\big)}
\end{equation}
\section{SM Lagrangian before and after spontaneous symmetry breaking}
Before spontaneous symmetry breaking:
\begin{align}
  \mathscr{L} = &~-\frac{1}{2}\Tr{\big[G_{\mu\nu}G^{\mu\nu}\big]} 
  -\frac{1}{2}\Tr{\big[W_{\mu\nu}W^{\mu\nu}\big]}
  -\frac{1}{2}\Tr{\big[B_{\mu\nu}B^{\mu\nu}\big]} \\
  &~+ \sum_{\psi} i\bar{\psi}\cancel{D}\psi + \sum_{f,g} -G^{fg}_{e}\bar{\chi}^{f}_{L}
  \Phi e^{g}_{R} - G_{D}^{fg}\bar{Q}_{L}^{f}\Phi D^{g}_{R} - G^{fg}_{U}\bar{Q}_{L}^{f}\Phi^{c}U^{g}_{R} \\
  &~+{(D_{\mu}\Phi)}^{\dagger}D_{\mu}\Phi - \lambda{\bigg( \Phi^{\dagger}\Phi -
    \frac{v^{2}}{2} \bigg)}^{2}
\end{align}
where $\psi= \chi_{L}, e_{r}, Q_{L}, U_{R}, D_{R}$ in the sum.

After spontaneous symmetry breaking:
\begin{equation}
  \mathscr{L} = \mathscr{L}_{QCD} + \mathscr{L}_{lept} + \mathscr{L}_{f, EM} + \mathscr{L}_{f, weak}
  + \mathscr{L}_{Y} + \mathscr{L}_{V} + \mathscr{L}_{H} + \mathscr{L}_{VH}
\end{equation}
with:
\begin{align*}
  \mathscr{L}_{QCD}     =&~-\frac{1}{4}G_{\mu\nu}^{a}G^{a\mu\nu} +
  \sum_{quarks}\bar{q}{(i\cancel{\partial}
    - m_{q} - g_{s}\frac{\lambda^{a}}{2}\cancel{G^{a}})}q, \\
  \mathscr{L}_{lept}    =&~\sum_{g=e,\mu,\tau}\bar{e}_{g}{(i\cancel{\partial} - m_{e_{g}})}e_{g}
  + \sum_{g}\bar{\nu}_{g}i\cancel{\partial}P_{L}\nu_{g}, \\
  \mathscr{L}_{f, EM}   =&~ eA_{\mu}\sum_{f}Q_{f}\bar{f}\gamma^{\mu}f, \\
  \mathscr{L}_{f, weak} =&~ \frac{g}{2\sqrt{2}}W_{\mu}\sum_{g}\bar{\nu}_{g}\gamma^{\mu}
  P_{L}e_{g} + h.c. \\
  +&~ \frac{g}{2\sqrt{2}}W_{\mu}\sum_{f,g}\bar{u}_{f}\gamma^{\mu}P_{L}
  V_{fg}d_{g} + h.c. \\
  +&~ \frac{g}{2\cos{\theta_{W}}}Z_{\mu}\sum_{fermions}\bar{f}\gamma^{\mu}
  {(T^{f}_{3}P_{L} - 2Q_{f}\sin^{2}{\theta_{W}})}f,\\
  \mathscr{L}_{Y}      =&~\\
  \mathscr{L}_{V}      =&~\\
  \mathscr{L}_{H}      =&~\\
  \mathscr{L}_{VH}     =&~\\
  \end{align*}
\begin{figure}[ht]
  \centering
  \begin{tikzpicture}
    \begin{feynman}[every dot={/tikz/draw=black}]
      \path (0,0) node[vertex,dot] (o1)%
      node[vertex,right=of o1,dot] (i1)%
      node[vertex,above right=of i1,crossed dot] (it)%
      node[vertex,above=of it,crossed dot] (ot)%
      node[vertex,below right=of it,dot] (i2)%
      node[vertex,below left=of i2, crossed dot] (ib)%
      node[vertex,below=of ib,crossed dot] (ob)%
      node[vertex,right=of i2,dot] (o2);
      
      \diagram* {
        (o1) -- [gluon] (i1),
        (i2) -- [gluon] (o2),
        (o1) -- [fermion, quarter left] (ot),
        (ot) -- [fermion, quarter left] (o2),
        (o2) -- [fermion, quarter left] (ob),
        (ob) -- [fermion, quarter left] (o1),
        (i1) -- [fermion, quarter left] (it),
        (it) -- [fermion, quarter left] (i2),
        (i2) -- [fermion, quarter left] (ib),
        (ib) -- [fermion, quarter left] (i1),
      };
    \end{feynman}
  \end{tikzpicture}%
  \label{fig:closed-loop}
\end{figure}
