\chapter{Truth tagging for pseudo-continuous b-tagging}
\label{app:truth-tag-pcbt}
Truth tagging has been extended to work in the pseudo-continuous tagging
strategy mentioned in section~\ref{sec:jets}. As was explained, truth tagging of
the tagged and non-tagged jets are decided randomly a priori, and the assigned
weight reflects the probability for the jets to be above the tagging threshold.
However, by definition, in pseudo-continuous working point there is no a priori
tagging threshold, which needs to be specified as an additional parameter.

Furthermore, there is a fundamental difference between the definition of
flavour-tagging efficiency in the cumulative and pseudo-continuous working
points. In the first case the efficiency defines the probability for a jet to
have a value \textit{above} the b-tagging requirement, while in the second case
the efficiency quantifies the probability for a jet to fall \textit{in one
  specific} bin of the pseudo-continuous distribution.

 Thus, an additional step is needed to convert this definition of efficiency in
 probability to \textit{pass} a cumulative tagging requirement.  In the
 pseudo-continuous working point the efficiency maps are provided as 3-D maps in
 ($p_{\mathrm{T}}, \eta, \text{weight}$), where weight is the b-tagging information divided
 in 5 bins. The probability for a jet to fall in a particular b-tagging
 bin ($i_{op}$) is
 \begin{equation}
   \text{Eff}_{\text{bin}}^{i_{op}} = \text{MCeff}_{\text{bin}}^{i_{op}} \cdot \text{SF}_{\text{bin}}^{i_{op}},
 \end{equation}
 with $\text{MCeff}_{\text{bin}}^{i_{op}}$ the MC efficiency of that particular
 bin, taken from the efficiency maps, and $\text{SF}_{\text{bin}}^{i_{op}}$ the
 flavour tagging scale factor associated to the bin. In order to recover the
 same definition of efficiency as the cumulative working point, the bins from 1
 to $i_{op}$ needs to be summed together as
 \begin{equation}
   \text{Eff}_{cut}^{i_{op}} = \sum_1^{i_{op}} \text{Eff}_{\text{bin}}^{i_{op}}.
 \end{equation}
 For example, the probability for a jet to pass the 70\% operating point is
 given by the sum of the efficiencies in the 70~--~60~\% and 60~--~0~\% bins,
 each of them corrected for the scale factor of that particular bin taken from
 the pseudo-continuous working point.
 
 The scale factors are defined in a different way with respect to the cumulative
 working point. In practice, they are a non-trivial extension of the efficiency
 and inefficiency scale factors used for the regular calibrations, modified so
 that the tag weight fractions (both in data and in MC) sum up to unity for each
 kinematic bin.