\chapter{BDTr Density Ratio Proof}
\label{app:bdtr-proof}

It is possible to turn the approximation in equation~\ref{eq:bdtr-approximation}
in section~\ref{sec:ND-reweight} into an equivalence. The following is a proof
that his can be done provided that $F$ is a monotonic function. Starting with
the definition of the probability density function
\begin{equation}
  p(s(\vec{x}) | \vec{\theta}_{0}) = \frac{d}{du} \int_{s(\vec{x'}) \leq u} p(\vec{x'} | \vec{\theta}_{0}) d\vec{x'}
  =  \int_{\mathbb{R}^{p}} \delta (u-s(\vec{x'})) p(\vec{x'} | \vec{\theta}_{0}) d\vec{x'}
\end{equation}
it is proven~\cite{Hormander1990} that this can be written as
\begin{equation}
  p(s(\vec{x}) | \vec{\theta}_{0}) = \int_{\vec{x'} \in \Omega_{u}} \frac{1}{|\bigtriangledown s(\vec{x'})|} p(\vec{x'} | \vec{\theta}_{0}) dS_{\vec{x'}},
  \label{eq:CordTransPDF}
\end{equation}
where $|\bigtriangledown s(\vec{x'})| = \sqrt{\sum^{p}_{i}
  |\frac{ds(\vec{x'})}{dx_{i}}|^{2}}$, and $dS_{\vec{x'}}$ is the
Euclidean surface measure of $\Omega_{u}$. The purpose of this coordinate
transformation of, is paired with the fact that as $F(\vec{x'})$ is
monotonic

with $r(\vec{x}; \vec{\theta}_{0}, \vec{\theta}_{1})$, there exists an
invertible function $m$ that satisfies
\begin{equation}
  \frac{p(\vec{x}|\vec{\theta}_{1})}{p(\vec{x}|\vec{\theta}_{0})} = m^{-1}(s(\boldsymbol(x)))
  \label{eq:invertibleFunc}
\end{equation}
Therefore combining equation~\ref{eq:CordTransPDF} and equation
\ref{eq:invertibleFunc}, one can extract the probability density ratio
\begin{equation}
  p(s(\vec{x}) | \vec{\theta}_{0}) = m(s(\vec{x})) \int_{\vec{x'} \in \Omega_{u}} \frac{1}{|\bigtriangledown s(\vec{x'})|} p(\vec{x'} | \vec{\theta}_{1}) dS_{\vec{x'}},
\end{equation}
Performing the same calculation for $p(s(\vec{x}), \vec{\theta}_{1})$
from equation~\ref{eq:CordTransPDF}, it can then be shown that
\begin{equation}
  \frac{p(s(\vec{x}) | \vec{\theta}_{1})}{p(s(\vec{x}) | \vec{\theta}_{0})} = \frac{p(\vec{x}|\vec{\theta}_{1})}{p(\vec{x}|\vec{\theta}_{0})} \frac{\int_{\vec{x'} \in \Omega_{u}} \frac{1}{|\bigtriangledown s(\vec{x'})|} p(\vec{x'} | \vec{\theta}_{1}) dS_{\vec{x'}}}{\int_{\vec{x'} \in \Omega_{u}} \frac{1}{|\bigtriangledown s(\vec{x'})|} p(\vec{x'} | \vec{\theta}_{1}) dS_{\vec{x'}}} =  \frac{p(\vec{x}|\vec{\theta}_{1})}{p(\vec{x}|\vec{\theta}_{0})},
\end{equation}
The significance of the above heuristic proof is that by substituting
$\vec{x} \rightarrow u = s(\vec{x})$, the probability density
ratio in $\mathbb{R}^{p}$ space is equivalent to the dimensionally reduced
$\mathbb{R}$ (1-dimension) space. This means that if one trains a BDT/NN
classifier on two MC models, $\vec{\theta}_{0}$ and $\vec{\theta}_{1}$, then the
ratio of the two MC model probability density functions, $p(s(\vec{x}) |
\vec{\theta}_{0/1})$, as a function of the classifier response can be derived
via the use of a single 1-dimensional binned/continuous function.

