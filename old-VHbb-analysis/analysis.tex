\chapter{Analysis Strategy}%
\label{sec:analysis}

This chapter gives a brief outline of the analysis strategy followed for a result shown at
ICHEP~2018~\cite{ATLAS-CONF-2018-036} henceforth referred to as the ICHEP result. Due to similarities
in the analysis procedures reference will also be made to the 2017 result publish in
JHEP~\cite{Hbb2017:ATLAS} that will be referred to as the JHEP result.

\section{Object reconstruction}

Object reconstruction is largely the same for both the ICHEP and JHEP results.

As mentioned in Chapter~\ref{sec:detector} tracks in the inner detector are used to reconstruct
interaction vertices, the method for this procedure is described in Ref.~\cite{ATLAS:vertex-reco}.
It was also previously mentioned that the interaction vertex associated with the highest energy
hard scattering event is selected as primary however strictly speaking energy of a track is not
measured. The actual procedure is to take the vertex with the highest sum of squared transverse
momenta.

Electrons are muons are both reconstructed under a number of different criteria. These
are referred to as loose or tight based on how stringent the requirements are for identification.
For the electrons deposits of energy in the calorimeters must match a track in the ID, the
parameters on which a loose or tight identification is made are the transverse momentum and
pseudo-rapidity~\cite{ATLAS:electron-reco:2016, ATLAS:electron-reco:2012}. For muons the
procedure~\cite{ATLAS:muon-reco:2016} is similar but instead of deposits of energy in the calorimeters
the requirement is signatures in the muon spectrometers. Only tau particles which decay hadronically
are reconstructed~\cite{ATLAS:tau-reco:2016, ATLAS:tau-reco:2017} so that they are not identified as
jets.

The treatment of jets is complicated and only some of the details are given here. The algorithm used
to reconstruct jets is the anti-$k_t$ algorithm~\cite{anti-kt}, and a tool known as a jet vertex
tagger~\cite{ATLAS:JVT:2016} is used mitigate pollution from jets that come from secondary hard
scatter vertices (known as pile-up vertices). The most obvious reason why jets are very important to
this analysis is that the signature contains two b quarks. Given that all quarks initiate jets in the
detector is important to have some way of identifying jets initiated by a b quark to those initiated
by other flavours. The procedure used to achieve this is known as b-tagging with the MV2c10
multivariate discriminant~\cite{ATLAS:btag:2017} output value used a metric on which a threshold is
decided above which a jet is considered b-tagged. Finally jets are calibrated using information
on the jet energy scale~\cite{ATLAS:JES:2015, ATLAS:JES:2017}, a measure of how well the energy of a
jet can be resolved in the detector.

Missing transverse energy is then calculated~\cite{ATLAS:exMET-reco:2015, ATLAS:MET-reco:2015} based
on the known energy of the initial proton-proton collision and the summed up energy and momenta of the
other reconstructed objects, this includes some objects not mentioned here.

\section{Event Selection}

The full event selection is different for all three channels considered in the analysis and can be
found in Ref.~\cite{ATLAS-CONF-2018-036}. Here a summary of some important details will be given.
All channels require the presence of exactly two b-tagged jets in the event of which the leading
b-tagged jet is required to have transverse momentum > 45 GeV. All channels also
require jet transverse momentum > 20 GeV for |$\eta$| < 2.5 and > 30 GeV for 2.5 < |$\eta$| < 4.5.
The zero lepton channel uses missing transverse energy as a value on which to trigger as does the
muon sub-channel of the one lepton channel whereas the electron sub-channel triggers using the single
lepton trigger. Both of the one lepton sub-channels require a tight lepton of their respective
flavours. Events for all channels may require exactly two jets, though for the zero and one lepton
channels events with exactly three jets are also selected whereas in the two lepton channel any event
with greater than three jets is also selected.

\section{Multivariate analysis}

Multivariate techniques are used to separate selected events into signal or background categories.
There is also a version of the analysis based on making simple threshold cuts on variables however
no details of that analysis will be given here. These algorithms are trained on simulated events which
will be discussed briefly in the next section.

In total when all channels and variations of event selections are considered there are eight signal
regions. These come from two and three jet selections in each of the three channels and additional
two regions which will not be discussed. Boosted decision trees (BDTs) are used as the multivariate
algorithm of choice and the final disciminating variables of the analysis are the outputs of these
algorithms. Two different BDTs are trained one to perform the actual analysis and another to validate
it. The analysis BDT is designed to separate Higgs boson events from all other backgrounds. The second
BDT is used to classify so-called diboson events where a Z boson is produced in association with
another vector boson of any type and then decays into two b quarks. Analysis of this process is chosen
as a validation method due to the similarity yet orthogonality with the signal process. The ICHEP and
JHEP results used the same BDT input variables with minor differences.

\section{Data sets and simulation}

A full table of the generators used to simulate signal and background processes can be found in
Ref.~\cite{ATLAS-CONF-2018-036}. Nothing more will be said here about the choice of these generators
however it should be noted that the choice of tools used simulate events is tightly coupled with the
results of the analysis. The performance of the BDTs discussed in the previous section can only be
trusted if it can also be trusted that the simulated events that they were trained on match up as
expected with events from real data. 

\section{Treatment of systematic uncertainties}

The full treatment of systematic uncertainties can be found in Ref.~\cite{ATLAS-CONF-2018-036},
here only two systematics will be discussed as they related to work completed by the author.

The decay of single top quarks in the $Wt$-channel is a background to the signal process in the one
lepton channel. At next-to-leading order there exists an overlap between a $Wt$ diagram and a
$t\bar{t}$ diagram~\cite{White:single-top} as shown in figure~\ref{fig:single-top-overlap}. This is a
problem as decays of pairs of top quarks ($t\bar{t}$) is also a background to the signal and so
something needs to be done to stop the double counting of events. In order to counteract this problem
two schemes are considered referred to as Diagram Removal and Diagram Subtraction, they are both
discussed in Ref.~\cite{White:single-top}. The differences resulting from the use of the two
schemes mentioned is the largest source of uncertainty on the single top systematic error.

\begin{figure}[ht]
  \centering
  \subfloat{
    \begin{tikzpicture}
      \begin{feynman}
        \vertex (g1);
        \vertex [below=of g1] (g2);
        \vertex [right=of g1] (a);
        \vertex [right=of g2] (b);
        \vertex [right=of a] (t1) {$t$};
        \vertex [right=of b] (t2);
        \vertex [right=of t2] (q) {$\overline{q}$};
        \vertex [above right=of t2] (w) {$W$};
        \vertex [below=of b] (inv);
        
        \diagram* {
          (g1) -- [gluon] (a),
          (g2) -- [gluon] (b),
          (b) -- [fermion] (a),
          (a) -- [fermion] (t1),
          (b) -- [anti fermion, edge label'=$\overline{t}$] (t2),
          (t2) -- [anti fermion] (q),
          (t2) -- [photon] (w),
          (b) -- [white] (inv),
        };
      \end{feynman}
      
    \end{tikzpicture}
  }
  \subfloat{
    \begin{tikzpicture}
      \begin{feynman}
        \vertex (g1);
        \vertex [below right=of g1] (g2);
        \vertex [below left=of g2] (g3);
        \vertex [right=of g2] (tt);
        \vertex [above right=of tt] (t1) {$t$};
        \vertex [below right=of tt] (t2);
        \vertex [above right=of t2] (w) {$W$};
        \vertex [below right=of t2] (q) {$\overline{q}$};
        
        \diagram* {
          (g1) -- [gluon] (g2),
          (g3) -- [gluon] (g2),
          (g2) -- [gluon] (tt),
          (tt) -- [fermion] (t1),
          (tt) -- [anti fermion, edge label'=$\overline{t}$] (t2),
          (t2) -- [anti fermion] (q),
          (t2) -- [photon] (w),
        };
      \end{feynman}
   \end{tikzpicture}
  }
  \subfloat{
    \begin{tikzpicture}
      \begin{feynman}
        \vertex (g1);
        \vertex [below right=of g1] (g2);
        \vertex [below left=of g2] (g3);
        \vertex [right=of g2] (tt);
        \vertex [above right=of tt] (t1) {$t$};
        \vertex [below right=of tt] (t2);
        \vertex [above right=of t2] (w) {$W$};
        \vertex [below right=of t2] (q) {$\overline{q}$};
        
        \diagram* {
          (g1) -- [fermion] (g2),
          (g3) -- [anti fermion] (g2),
          (g2) -- [gluon] (tt),
          (tt) -- [fermion] (t1),
          (tt) -- [anti fermion, edge label'=$\overline{t}$] (t2),
          (t2) -- [anti fermion] (q),
          (t2) -- [photon] (w),
        };
      \end{feynman}
    \end{tikzpicture}
  }
  \caption{Leading order $t\overline{t}$ diagrams that also contribute to
    $Wt$ at next-to-leading order.}%
  \label{fig:single-top-overlap}
\end{figure}

Currently this uncertainty is derived by analysing the normalisation, shapes and acceptance of the
reconstructed di-jet mass $m_{bb}$ and the transverse momentum of the vector boson. A potential
improvement to these uncertainties is to use a bespoke BDT re-weighting algorithm to quantify the
difference between the diagram removal and digram subtraction samples in a higher dimensional phase
space, results of a preliminary study into this technique will be shown in Chapter~\ref{sec:results}.

Another systematic uncertainty comes from diboson production, those events where any combination
of gauge bosons are produced in association with one another. Events where $WW$ only form a tiny
fraction of the total background and so are not that important. The same variables are used as in
the single top procedure to look at normalisation, relative acceptance and shape in order to derive
systematic uncertainties. 
