\usepackage{graphicx}
\usepackage{graphics}
\usepackage{subfig}
\usepackage{verbatim}
\usepackage{latexsym}
\usepackage{amsmath}
\usepackage{amssymb}
\usepackage{mathrsfs}
\usepackage{cancel}
\usepackage{gensymb}
\usepackage{bm}
\usepackage{authblk}
\usepackage{verbatim}
\usepackage{hhline}
\usepackage{booktabs}
%\usepackage{colortbl}
\usepackage{setspace}
\usepackage[sort&compress, numbers, comma]{natbib}
\usepackage[]{appendix}
\usepackage[hyphens]{url}
\usepackage{hyperref}
\usepackage[table]{xcolor}
\usepackage{titlesec}
\usepackage{tikz}
\usepackage{upgreek}
\usepackage[compat=1.1.0]{tikz-feynman}
\usepackage{contour}
\usepackage{pgfplots}
\usepackage[left=2cm,right=2cm,top=2cm,bottom=3cm]{geometry}
%\usepackage{xpatch}
%\makeatletter
%\xpatchcmd\@memb@bchap{\phantomsection}{}{}{\fail}
%\makeatother

\usetikzlibrary{arrows}
\usetikzlibrary{trees}
\usetikzlibrary{decorations.pathmorphing}
\usetikzlibrary{decorations.markings}

\graphicspath{{images/}}

\setlength{\arrayrulewidth}{0.3mm}

\DeclareMathOperator{\Tr}{Tr}

%\titleformat{\chapter}{\normalfont\huge}{\thechapter.}{20pt}{\huge\bf}
%\titleformat{\section}{\normalfont\Large}{\thesection.}{16pt}{\Large\it\bf}
%\titleformat{\subsection}{\normalfont\large}{\thesubsection.}{14pt}{\large\it\bf}
%\titleformat{\subsection}{\normalfont\large}{\thesubsection.}{12pt}{\large\bf}

\hypersetup{
    colorlinks,
    linkcolor={red!50!black},
    citecolor={blue!50!black},
    urlcolor={blue!80!black}
}

%\tikzset{/tikzfeynman/warn luatex=false, /tikzfeynman/warn all=false}

%\tikzset{%
%  every neuron/.style={
%    circle,
%    minimum size=22pt,
%    draw
%  },
%  neuron missing/.style={
%    draw=none, 
%    scale=3,
%    text height=0.3cm,
%    execute at end node=\color{black}\tiny{$\vdots$}
%  },
%}

%\tikzset{
%  photon/.style={decorate, decoration={snake}, draw=black},
%  higgs/.style={draw=black, style=dashed},
%  fermion/.style={draw=black, postaction={decorate},
%    decoration={markings,mark=at position .55 with {\arrow[draw=black]{>}}}},
%  anti-fermion/.style={draw=black, postaction={decorate},
%    decoration={markings,mark=at position .55 with {\arrow[draw=black]{<}}}}}

\pgfplotsset{compat=1.14}

\pgfdeclarefunctionalshading{sphere}{\pgfpoint{-25bp}{-25bp}}{\pgfpoint{25bp}{25bp}}{}{
%% calculate unit coordinates
25 div exch
25 div exch
%% copy stack
2 copy 
%% compute -z^2 of the current position 
dup mul exch
dup mul add
1.0 sub
%% and the -z^2 of the light source 
0.3 dup mul
-0.5 dup mul add
1.0 sub
%% now their sqrt product
mul abs sqrt
%% and the sum product of the rest
exch 0.3 mul add
exch -0.5 mul add
%% max(dotprod,0)
dup abs add 2.0 div 
%% matte-ify
0.6 mul 0.4 add
%% currently there is just one number in the stack.
%% we need three corresponding to the RGB values
dup
0.4
}

\newcolumntype{L}[1]{>{\raggedright\let\newline\\\arraybackslash\hspace{0pt}}m{#1}}
\newcolumntype{C}[1]{>{\centering\let\newline\\\arraybackslash\hspace{0pt}}m{#1}}
\newcolumntype{R}[1]{>{\raggedleft\let\newline\\\arraybackslash\hspace{0pt}}m{#1}}

%\input{template/blocked.sty}
%\input{template/uhead.sty}
%\input{template/boxit.sty}
\pagestyle{empty}

%\setlength{\parskip}{2ex plus 0.5ex minus 0.2ex}
\setlength{\parindent}{12pt}

\makeatletter  %to avoid error messages generated by "\@". Makes Latex treat "@" like a letter

\linespread{1.5}
\def\submitdate#1{\gdef\@submitdate{#1}}

\def\maketitle{
  \begin{titlepage}{
      \setcounter{page}{1}
      \large Particle Physics Research Centre \\
      School of Physics and Astronomy \\
      Queen Mary University of London\\
      \rmfamily
      \vskip 3in
      \Large \bfseries \@title \par
    }
    \vskip 0.3in
    \par
        {\Large \@author}
        \linebreak
        \linebreak
            {\Large Supervisor: Dr. Jonathan Hays}
            \vskip 4in
            \par
            Submitted in partial fulfillment of the requirements of the Degree
            of Doctor of Philosophy \@submitdate.
            \vfil
  \end{titlepage}
}

\def\titlepage{
  \newpage
  \centering
  \linespread{1}
  \normalsize
  \vbox to \vsize\bgroup\vbox to 9in\bgroup
}
\def\endtitlepage{
  \par
  \kern 0pt
  \egroup
  \vss
  \egroup
  \clearpage
}

% \def\abstract{
%   \clearpage
%   \begin{center}{
%       \large\bfseries Abstract}
%   \end{center}
%   \small
%   %\def\baselinestretch{1.5}
%   \linespread{1.5}
%   \normalsize
%   \clearpage
% }
% \def\endabstract{
%   \par
% }

\newenvironment{acknowledgements}{
  \clearpage
  \begin{center}{
      \large \bfseries Acknowledgements}
  \end{center}
  \small
  \linespread{1.5}
  \normalsize
}{\clearpage}
\def\endacknowledgements{
  \par
}

% \newenvironment{dedication}{
%   \clearpage
%   \begin{center}{
%       \large \bfseries Dedication}
%   \end{center}
%   \small
%   \linespread{1.5}
%   \normalsize
% }{\clearpage}
% \def\enddedication{
%   \par
% }


\makeatother  %to avoid error messages generated by "\@". Makes Latex treat "@" like a letter


%\newcommand{\ipc}{{\sf ipc}}

%\newcommand{\Prob}{\bbbp}
%\newcommand{\Real}{\bbbr}
%\newcommand{\real}{\Real}
%\newcommand{\Int}{\bbbz}
%\newcommand{\Nat}{\bbbn}

%\newcommand{\NN}{{\sf I\kern-0.14emN}}   % Natural numbers
%\newcommand{\ZZ}{{\sf Z\kern-0.45emZ}}   % Integers
%\newcommand{\QQQ}{{\sf C\kern-0.48emQ}}   % Rational numbers
%\newcommand{\RR}{{\sf I\kern-0.14emR}}   % Real numbers
%\newcommand{\KK}{{\cal K}}
%\newcommand{\OO}{{\cal O}}
%\newcommand{\AAA}{{\bf A}}
%\newcommand{\HH}{{\bf H}}
%\newcommand{\II}{{\bf I}}
%\newcommand{\LL}{{\bf L}}
%\newcommand{\PP}{{\bf P}}
%\newcommand{\PPprime}{{\bf P'}}
%\newcommand{\QQ}{{\bf Q}}
%\newcommand{\UU}{{\bf U}}
%\newcommand{\UUprime}{{\bf U'}}
%\newcommand{\zzero}{{\bf 0}}
%\newcommand{\ppi}{\mbox{\boldmath $\pi$}}
%\newcommand{\aalph}{\mbox{\boldmath $\alpha$}}
%\newcommand{\bb}{{\bf b}}
%\newcommand{\ee}{{\bf e}}
%\newcommand{\mmu}{\mbox{\boldmath $\mu$}}
%\newcommand{\vv}{{\bf v}}
%\newcommand{\xx}{{\bf x}}
%\newcommand{\yy}{{\bf y}}
%\newcommand{\zz}{{\bf z}}
%\newcommand{\oomeg}{\mbox{\boldmath $\omega$}}
%\newcommand{\res}{{\bf res}}
%\newcommand{\cchi}{{\mbox{\raisebox{.4ex}{$\chi$}}}}
%\newcommand{\cchi}{{\cal X}}
%\newcommand{\cchi}{\mbox{\Large $\chi$}}

% Logical operators and symbols
%\newcommand{\imply}{\Rightarrow}
%\newcommand{\bimply}{\Leftrightarrow}
%\newcommand{\union}{\cup}
%\newcommand{\intersect}{\cap}
%\newcommand{\boolor}{\vee}
%\newcommand{\booland}{\wedge}
%\newcommand{\boolimply}{\imply}
%\newcommand{\boolbimply}{\bimply}
%\newcommand{\boolnot}{\neg}
%\newcommand{\boolsat}{\!\models}
%\newcommand{\boolnsat}{\!\not\models}


%\newcommand{\op}[1]{\mathrm{#1}}
%\newcommand{\s}[1]{\ensuremath{\mathcal #1}}

% Properly styled differentiation and integration operators
%\newcommand{\diff}[1]{\mathrm{\frac{d}{d\mathit{#1}}}}
%\newcommand{\diffII}[1]{\mathrm{\frac{d^2}{d\mathit{#1}^2}}}
%\newcommand{\intg}[4]{\int_{#3}^{#4} #1 \, \mathrm{d}#2}
%\newcommand{\intgd}[4]{\int\!\!\!\!\int_{#4} #1 \, \mathrm{d}#2 \, \mathrm{d}#3}

% Large () brackets on different lines of an eqnarray environment
%\newcommand{\Leftbrace}[1]{\left(\raisebox{0mm}[#1][#1]{}\right.}
%\newcommand{\Rightbrace}[1]{\left.\raisebox{0mm}[#1][#1]{}\right)}

% Funky symobols for footnotes
%\newcommand{\symbolfootnote}{\renewcommand{\thefootnote}{\fnsymbol{footnote}}}
% now add \symbolfootnote to the beginning of the document...

\newcommand{\normallinespacing}{\renewcommand{\baselinestretch}{1.5} \normalsize}
%\newcommand{\mediumlinespacing}{\renewcommand{\baselinestretch}{1.2} \normalsize}
%\newcommand{\narrowlinespacing}{\renewcommand{\baselinestretch}{1.0} \normalsize}
%\newcommand{\bump}{\noalign{\vspace*{\doublerulesep}}}
%\newcommand{\cell}{\multicolumn{1}{}{}}
%\newcommand{\spann}{\mbox{span}}
%\newcommand{\diagg}{\mbox{diag}}
%\newcommand{\modd}{\mbox{mod}}
%\newcommand{\minn}{\mbox{min}}
%\newcommand{\andd}{\mbox{and}}
%\newcommand{\forr}{\mbox{for}}
%\newcommand{\EE}{\mbox{E}}

%\newcommand{\deff}{\stackrel{\mathrm{def}}{=}}
%\newcommand{\syncc}{~\stackrel{\textstyle \rhd\kern-0.57em\lhd}{\scriptstyle L}~}

%\def\coop{\mbox{\large $\rhd\!\!\!\lhd$}}
%\newcommand{\sync}[1]{\raisebox{-1.0ex}{$\;\stackrel{\coop}{\scriptscriptstyle
%#1}\,$}}

%\newtheorem{definition}{Definition}[chapter]
%\newtheorem{theorem}{Theorem}[chapter]

%\newcommand{\Figref}[1]{Figure~\ref{#1}}
%\newcommand{\fig}[3]{
% \begin{figure}[!ht]
% \begin{center}
% \scalebox{#3}{\includegraphics{figs/#1.ps}}
% \vspace{-0.1in}
% \caption[ ]{\label{#1} #2}
% \end{center}
% \end{figure}
%}

%\newcommand{\figtwo}[8]{
% \begin{figure}
% \parbox[b]{#4 \textwidth}{
% \begin{center}
% \scalebox{#3}{\includegraphics{figs/#1.ps}}
% \vspace{-0.1in}
% \caption{\label{#1}#2}
% \end{center}
% }
% \hfill
% \parbox[b]{#8 \textwidth}{
% \begin{center}
% \scalebox{#7}{\includegraphics{figs/#5.ps}}
% \vspace{-0.1in}
% \caption{\label{#5}#6}
% \end{center}
% }
% \end{figure}
%}
